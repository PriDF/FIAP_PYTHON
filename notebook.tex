
% Default to the notebook output style

    


% Inherit from the specified cell style.




    
\documentclass[11pt]{article}

    
    
    \usepackage[T1]{fontenc}
    % Nicer default font (+ math font) than Computer Modern for most use cases
    \usepackage{mathpazo}

    % Basic figure setup, for now with no caption control since it's done
    % automatically by Pandoc (which extracts ![](path) syntax from Markdown).
    \usepackage{graphicx}
    % We will generate all images so they have a width \maxwidth. This means
    % that they will get their normal width if they fit onto the page, but
    % are scaled down if they would overflow the margins.
    \makeatletter
    \def\maxwidth{\ifdim\Gin@nat@width>\linewidth\linewidth
    \else\Gin@nat@width\fi}
    \makeatother
    \let\Oldincludegraphics\includegraphics
    % Set max figure width to be 80% of text width, for now hardcoded.
    \renewcommand{\includegraphics}[1]{\Oldincludegraphics[width=.8\maxwidth]{#1}}
    % Ensure that by default, figures have no caption (until we provide a
    % proper Figure object with a Caption API and a way to capture that
    % in the conversion process - todo).
    \usepackage{caption}
    \DeclareCaptionLabelFormat{nolabel}{}
    \captionsetup{labelformat=nolabel}

    \usepackage{adjustbox} % Used to constrain images to a maximum size 
    \usepackage{xcolor} % Allow colors to be defined
    \usepackage{enumerate} % Needed for markdown enumerations to work
    \usepackage{geometry} % Used to adjust the document margins
    \usepackage{amsmath} % Equations
    \usepackage{amssymb} % Equations
    \usepackage{textcomp} % defines textquotesingle
    % Hack from http://tex.stackexchange.com/a/47451/13684:
    \AtBeginDocument{%
        \def\PYZsq{\textquotesingle}% Upright quotes in Pygmentized code
    }
    \usepackage{upquote} % Upright quotes for verbatim code
    \usepackage{eurosym} % defines \euro
    \usepackage[mathletters]{ucs} % Extended unicode (utf-8) support
    \usepackage[utf8x]{inputenc} % Allow utf-8 characters in the tex document
    \usepackage{fancyvrb} % verbatim replacement that allows latex
    \usepackage{grffile} % extends the file name processing of package graphics 
                         % to support a larger range 
    % The hyperref package gives us a pdf with properly built
    % internal navigation ('pdf bookmarks' for the table of contents,
    % internal cross-reference links, web links for URLs, etc.)
    \usepackage{hyperref}
    \usepackage{longtable} % longtable support required by pandoc >1.10
    \usepackage{booktabs}  % table support for pandoc > 1.12.2
    \usepackage[inline]{enumitem} % IRkernel/repr support (it uses the enumerate* environment)
    \usepackage[normalem]{ulem} % ulem is needed to support strikethroughs (\sout)
                                % normalem makes italics be italics, not underlines
    

    
    
    % Colors for the hyperref package
    \definecolor{urlcolor}{rgb}{0,.145,.698}
    \definecolor{linkcolor}{rgb}{.71,0.21,0.01}
    \definecolor{citecolor}{rgb}{.12,.54,.11}

    % ANSI colors
    \definecolor{ansi-black}{HTML}{3E424D}
    \definecolor{ansi-black-intense}{HTML}{282C36}
    \definecolor{ansi-red}{HTML}{E75C58}
    \definecolor{ansi-red-intense}{HTML}{B22B31}
    \definecolor{ansi-green}{HTML}{00A250}
    \definecolor{ansi-green-intense}{HTML}{007427}
    \definecolor{ansi-yellow}{HTML}{DDB62B}
    \definecolor{ansi-yellow-intense}{HTML}{B27D12}
    \definecolor{ansi-blue}{HTML}{208FFB}
    \definecolor{ansi-blue-intense}{HTML}{0065CA}
    \definecolor{ansi-magenta}{HTML}{D160C4}
    \definecolor{ansi-magenta-intense}{HTML}{A03196}
    \definecolor{ansi-cyan}{HTML}{60C6C8}
    \definecolor{ansi-cyan-intense}{HTML}{258F8F}
    \definecolor{ansi-white}{HTML}{C5C1B4}
    \definecolor{ansi-white-intense}{HTML}{A1A6B2}

    % commands and environments needed by pandoc snippets
    % extracted from the output of `pandoc -s`
    \providecommand{\tightlist}{%
      \setlength{\itemsep}{0pt}\setlength{\parskip}{0pt}}
    \DefineVerbatimEnvironment{Highlighting}{Verbatim}{commandchars=\\\{\}}
    % Add ',fontsize=\small' for more characters per line
    \newenvironment{Shaded}{}{}
    \newcommand{\KeywordTok}[1]{\textcolor[rgb]{0.00,0.44,0.13}{\textbf{{#1}}}}
    \newcommand{\DataTypeTok}[1]{\textcolor[rgb]{0.56,0.13,0.00}{{#1}}}
    \newcommand{\DecValTok}[1]{\textcolor[rgb]{0.25,0.63,0.44}{{#1}}}
    \newcommand{\BaseNTok}[1]{\textcolor[rgb]{0.25,0.63,0.44}{{#1}}}
    \newcommand{\FloatTok}[1]{\textcolor[rgb]{0.25,0.63,0.44}{{#1}}}
    \newcommand{\CharTok}[1]{\textcolor[rgb]{0.25,0.44,0.63}{{#1}}}
    \newcommand{\StringTok}[1]{\textcolor[rgb]{0.25,0.44,0.63}{{#1}}}
    \newcommand{\CommentTok}[1]{\textcolor[rgb]{0.38,0.63,0.69}{\textit{{#1}}}}
    \newcommand{\OtherTok}[1]{\textcolor[rgb]{0.00,0.44,0.13}{{#1}}}
    \newcommand{\AlertTok}[1]{\textcolor[rgb]{1.00,0.00,0.00}{\textbf{{#1}}}}
    \newcommand{\FunctionTok}[1]{\textcolor[rgb]{0.02,0.16,0.49}{{#1}}}
    \newcommand{\RegionMarkerTok}[1]{{#1}}
    \newcommand{\ErrorTok}[1]{\textcolor[rgb]{1.00,0.00,0.00}{\textbf{{#1}}}}
    \newcommand{\NormalTok}[1]{{#1}}
    
    % Additional commands for more recent versions of Pandoc
    \newcommand{\ConstantTok}[1]{\textcolor[rgb]{0.53,0.00,0.00}{{#1}}}
    \newcommand{\SpecialCharTok}[1]{\textcolor[rgb]{0.25,0.44,0.63}{{#1}}}
    \newcommand{\VerbatimStringTok}[1]{\textcolor[rgb]{0.25,0.44,0.63}{{#1}}}
    \newcommand{\SpecialStringTok}[1]{\textcolor[rgb]{0.73,0.40,0.53}{{#1}}}
    \newcommand{\ImportTok}[1]{{#1}}
    \newcommand{\DocumentationTok}[1]{\textcolor[rgb]{0.73,0.13,0.13}{\textit{{#1}}}}
    \newcommand{\AnnotationTok}[1]{\textcolor[rgb]{0.38,0.63,0.69}{\textbf{\textit{{#1}}}}}
    \newcommand{\CommentVarTok}[1]{\textcolor[rgb]{0.38,0.63,0.69}{\textbf{\textit{{#1}}}}}
    \newcommand{\VariableTok}[1]{\textcolor[rgb]{0.10,0.09,0.49}{{#1}}}
    \newcommand{\ControlFlowTok}[1]{\textcolor[rgb]{0.00,0.44,0.13}{\textbf{{#1}}}}
    \newcommand{\OperatorTok}[1]{\textcolor[rgb]{0.40,0.40,0.40}{{#1}}}
    \newcommand{\BuiltInTok}[1]{{#1}}
    \newcommand{\ExtensionTok}[1]{{#1}}
    \newcommand{\PreprocessorTok}[1]{\textcolor[rgb]{0.74,0.48,0.00}{{#1}}}
    \newcommand{\AttributeTok}[1]{\textcolor[rgb]{0.49,0.56,0.16}{{#1}}}
    \newcommand{\InformationTok}[1]{\textcolor[rgb]{0.38,0.63,0.69}{\textbf{\textit{{#1}}}}}
    \newcommand{\WarningTok}[1]{\textcolor[rgb]{0.38,0.63,0.69}{\textbf{\textit{{#1}}}}}
    
    
    % Define a nice break command that doesn't care if a line doesn't already
    % exist.
    \def\br{\hspace*{\fill} \\* }
    % Math Jax compatability definitions
    \def\gt{>}
    \def\lt{<}
    % Document parameters
    \title{automobile-price-prediction}
    
    
    

    % Pygments definitions
    
\makeatletter
\def\PY@reset{\let\PY@it=\relax \let\PY@bf=\relax%
    \let\PY@ul=\relax \let\PY@tc=\relax%
    \let\PY@bc=\relax \let\PY@ff=\relax}
\def\PY@tok#1{\csname PY@tok@#1\endcsname}
\def\PY@toks#1+{\ifx\relax#1\empty\else%
    \PY@tok{#1}\expandafter\PY@toks\fi}
\def\PY@do#1{\PY@bc{\PY@tc{\PY@ul{%
    \PY@it{\PY@bf{\PY@ff{#1}}}}}}}
\def\PY#1#2{\PY@reset\PY@toks#1+\relax+\PY@do{#2}}

\expandafter\def\csname PY@tok@w\endcsname{\def\PY@tc##1{\textcolor[rgb]{0.73,0.73,0.73}{##1}}}
\expandafter\def\csname PY@tok@c\endcsname{\let\PY@it=\textit\def\PY@tc##1{\textcolor[rgb]{0.25,0.50,0.50}{##1}}}
\expandafter\def\csname PY@tok@cp\endcsname{\def\PY@tc##1{\textcolor[rgb]{0.74,0.48,0.00}{##1}}}
\expandafter\def\csname PY@tok@k\endcsname{\let\PY@bf=\textbf\def\PY@tc##1{\textcolor[rgb]{0.00,0.50,0.00}{##1}}}
\expandafter\def\csname PY@tok@kp\endcsname{\def\PY@tc##1{\textcolor[rgb]{0.00,0.50,0.00}{##1}}}
\expandafter\def\csname PY@tok@kt\endcsname{\def\PY@tc##1{\textcolor[rgb]{0.69,0.00,0.25}{##1}}}
\expandafter\def\csname PY@tok@o\endcsname{\def\PY@tc##1{\textcolor[rgb]{0.40,0.40,0.40}{##1}}}
\expandafter\def\csname PY@tok@ow\endcsname{\let\PY@bf=\textbf\def\PY@tc##1{\textcolor[rgb]{0.67,0.13,1.00}{##1}}}
\expandafter\def\csname PY@tok@nb\endcsname{\def\PY@tc##1{\textcolor[rgb]{0.00,0.50,0.00}{##1}}}
\expandafter\def\csname PY@tok@nf\endcsname{\def\PY@tc##1{\textcolor[rgb]{0.00,0.00,1.00}{##1}}}
\expandafter\def\csname PY@tok@nc\endcsname{\let\PY@bf=\textbf\def\PY@tc##1{\textcolor[rgb]{0.00,0.00,1.00}{##1}}}
\expandafter\def\csname PY@tok@nn\endcsname{\let\PY@bf=\textbf\def\PY@tc##1{\textcolor[rgb]{0.00,0.00,1.00}{##1}}}
\expandafter\def\csname PY@tok@ne\endcsname{\let\PY@bf=\textbf\def\PY@tc##1{\textcolor[rgb]{0.82,0.25,0.23}{##1}}}
\expandafter\def\csname PY@tok@nv\endcsname{\def\PY@tc##1{\textcolor[rgb]{0.10,0.09,0.49}{##1}}}
\expandafter\def\csname PY@tok@no\endcsname{\def\PY@tc##1{\textcolor[rgb]{0.53,0.00,0.00}{##1}}}
\expandafter\def\csname PY@tok@nl\endcsname{\def\PY@tc##1{\textcolor[rgb]{0.63,0.63,0.00}{##1}}}
\expandafter\def\csname PY@tok@ni\endcsname{\let\PY@bf=\textbf\def\PY@tc##1{\textcolor[rgb]{0.60,0.60,0.60}{##1}}}
\expandafter\def\csname PY@tok@na\endcsname{\def\PY@tc##1{\textcolor[rgb]{0.49,0.56,0.16}{##1}}}
\expandafter\def\csname PY@tok@nt\endcsname{\let\PY@bf=\textbf\def\PY@tc##1{\textcolor[rgb]{0.00,0.50,0.00}{##1}}}
\expandafter\def\csname PY@tok@nd\endcsname{\def\PY@tc##1{\textcolor[rgb]{0.67,0.13,1.00}{##1}}}
\expandafter\def\csname PY@tok@s\endcsname{\def\PY@tc##1{\textcolor[rgb]{0.73,0.13,0.13}{##1}}}
\expandafter\def\csname PY@tok@sd\endcsname{\let\PY@it=\textit\def\PY@tc##1{\textcolor[rgb]{0.73,0.13,0.13}{##1}}}
\expandafter\def\csname PY@tok@si\endcsname{\let\PY@bf=\textbf\def\PY@tc##1{\textcolor[rgb]{0.73,0.40,0.53}{##1}}}
\expandafter\def\csname PY@tok@se\endcsname{\let\PY@bf=\textbf\def\PY@tc##1{\textcolor[rgb]{0.73,0.40,0.13}{##1}}}
\expandafter\def\csname PY@tok@sr\endcsname{\def\PY@tc##1{\textcolor[rgb]{0.73,0.40,0.53}{##1}}}
\expandafter\def\csname PY@tok@ss\endcsname{\def\PY@tc##1{\textcolor[rgb]{0.10,0.09,0.49}{##1}}}
\expandafter\def\csname PY@tok@sx\endcsname{\def\PY@tc##1{\textcolor[rgb]{0.00,0.50,0.00}{##1}}}
\expandafter\def\csname PY@tok@m\endcsname{\def\PY@tc##1{\textcolor[rgb]{0.40,0.40,0.40}{##1}}}
\expandafter\def\csname PY@tok@gh\endcsname{\let\PY@bf=\textbf\def\PY@tc##1{\textcolor[rgb]{0.00,0.00,0.50}{##1}}}
\expandafter\def\csname PY@tok@gu\endcsname{\let\PY@bf=\textbf\def\PY@tc##1{\textcolor[rgb]{0.50,0.00,0.50}{##1}}}
\expandafter\def\csname PY@tok@gd\endcsname{\def\PY@tc##1{\textcolor[rgb]{0.63,0.00,0.00}{##1}}}
\expandafter\def\csname PY@tok@gi\endcsname{\def\PY@tc##1{\textcolor[rgb]{0.00,0.63,0.00}{##1}}}
\expandafter\def\csname PY@tok@gr\endcsname{\def\PY@tc##1{\textcolor[rgb]{1.00,0.00,0.00}{##1}}}
\expandafter\def\csname PY@tok@ge\endcsname{\let\PY@it=\textit}
\expandafter\def\csname PY@tok@gs\endcsname{\let\PY@bf=\textbf}
\expandafter\def\csname PY@tok@gp\endcsname{\let\PY@bf=\textbf\def\PY@tc##1{\textcolor[rgb]{0.00,0.00,0.50}{##1}}}
\expandafter\def\csname PY@tok@go\endcsname{\def\PY@tc##1{\textcolor[rgb]{0.53,0.53,0.53}{##1}}}
\expandafter\def\csname PY@tok@gt\endcsname{\def\PY@tc##1{\textcolor[rgb]{0.00,0.27,0.87}{##1}}}
\expandafter\def\csname PY@tok@err\endcsname{\def\PY@bc##1{\setlength{\fboxsep}{0pt}\fcolorbox[rgb]{1.00,0.00,0.00}{1,1,1}{\strut ##1}}}
\expandafter\def\csname PY@tok@kc\endcsname{\let\PY@bf=\textbf\def\PY@tc##1{\textcolor[rgb]{0.00,0.50,0.00}{##1}}}
\expandafter\def\csname PY@tok@kd\endcsname{\let\PY@bf=\textbf\def\PY@tc##1{\textcolor[rgb]{0.00,0.50,0.00}{##1}}}
\expandafter\def\csname PY@tok@kn\endcsname{\let\PY@bf=\textbf\def\PY@tc##1{\textcolor[rgb]{0.00,0.50,0.00}{##1}}}
\expandafter\def\csname PY@tok@kr\endcsname{\let\PY@bf=\textbf\def\PY@tc##1{\textcolor[rgb]{0.00,0.50,0.00}{##1}}}
\expandafter\def\csname PY@tok@bp\endcsname{\def\PY@tc##1{\textcolor[rgb]{0.00,0.50,0.00}{##1}}}
\expandafter\def\csname PY@tok@fm\endcsname{\def\PY@tc##1{\textcolor[rgb]{0.00,0.00,1.00}{##1}}}
\expandafter\def\csname PY@tok@vc\endcsname{\def\PY@tc##1{\textcolor[rgb]{0.10,0.09,0.49}{##1}}}
\expandafter\def\csname PY@tok@vg\endcsname{\def\PY@tc##1{\textcolor[rgb]{0.10,0.09,0.49}{##1}}}
\expandafter\def\csname PY@tok@vi\endcsname{\def\PY@tc##1{\textcolor[rgb]{0.10,0.09,0.49}{##1}}}
\expandafter\def\csname PY@tok@vm\endcsname{\def\PY@tc##1{\textcolor[rgb]{0.10,0.09,0.49}{##1}}}
\expandafter\def\csname PY@tok@sa\endcsname{\def\PY@tc##1{\textcolor[rgb]{0.73,0.13,0.13}{##1}}}
\expandafter\def\csname PY@tok@sb\endcsname{\def\PY@tc##1{\textcolor[rgb]{0.73,0.13,0.13}{##1}}}
\expandafter\def\csname PY@tok@sc\endcsname{\def\PY@tc##1{\textcolor[rgb]{0.73,0.13,0.13}{##1}}}
\expandafter\def\csname PY@tok@dl\endcsname{\def\PY@tc##1{\textcolor[rgb]{0.73,0.13,0.13}{##1}}}
\expandafter\def\csname PY@tok@s2\endcsname{\def\PY@tc##1{\textcolor[rgb]{0.73,0.13,0.13}{##1}}}
\expandafter\def\csname PY@tok@sh\endcsname{\def\PY@tc##1{\textcolor[rgb]{0.73,0.13,0.13}{##1}}}
\expandafter\def\csname PY@tok@s1\endcsname{\def\PY@tc##1{\textcolor[rgb]{0.73,0.13,0.13}{##1}}}
\expandafter\def\csname PY@tok@mb\endcsname{\def\PY@tc##1{\textcolor[rgb]{0.40,0.40,0.40}{##1}}}
\expandafter\def\csname PY@tok@mf\endcsname{\def\PY@tc##1{\textcolor[rgb]{0.40,0.40,0.40}{##1}}}
\expandafter\def\csname PY@tok@mh\endcsname{\def\PY@tc##1{\textcolor[rgb]{0.40,0.40,0.40}{##1}}}
\expandafter\def\csname PY@tok@mi\endcsname{\def\PY@tc##1{\textcolor[rgb]{0.40,0.40,0.40}{##1}}}
\expandafter\def\csname PY@tok@il\endcsname{\def\PY@tc##1{\textcolor[rgb]{0.40,0.40,0.40}{##1}}}
\expandafter\def\csname PY@tok@mo\endcsname{\def\PY@tc##1{\textcolor[rgb]{0.40,0.40,0.40}{##1}}}
\expandafter\def\csname PY@tok@ch\endcsname{\let\PY@it=\textit\def\PY@tc##1{\textcolor[rgb]{0.25,0.50,0.50}{##1}}}
\expandafter\def\csname PY@tok@cm\endcsname{\let\PY@it=\textit\def\PY@tc##1{\textcolor[rgb]{0.25,0.50,0.50}{##1}}}
\expandafter\def\csname PY@tok@cpf\endcsname{\let\PY@it=\textit\def\PY@tc##1{\textcolor[rgb]{0.25,0.50,0.50}{##1}}}
\expandafter\def\csname PY@tok@c1\endcsname{\let\PY@it=\textit\def\PY@tc##1{\textcolor[rgb]{0.25,0.50,0.50}{##1}}}
\expandafter\def\csname PY@tok@cs\endcsname{\let\PY@it=\textit\def\PY@tc##1{\textcolor[rgb]{0.25,0.50,0.50}{##1}}}

\def\PYZbs{\char`\\}
\def\PYZus{\char`\_}
\def\PYZob{\char`\{}
\def\PYZcb{\char`\}}
\def\PYZca{\char`\^}
\def\PYZam{\char`\&}
\def\PYZlt{\char`\<}
\def\PYZgt{\char`\>}
\def\PYZsh{\char`\#}
\def\PYZpc{\char`\%}
\def\PYZdl{\char`\$}
\def\PYZhy{\char`\-}
\def\PYZsq{\char`\'}
\def\PYZdq{\char`\"}
\def\PYZti{\char`\~}
% for compatibility with earlier versions
\def\PYZat{@}
\def\PYZlb{[}
\def\PYZrb{]}
\makeatother


    % Exact colors from NB
    \definecolor{incolor}{rgb}{0.0, 0.0, 0.5}
    \definecolor{outcolor}{rgb}{0.545, 0.0, 0.0}



    
    % Prevent overflowing lines due to hard-to-break entities
    \sloppy 
    % Setup hyperref package
    \hypersetup{
      breaklinks=true,  % so long urls are correctly broken across lines
      colorlinks=true,
      urlcolor=urlcolor,
      linkcolor=linkcolor,
      citecolor=citecolor,
      }
    % Slightly bigger margins than the latex defaults
    
    \geometry{verbose,tmargin=1in,bmargin=1in,lmargin=1in,rmargin=1in}
    
    

    \begin{document}
    
    
    \maketitle
    
    

    
    \section{MBA FIAP Inteligência Artificial \& Machine
Learning}\label{mba-fiap-inteliguxeancia-artificial-machine-learning}

\begin{figure}
\centering
\includegraphics{img/ml.png}
\caption{Image of Dementia}
\end{figure}

\subsection{Programando IA com Python}\label{programando-ia-com-python}

\subsection{Projeto Final: Estimando Preços de Automóveis Utilizando
Modelos de
Regressão}\label{projeto-final-estimando-preuxe7os-de-automuxf3veis-utilizando-modelos-de-regressuxe3o}

Este projeto final tem como objetivo explorar os conhecimentos
adquiridos nas aulas práticas. Por meio uma trilha guiada para construir
um classificador que permitirá predizer o valor de um automóvel baseado
em algumas características que cada grupo deverá escolher.

Este projeto poderá ser feita por grupos de até 4 pessoas.

\begin{longtable}[]{@{}llc@{}}
\toprule
Nome dos Integrantes & RM & Turma\tabularnewline
\midrule
\endhead
Cauê Engelmann & RM 331199 & \texttt{2IA}\tabularnewline
Marcelo Gulfier & RM 330738 & \texttt{2IA}\tabularnewline
Marcos Massaharu Muto & RM 330930 & \texttt{2IA}\tabularnewline
Priscila Daniele Fritsch Gonçalves & RM 331893 &
\texttt{2IA}\tabularnewline
\bottomrule
\end{longtable}

Por ser um projeto guiado, fique atento quando houver as marcações
\textbf{Implementação} indica que é necessário realizar alguma
implementação em Python no bloco a seguir onde há a inscrição
\texttt{\#\#IMPLEMENTAR} e \textbf{Resposta} indica que é esperado uma
resposta objetiva relacionado a algum questionamento. Cada grupo pode
utilizar nas respostas objetivas quaisquer itens necessários que
enriqueçam seu ponto vista, como gráficos e, até mesmo, trechos de
código-fonte.

Pode-se utilizar quantos blocos forem necessários para realizar
determinadas implementações ou utilizá-las para justificar as respostas.
Não é obrigatório utilizar somente o bloco indicado.

Ao final não se esqueça de subir os arquivos do projeto nas contas do
GitHub de cada membro, ou subir na do representante do grupo e os
membros realizarem o fork do projeto.

A avaliação terá mais ênfase nos seguintes tópicos de desenvolvimento do
projeto:

\begin{enumerate}
\def\labelenumi{\arabic{enumi}.}
\tightlist
\item
  \textbf{Exploração de Dados}
\item
  \textbf{Preparação de Dados}
\item
  \textbf{Desenvolvimento do Modelo}
\item
  \textbf{Treinamento e Teste do Modelo}
\item
  \textbf{Validação e Otimização do Modelo}
\item
  \textbf{Conclusões Finais}
\end{enumerate}

\subsection{Exploração de Dados}\label{explorauxe7uxe3o-de-dados}

    Os dados que serão utilizados foram modificados para propocionar uma
experiência que explore melhor as técnicas de processamento e preparação
de dados aprendidas.

Originalmente os dados foram extraídos do Kaggle deste
\href{https://www.kaggle.com/nisargpatel/automobiles/data}{dataset}.

    \textbf{Implementação}

Carregue o dataset "automobile-mod.csv" que se encontra na pasta "data"
e faça uma inspeção nas 10 primeiras linhas para identificação básica
dos atributos.

O dataset original "automobile.csv" se encontra na mesma pasta apenas
como referência. Não deverá ser utilizado.

\textbf{Atualizado em 16/07/2017}

    \begin{Verbatim}[commandchars=\\\{\}]
{\color{incolor}In [{\color{incolor}1}]:} \PY{k+kn}{import} \PY{n+nn}{seaborn} \PY{k}{as} \PY{n+nn}{sns}
        \PY{k+kn}{import} \PY{n+nn}{matplotlib}\PY{n+nn}{.}\PY{n+nn}{pyplot} \PY{k}{as} \PY{n+nn}{plt}
        \PY{k+kn}{import} \PY{n+nn}{pandas} \PY{k}{as} \PY{n+nn}{pd}
        \PY{k+kn}{import} \PY{n+nn}{numpy} \PY{k}{as} \PY{n+nn}{np}
        \PY{k+kn}{import} \PY{n+nn}{warnings}
        \PY{n}{warnings}\PY{o}{.}\PY{n}{filterwarnings}\PY{p}{(}\PY{l+s+s1}{\PYZsq{}}\PY{l+s+s1}{ignore}\PY{l+s+s1}{\PYZsq{}}\PY{p}{)}
        
        \PY{o}{\PYZpc{}}\PY{k}{matplotlib} inline
\end{Verbatim}


    \begin{Verbatim}[commandchars=\\\{\}]
{\color{incolor}In [{\color{incolor}2}]:} \PY{n}{automobile} \PY{o}{=} \PY{n}{pd}\PY{o}{.}\PY{n}{read\PYZus{}csv}\PY{p}{(}\PY{l+s+s2}{\PYZdq{}}\PY{l+s+s2}{data/automobile\PYZhy{}mod.csv}\PY{l+s+s2}{\PYZdq{}}\PY{p}{,} \PY{n}{sep}\PY{o}{=}\PY{l+s+s2}{\PYZdq{}}\PY{l+s+s2}{;}\PY{l+s+s2}{\PYZdq{}}\PY{p}{)}
        \PY{n}{automobile}\PY{o}{.}\PY{n}{head}\PY{p}{(}\PY{l+m+mi}{10}\PY{p}{)}
\end{Verbatim}


\begin{Verbatim}[commandchars=\\\{\}]
{\color{outcolor}Out[{\color{outcolor}2}]:}           make fuel\_type aspiration number\_of\_doors   body\_style drive\_wheels  \textbackslash{}
        0  alfa-romero       gas        std             two  convertible          rwd   
        1  alfa-romero       gas        std             two  convertible          rwd   
        2  alfa-romero       gas        std             two    hatchback          rwd   
        3         audi       gas        std            four        sedan          fwd   
        4         audi       gas        std            four        sedan          4wd   
        5         audi       gas        std             two        sedan          fwd   
        6         audi       gas        std            four        sedan          fwd   
        7         audi       gas        std            four        wagon          fwd   
        8         audi       gas      turbo            four        sedan          fwd   
        9          bmw       gas        std             two        sedan          rwd   
        
          engine\_location  wheel\_base  length  width  {\ldots}    engine\_size  fuel\_system  \textbackslash{}
        0           front        88.6   168.8   64.1  {\ldots}            130         mpfi   
        1           front        88.6   168.8   64.1  {\ldots}            130         mpfi   
        2           front        94.5   171.2   65.5  {\ldots}            152         mpfi   
        3           front        99.8   176.6   66.2  {\ldots}            109         mpfi   
        4           front        99.4   176.6   66.4  {\ldots}            136         mpfi   
        5           front        99.8   177.3   66.3  {\ldots}            136         mpfi   
        6           front       105.8   192.7   71.4  {\ldots}            136         mpfi   
        7           front       105.8   192.7   71.4  {\ldots}            136         mpfi   
        8           front       105.8   192.7   71.4  {\ldots}            131         mpfi   
        9           front       101.2   176.8   64.8  {\ldots}            108         mpfi   
        
           bore stroke  compression\_ratio horsepower  peak\_rpm  city\_mpg  highway\_mpg  \textbackslash{}
        0  3.47   2.68                9.0        111    5000.0      21.0         27.0   
        1  3.47   2.68                9.0        111    5000.0      21.0         27.0   
        2  2.68   3.47                9.0        154    5000.0      19.0         26.0   
        3  3.19   3.40               10.0        102    5500.0      24.0         30.0   
        4  3.19   3.40                8.0        115    5500.0      18.0         22.0   
        5  3.19   3.40                8.5        110    5500.0      19.0         25.0   
        6  3.19   3.40                8.5        110    5500.0      19.0         25.0   
        7  3.19   3.40                8.5        110    5500.0      19.0         25.0   
        8  3.13   3.40                8.3        140    5500.0      17.0         20.0   
        9  3.50   2.80                8.8        101    5800.0      23.0         29.0   
        
           price  
        0  13495  
        1  16500  
        2  16500  
        3  13950  
        4  17450  
        5  15250  
        6  17710  
        7  18920  
        8  23875  
        9  16430  
        
        [10 rows x 24 columns]
\end{Verbatim}
            
    Relação das coluna e seu significado:

\begin{enumerate}
\def\labelenumi{\arabic{enumi}.}
\tightlist
\item
  make: fabricante
\item
  fuel\_type: tipo de combustível
\item
  aspiration: tipo de aspiração do motor, ex. turbo ou padrão (std)
\item
  body\_style: estilo do carro, ex. sedan ou hatchback
\item
  drive\_wheels: tração do carro, ex. rwd (tração traseira) ou frw
  (tração dianteira)
\item
  wheel\_base: entre-eixos, distância entre o eixo dianteiro e o eixo
  traseiro
\item
  length: comprimento
\item
  width: largura
\item
  height: altura
\item
  curb\_wheight: peso
\item
  engine\_type: tipo do motor
\item
  number\_of\_cylinders: cilindrada, quantidade de cilindros
\item
  engine\_size: tamanho do motor
\item
  fuel\_system: sistema de injeção
\item
  bore: diâmetro do cilindro
\item
  stroke: diâmetro do pistão
\item
  compression\_ratio: razão de compressão
\item
  horsepower: cavalo de força ou HP
\item
  peak\_rpm: pico de RPM (rotações por minuto)
\item
  city\_mpg: consumo em mpg (milhas por galão) na cidade
\item
  highway\_mpg: consumo em mpg (milhas por galão) na estrada
\item
  price: preço (\textbf{Variável Alvo})
\end{enumerate}

    \subsubsection{Correlacionamento dos
atributos}\label{correlacionamento-dos-atributos}

    Vamos utilizar algumas suposições e validar se elas são verdadeiras, por
exemplo, o preço do carro pode variar com seu consumo, tamanho ou força?
Vamos explorar estas hipósteses ou outras que o grupo julgue relevante.

    \begin{Verbatim}[commandchars=\\\{\}]
{\color{incolor}In [{\color{incolor}3}]:} \PY{n}{auto\PYZus{}correlacao} \PY{o}{=} \PY{n}{automobile}\PY{p}{[}\PY{p}{[}\PY{l+s+s2}{\PYZdq{}}\PY{l+s+s2}{city\PYZus{}mpg}\PY{l+s+s2}{\PYZdq{}}\PY{p}{,} \PY{l+s+s2}{\PYZdq{}}\PY{l+s+s2}{highway\PYZus{}mpg}\PY{l+s+s2}{\PYZdq{}}\PY{p}{,} \PY{l+s+s2}{\PYZdq{}}\PY{l+s+s2}{length}\PY{l+s+s2}{\PYZdq{}}\PY{p}{,} \PY{l+s+s2}{\PYZdq{}}\PY{l+s+s2}{width}\PY{l+s+s2}{\PYZdq{}}\PY{p}{,} \PY{l+s+s2}{\PYZdq{}}\PY{l+s+s2}{height}\PY{l+s+s2}{\PYZdq{}}\PY{p}{,} \PY{l+s+s2}{\PYZdq{}}\PY{l+s+s2}{horsepower}\PY{l+s+s2}{\PYZdq{}}\PY{p}{,} \PY{l+s+s2}{\PYZdq{}}\PY{l+s+s2}{price}\PY{l+s+s2}{\PYZdq{}}\PY{p}{]}\PY{p}{]}\PY{o}{.}\PY{n}{corr}\PY{p}{(}\PY{p}{)}
        \PY{n}{auto\PYZus{}correlacao}
\end{Verbatim}


\begin{Verbatim}[commandchars=\\\{\}]
{\color{outcolor}Out[{\color{outcolor}3}]:}              city\_mpg  highway\_mpg    length     width    height  horsepower  \textbackslash{}
        city\_mpg     1.000000     0.875933 -0.633185 -0.620316 -0.069029   -0.763573   
        highway\_mpg  0.875933     1.000000 -0.709308 -0.697742 -0.117146   -0.776634   
        length      -0.633185    -0.709308  1.000000  0.857170  0.492063    0.577923   
        width       -0.620316    -0.697742  0.857170  1.000000  0.306002    0.613488   
        height      -0.069029    -0.117146  0.492063  0.306002  1.000000   -0.085544   
        horsepower  -0.763573    -0.776634  0.577923  0.613488 -0.085544    1.000000   
        price       -0.657661    -0.712812  0.690628  0.751265  0.135486    0.810795   
        
                        price  
        city\_mpg    -0.657661  
        highway\_mpg -0.712812  
        length       0.690628  
        width        0.751265  
        height       0.135486  
        horsepower   0.810795  
        price        1.000000  
\end{Verbatim}
            
    \begin{Verbatim}[commandchars=\\\{\}]
{\color{incolor}In [{\color{incolor}4}]:} \PY{n}{auto\PYZus{}pricey}\PY{o}{=}\PY{n}{automobile}\PY{o}{.}\PY{n}{sort\PYZus{}values}\PY{p}{(}\PY{p}{[}\PY{l+s+s2}{\PYZdq{}}\PY{l+s+s2}{price}\PY{l+s+s2}{\PYZdq{}}\PY{p}{]}\PY{p}{,} \PY{n}{ascending}\PY{o}{=}\PY{k+kc}{False}\PY{p}{)}\PY{o}{.}\PY{n}{head}\PY{p}{(}\PY{l+m+mi}{10}\PY{p}{)}\PY{p}{[}\PY{p}{[}\PY{l+s+s2}{\PYZdq{}}\PY{l+s+s2}{make}\PY{l+s+s2}{\PYZdq{}}\PY{p}{,} \PY{l+s+s2}{\PYZdq{}}\PY{l+s+s2}{price}\PY{l+s+s2}{\PYZdq{}}\PY{p}{]}\PY{p}{]}
        \PY{n}{auto\PYZus{}biggest}\PY{o}{=}\PY{n}{automobile}\PY{o}{.}\PY{n}{sort\PYZus{}values}\PY{p}{(}\PY{p}{[}\PY{l+s+s2}{\PYZdq{}}\PY{l+s+s2}{length}\PY{l+s+s2}{\PYZdq{}}\PY{p}{,} \PY{l+s+s2}{\PYZdq{}}\PY{l+s+s2}{width}\PY{l+s+s2}{\PYZdq{}}\PY{p}{,} \PY{l+s+s2}{\PYZdq{}}\PY{l+s+s2}{height}\PY{l+s+s2}{\PYZdq{}}\PY{p}{]}\PY{p}{,} \PY{n}{ascending}\PY{o}{=}\PY{k+kc}{False}\PY{p}{)}\PY{o}{.}\PY{n}{head}\PY{p}{(}\PY{l+m+mi}{10}\PY{p}{)}\PY{p}{[}\PY{p}{[}\PY{l+s+s2}{\PYZdq{}}\PY{l+s+s2}{length}\PY{l+s+s2}{\PYZdq{}}\PY{p}{,} \PY{l+s+s2}{\PYZdq{}}\PY{l+s+s2}{width}\PY{l+s+s2}{\PYZdq{}}\PY{p}{,} \PY{l+s+s2}{\PYZdq{}}\PY{l+s+s2}{height}\PY{l+s+s2}{\PYZdq{}}\PY{p}{]}\PY{p}{]}
        \PY{n}{auto\PYZus{}powerful}\PY{o}{=}\PY{n}{automobile}\PY{o}{.}\PY{n}{sort\PYZus{}values}\PY{p}{(}\PY{p}{[}\PY{l+s+s2}{\PYZdq{}}\PY{l+s+s2}{horsepower}\PY{l+s+s2}{\PYZdq{}}\PY{p}{]}\PY{p}{,} \PY{n}{ascending}\PY{o}{=}\PY{k+kc}{False}\PY{p}{)}\PY{o}{.}\PY{n}{head}\PY{p}{(}\PY{l+m+mi}{10}\PY{p}{)}\PY{p}{[}\PY{p}{[}\PY{l+s+s2}{\PYZdq{}}\PY{l+s+s2}{horsepower}\PY{l+s+s2}{\PYZdq{}}\PY{p}{]}\PY{p}{]}
        \PY{n}{auto\PYZus{}economic}\PY{o}{=}\PY{n}{automobile}\PY{o}{.}\PY{n}{sort\PYZus{}values}\PY{p}{(}\PY{p}{[}\PY{l+s+s2}{\PYZdq{}}\PY{l+s+s2}{city\PYZus{}mpg}\PY{l+s+s2}{\PYZdq{}}\PY{p}{,} \PY{l+s+s2}{\PYZdq{}}\PY{l+s+s2}{highway\PYZus{}mpg}\PY{l+s+s2}{\PYZdq{}}\PY{p}{]}\PY{p}{)}\PY{o}{.}\PY{n}{head}\PY{p}{(}\PY{l+m+mi}{10}\PY{p}{)}\PY{p}{[}\PY{p}{[}\PY{l+s+s2}{\PYZdq{}}\PY{l+s+s2}{city\PYZus{}mpg}\PY{l+s+s2}{\PYZdq{}}\PY{p}{,} \PY{l+s+s2}{\PYZdq{}}\PY{l+s+s2}{highway\PYZus{}mpg}\PY{l+s+s2}{\PYZdq{}}\PY{p}{]}\PY{p}{]}
        \PY{n+nb}{print}\PY{p}{(}\PY{l+s+s2}{\PYZdq{}}\PY{l+s+s2}{10 com maiores preços}\PY{l+s+s2}{\PYZdq{}}\PY{p}{)}
        \PY{n+nb}{print}\PY{p}{(}\PY{n}{auto\PYZus{}pricey}\PY{p}{)}
        \PY{n+nb}{print}\PY{p}{(}\PY{l+s+s2}{\PYZdq{}}\PY{l+s+se}{\PYZbs{}n}\PY{l+s+se}{\PYZbs{}n}\PY{l+s+s2}{10 com maiores dimensões}\PY{l+s+s2}{\PYZdq{}}\PY{p}{)}
        \PY{n+nb}{print}\PY{p}{(}\PY{n}{auto\PYZus{}biggest}\PY{p}{)}
        \PY{n+nb}{print}\PY{p}{(}\PY{l+s+s2}{\PYZdq{}}\PY{l+s+se}{\PYZbs{}n}\PY{l+s+se}{\PYZbs{}n}\PY{l+s+s2}{10 com maiores potências}\PY{l+s+s2}{\PYZdq{}}\PY{p}{)}
        \PY{n+nb}{print}\PY{p}{(}\PY{n}{auto\PYZus{}powerful}\PY{p}{)}
        \PY{n+nb}{print}\PY{p}{(}\PY{l+s+s2}{\PYZdq{}}\PY{l+s+se}{\PYZbs{}n}\PY{l+s+se}{\PYZbs{}n}\PY{l+s+s2}{10 mais econômicos}\PY{l+s+s2}{\PYZdq{}}\PY{p}{)}
        \PY{n+nb}{print}\PY{p}{(}\PY{n}{auto\PYZus{}economic}\PY{p}{)}
\end{Verbatim}


    \begin{Verbatim}[commandchars=\\\{\}]
10 com maiores preços
              make  price
71   mercedes-benz  45400
15             bmw  41315
70   mercedes-benz  40960
125        porsche  37028
16             bmw  36880
46          jaguar  36000
45          jaguar  35550
69   mercedes-benz  35056
68   mercedes-benz  34184
124        porsche  34028


10 com maiores dimensões
     length  width  height
70    208.1   71.7    56.7
68    202.6   71.7    56.5
67    202.6   71.7    56.3
44    199.6   69.6    52.8
45    199.6   69.6    52.8
71    199.2   72.0    55.4
106   198.9   68.4    58.7
107   198.9   68.4    58.7
111   198.9   68.4    58.7
110   198.9   68.4    56.7


10 com maiores potências
     horsepower
46          262
125         207
124         207
123         207
102         200
70          184
71          184
16          182
15          182
14          182


10 mais econômicos
     city\_mpg  highway\_mpg
66        0.0         25.0
181       0.0         34.0
46       13.0         17.0
70       14.0         16.0
71       14.0         16.0
44       15.0         19.0
45       15.0         19.0
16       15.0         20.0
68       16.0         18.0
69       16.0         18.0

    \end{Verbatim}

    \begin{Verbatim}[commandchars=\\\{\}]
{\color{incolor}In [{\color{incolor}5}]:} \PY{n}{auto\PYZus{}pricey\PYZus{}biggest} \PY{o}{=} \PY{n}{pd}\PY{o}{.}\PY{n}{merge}\PY{p}{(}\PY{n}{auto\PYZus{}pricey}\PY{p}{,} \PY{n}{auto\PYZus{}biggest}\PY{p}{,} \PY{n}{left\PYZus{}index}\PY{o}{=}\PY{k+kc}{True}\PY{p}{,} \PY{n}{right\PYZus{}index}\PY{o}{=}\PY{k+kc}{True}\PY{p}{)}
        \PY{n}{auto\PYZus{}pricey\PYZus{}powerful} \PY{o}{=} \PY{n}{pd}\PY{o}{.}\PY{n}{merge}\PY{p}{(}\PY{n}{auto\PYZus{}pricey}\PY{p}{,} \PY{n}{auto\PYZus{}powerful}\PY{p}{,} \PY{n}{left\PYZus{}index}\PY{o}{=}\PY{k+kc}{True}\PY{p}{,} \PY{n}{right\PYZus{}index}\PY{o}{=}\PY{k+kc}{True}\PY{p}{)}
        \PY{n}{auto\PYZus{}pricey\PYZus{}economic} \PY{o}{=} \PY{n}{pd}\PY{o}{.}\PY{n}{merge}\PY{p}{(}\PY{n}{auto\PYZus{}pricey}\PY{p}{,} \PY{n}{auto\PYZus{}economic}\PY{p}{,} \PY{n}{left\PYZus{}index}\PY{o}{=}\PY{k+kc}{True}\PY{p}{,} \PY{n}{right\PYZus{}index}\PY{o}{=}\PY{k+kc}{True}\PY{p}{)}
        \PY{n+nb}{print}\PY{p}{(}\PY{l+s+s2}{\PYZdq{}}\PY{l+s+s2}{Preço X Tamanho}\PY{l+s+s2}{\PYZdq{}}\PY{p}{)}
        \PY{n+nb}{print}\PY{p}{(}\PY{n}{auto\PYZus{}pricey\PYZus{}biggest}\PY{p}{)}
        \PY{n+nb}{print}\PY{p}{(}\PY{l+s+s2}{\PYZdq{}}\PY{l+s+s2}{\PYZdq{}}\PY{p}{)}
        \PY{n+nb}{print}\PY{p}{(}\PY{l+s+s2}{\PYZdq{}}\PY{l+s+s2}{Preço X Força}\PY{l+s+s2}{\PYZdq{}}\PY{p}{)}
        \PY{n+nb}{print}\PY{p}{(}\PY{n}{auto\PYZus{}pricey\PYZus{}powerful}\PY{p}{)}
        \PY{n+nb}{print}\PY{p}{(}\PY{l+s+s2}{\PYZdq{}}\PY{l+s+s2}{\PYZdq{}}\PY{p}{)}
        \PY{n+nb}{print}\PY{p}{(}\PY{l+s+s2}{\PYZdq{}}\PY{l+s+s2}{Preço X Consumo}\PY{l+s+s2}{\PYZdq{}}\PY{p}{)}
        \PY{n+nb}{print}\PY{p}{(}\PY{n}{auto\PYZus{}pricey\PYZus{}economic}\PY{p}{)}
\end{Verbatim}


    \begin{Verbatim}[commandchars=\\\{\}]
Preço X Tamanho
             make  price  length  width  height
71  mercedes-benz  45400   199.2   72.0    55.4
70  mercedes-benz  40960   208.1   71.7    56.7
45         jaguar  35550   199.6   69.6    52.8
68  mercedes-benz  34184   202.6   71.7    56.5

Preço X Força
              make  price  horsepower
71   mercedes-benz  45400         184
15             bmw  41315         182
70   mercedes-benz  40960         184
125        porsche  37028         207
16             bmw  36880         182
46          jaguar  36000         262
124        porsche  34028         207

Preço X Consumo
             make  price  city\_mpg  highway\_mpg
71  mercedes-benz  45400      14.0         16.0
70  mercedes-benz  40960      14.0         16.0
16            bmw  36880      15.0         20.0
46         jaguar  36000      13.0         17.0
45         jaguar  35550      15.0         19.0
69  mercedes-benz  35056      16.0         18.0
68  mercedes-benz  34184      16.0         18.0

    \end{Verbatim}

    \textbf{Pergunta:} Cite um exemplo de pelo menos os 3 cenários propostos
que corroboram a leitura dos dados apresentados, justique sua resposta.

    \textbf{Resposta:} No procedimento acima foram listadas as
classificações dos 10 carros com maior preço, maior tamanho, maior força
e menor consumo. A seguir a tabela de preços foi relacionada com as
outras três tabelas resultando nas tabelas ilustradas acima.

Baseado nesse cruzamento, nota-se que há quatro carros em comum entre os
10 maiores preços e os 10 maiores tamanhos. Entre maiores preços e
maiores forças há 7 carros em comum, bem como maiores preços e maior
consumo (menor eficiência *mpg).

Para corroborar a observação feita acima, nota-se, pela matriz de
correlação, que a relação entre preço e força (horsepower) é de 0.81,
indicando uma alta correlação positiva, ou seja, quanto maior a força
maior o preço. A correlação entre largura e comprimento e preço varia
entre 0.69 e 0.75, sendo razoavelmente alto o índice de correlação. Já a
correlação entre milhas por galão e preço varia entre -0.65 e -0.71,
indicando um índice de correlação negativo razoavelmente alto, ou seja,
quanto menor a eficiência do carro, maior o preço.

Também nota-se que os carros com índice 70 e 71 aparecem nos três
comparativos.

Portanto, pode-se afirmar que o preço dos automóveis é correlacionado
positivamente ao tamanho e força e negativamente em relação à
eficiência.

    \subsection{Preparação dos Dados}\label{preparauxe7uxe3o-dos-dados}

    \subsubsection{Identificação de Outliers
Visualmente}\label{identificauxe7uxe3o-de-outliers-visualmente}

    Utilize visualizações gráficas para encontrar outliers em todos os
atributos deste dataset. Ainda não vamos remover ou atualizar os
valores, por enquanto é uma análise exploratória.

Lembre-se que cada atributo possui um valor e dimensão diferente,
portanto comparações devem estar em uma mesma base, por exemplo,
\textbf{price} não pode ser comparado com \textbf{width} pois os eixos
\textbf{y} estarão sem proporção.

    \begin{Verbatim}[commandchars=\\\{\}]
{\color{incolor}In [{\color{incolor}6}]:} \PY{c+c1}{\PYZsh{}Converte a informação sobre número de portas para um valor numérico }
        \PY{n}{mapa\PYZus{}str\PYZus{}num} \PY{o}{=} \PY{p}{\PYZob{}}\PY{l+s+s2}{\PYZdq{}}\PY{l+s+s2}{one}\PY{l+s+s2}{\PYZdq{}}\PY{p}{:}\PY{n+nb}{int}\PY{p}{(}\PY{l+m+mi}{1}\PY{p}{)}\PY{p}{,}\PY{l+s+s2}{\PYZdq{}}\PY{l+s+s2}{two}\PY{l+s+s2}{\PYZdq{}}\PY{p}{:}\PY{n+nb}{int}\PY{p}{(}\PY{l+m+mi}{2}\PY{p}{)}\PY{p}{,}\PY{l+s+s2}{\PYZdq{}}\PY{l+s+s2}{three}\PY{l+s+s2}{\PYZdq{}}\PY{p}{:}\PY{n+nb}{int}\PY{p}{(}\PY{l+m+mi}{3}\PY{p}{)}\PY{p}{,}\PY{l+s+s2}{\PYZdq{}}\PY{l+s+s2}{four}\PY{l+s+s2}{\PYZdq{}}\PY{p}{:}\PY{n+nb}{int}\PY{p}{(}\PY{l+m+mi}{4}\PY{p}{)}\PY{p}{,}\PY{l+s+s2}{\PYZdq{}}\PY{l+s+s2}{five}\PY{l+s+s2}{\PYZdq{}}\PY{p}{:}\PY{n+nb}{int}\PY{p}{(}\PY{l+m+mi}{5}\PY{p}{)}\PY{p}{,}\PY{l+s+s2}{\PYZdq{}}\PY{l+s+s2}{six}\PY{l+s+s2}{\PYZdq{}}\PY{p}{:}\PY{n+nb}{int}\PY{p}{(}\PY{l+m+mi}{6}\PY{p}{)}\PY{p}{,}
                         \PY{l+s+s2}{\PYZdq{}}\PY{l+s+s2}{seven}\PY{l+s+s2}{\PYZdq{}}\PY{p}{:}\PY{n+nb}{int}\PY{p}{(}\PY{l+m+mi}{7}\PY{p}{)}\PY{p}{,}\PY{l+s+s2}{\PYZdq{}}\PY{l+s+s2}{eight}\PY{l+s+s2}{\PYZdq{}}\PY{p}{:}\PY{n+nb}{int}\PY{p}{(}\PY{l+m+mi}{8}\PY{p}{)}\PY{p}{,}\PY{l+s+s2}{\PYZdq{}}\PY{l+s+s2}{nine}\PY{l+s+s2}{\PYZdq{}}\PY{p}{:}\PY{n+nb}{int}\PY{p}{(}\PY{l+m+mi}{9}\PY{p}{)}\PY{p}{,}\PY{l+s+s2}{\PYZdq{}}\PY{l+s+s2}{ten}\PY{l+s+s2}{\PYZdq{}}\PY{p}{:}\PY{n+nb}{int}\PY{p}{(}\PY{l+m+mi}{10}\PY{p}{)}\PY{p}{,}\PY{l+s+s2}{\PYZdq{}}\PY{l+s+s2}{eleven}\PY{l+s+s2}{\PYZdq{}}\PY{p}{:}\PY{n+nb}{int}\PY{p}{(}\PY{l+m+mi}{11}\PY{p}{)}\PY{p}{,}\PY{l+s+s2}{\PYZdq{}}\PY{l+s+s2}{twelve}\PY{l+s+s2}{\PYZdq{}}\PY{p}{:}\PY{n+nb}{int}\PY{p}{(}\PY{l+m+mi}{12}\PY{p}{)}\PY{p}{\PYZcb{}}
        
        
        \PY{n}{automobile}\PY{p}{[}\PY{l+s+s2}{\PYZdq{}}\PY{l+s+s2}{number\PYZus{}of\PYZus{}doors}\PY{l+s+s2}{\PYZdq{}}\PY{p}{]} \PY{o}{=} \PY{n}{automobile}\PY{p}{[}\PY{l+s+s2}{\PYZdq{}}\PY{l+s+s2}{number\PYZus{}of\PYZus{}doors}\PY{l+s+s2}{\PYZdq{}}\PY{p}{]}\PY{o}{.}\PY{n}{map}\PY{p}{(}\PY{n}{mapa\PYZus{}str\PYZus{}num}\PY{p}{)} 
        \PY{n}{automobile}\PY{p}{[}\PY{l+s+s2}{\PYZdq{}}\PY{l+s+s2}{number\PYZus{}of\PYZus{}cylinders}\PY{l+s+s2}{\PYZdq{}}\PY{p}{]} \PY{o}{=} \PY{n}{automobile}\PY{p}{[}\PY{l+s+s2}{\PYZdq{}}\PY{l+s+s2}{number\PYZus{}of\PYZus{}cylinders}\PY{l+s+s2}{\PYZdq{}}\PY{p}{]}\PY{o}{.}\PY{n}{map}\PY{p}{(}\PY{n}{mapa\PYZus{}str\PYZus{}num}\PY{p}{)}
        
        \PY{n}{automobile}\PY{p}{[}\PY{p}{[}\PY{l+s+s2}{\PYZdq{}}\PY{l+s+s2}{number\PYZus{}of\PYZus{}cylinders}\PY{l+s+s2}{\PYZdq{}}\PY{p}{,}\PY{l+s+s2}{\PYZdq{}}\PY{l+s+s2}{number\PYZus{}of\PYZus{}doors}\PY{l+s+s2}{\PYZdq{}}\PY{p}{]}\PY{p}{]}\PY{o}{.}\PY{n}{head}\PY{p}{(}\PY{l+m+mi}{10}\PY{p}{)}
\end{Verbatim}


\begin{Verbatim}[commandchars=\\\{\}]
{\color{outcolor}Out[{\color{outcolor}6}]:}    number\_of\_cylinders  number\_of\_doors
        0                    4                2
        1                    4                2
        2                    6                2
        3                    4                4
        4                    5                4
        5                    5                2
        6                    5                4
        7                    5                4
        8                    5                4
        9                    4                2
\end{Verbatim}
            
    \begin{Verbatim}[commandchars=\\\{\}]
{\color{incolor}In [{\color{incolor}7}]:} \PY{k}{def} \PY{n+nf}{normalize}\PY{p}{(}\PY{n}{df}\PY{p}{,}\PY{n}{colunas}\PY{p}{)}\PY{p}{:}
            \PY{n}{df\PYZus{}aux} \PY{o}{=} \PY{n}{df}\PY{o}{.}\PY{n}{copy}\PY{p}{(}\PY{p}{)}
            \PY{k}{for} \PY{n}{col} \PY{o+ow}{in} \PY{n}{colunas}\PY{p}{:}
                \PY{n}{df\PYZus{}aux}\PY{p}{[}\PY{n}{col}\PY{p}{]} \PY{o}{=} \PY{p}{(}\PY{n}{df\PYZus{}aux}\PY{p}{[}\PY{n}{col}\PY{p}{]} \PY{o}{\PYZhy{}} \PY{n}{df\PYZus{}aux}\PY{p}{[}\PY{n}{col}\PY{p}{]}\PY{o}{.}\PY{n}{mean}\PY{p}{(}\PY{p}{)}\PY{p}{)} \PY{o}{/} \PY{p}{(}\PY{n}{df\PYZus{}aux}\PY{p}{[}\PY{n}{col}\PY{p}{]}\PY{o}{.}\PY{n}{max}\PY{p}{(}\PY{p}{)} \PY{o}{\PYZhy{}} \PY{n}{df\PYZus{}aux}\PY{p}{[}\PY{n}{col}\PY{p}{]}\PY{o}{.}\PY{n}{min}\PY{p}{(}\PY{p}{)}\PY{p}{)}
            \PY{k}{return} \PY{p}{(}\PY{n}{df\PYZus{}aux}\PY{p}{)} 
        
        \PY{n}{cols\PYZus{}numericas} \PY{o}{=} \PY{p}{[}\PY{l+s+s2}{\PYZdq{}}\PY{l+s+s2}{number\PYZus{}of\PYZus{}doors}\PY{l+s+s2}{\PYZdq{}}\PY{p}{,}\PY{l+s+s2}{\PYZdq{}}\PY{l+s+s2}{wheel\PYZus{}base}\PY{l+s+s2}{\PYZdq{}}\PY{p}{,}\PY{l+s+s2}{\PYZdq{}}\PY{l+s+s2}{length}\PY{l+s+s2}{\PYZdq{}}\PY{p}{,}\PY{l+s+s2}{\PYZdq{}}\PY{l+s+s2}{width}\PY{l+s+s2}{\PYZdq{}}\PY{p}{,}\PY{l+s+s2}{\PYZdq{}}\PY{l+s+s2}{height}\PY{l+s+s2}{\PYZdq{}}\PY{p}{,}
                              \PY{l+s+s2}{\PYZdq{}}\PY{l+s+s2}{curb\PYZus{}weight}\PY{l+s+s2}{\PYZdq{}}\PY{p}{,}\PY{l+s+s2}{\PYZdq{}}\PY{l+s+s2}{number\PYZus{}of\PYZus{}cylinders}\PY{l+s+s2}{\PYZdq{}}\PY{p}{,}\PY{l+s+s2}{\PYZdq{}}\PY{l+s+s2}{engine\PYZus{}size}\PY{l+s+s2}{\PYZdq{}}\PY{p}{,}
                              \PY{l+s+s2}{\PYZdq{}}\PY{l+s+s2}{bore}\PY{l+s+s2}{\PYZdq{}}\PY{p}{,}\PY{l+s+s2}{\PYZdq{}}\PY{l+s+s2}{stroke}\PY{l+s+s2}{\PYZdq{}}\PY{p}{,}\PY{l+s+s2}{\PYZdq{}}\PY{l+s+s2}{compression\PYZus{}ratio}\PY{l+s+s2}{\PYZdq{}}\PY{p}{,}\PY{l+s+s2}{\PYZdq{}}\PY{l+s+s2}{horsepower}\PY{l+s+s2}{\PYZdq{}}\PY{p}{,}\PY{l+s+s2}{\PYZdq{}}\PY{l+s+s2}{peak\PYZus{}rpm}\PY{l+s+s2}{\PYZdq{}}\PY{p}{,}
                              \PY{l+s+s2}{\PYZdq{}}\PY{l+s+s2}{city\PYZus{}mpg}\PY{l+s+s2}{\PYZdq{}}\PY{p}{,}\PY{l+s+s2}{\PYZdq{}}\PY{l+s+s2}{highway\PYZus{}mpg}\PY{l+s+s2}{\PYZdq{}}\PY{p}{,}\PY{l+s+s2}{\PYZdq{}}\PY{l+s+s2}{price}\PY{l+s+s2}{\PYZdq{}}\PY{p}{]}
        
        \PY{c+c1}{\PYZsh{}Normaliza as informações númericas do dataframe de automóveis }
        \PY{n}{automobile\PYZus{}norm} \PY{o}{=} \PY{n}{normalize}\PY{p}{(}\PY{n}{automobile}\PY{p}{,}\PY{n}{cols\PYZus{}numericas}\PY{p}{)}
        
        \PY{n}{automobile\PYZus{}norm}\PY{p}{[}\PY{n}{cols\PYZus{}numericas}\PY{p}{]}\PY{o}{.}\PY{n}{head}\PY{p}{(}\PY{l+m+mi}{10}\PY{p}{)}
\end{Verbatim}


\begin{Verbatim}[commandchars=\\\{\}]
{\color{outcolor}Out[{\color{outcolor}7}]:}    number\_of\_doors  wheel\_base    length     width    height  curb\_weight  \textbackslash{}
        0        -0.567164   -0.297289 -0.080612 -0.152911 -0.413889    -0.002974   
        1        -0.567164   -0.297289 -0.080612 -0.152911 -0.413889    -0.002974   
        2        -0.567164   -0.125277 -0.044791 -0.033253 -0.113889     0.103698   
        3         0.432836    0.029242  0.035806  0.026577  0.044444    -0.084820   
        4         0.432836    0.017580  0.035806  0.043671  0.044444     0.104086   
        5        -0.567164    0.029242  0.046254  0.035124 -0.055556    -0.018878   
        6         0.432836    0.204169  0.276105  0.471021  0.161111     0.111844   
        7         0.432836    0.204169  0.276105  0.471021  0.161111     0.154513   
        8         0.432836    0.204169  0.276105  0.471021  0.177778     0.205715   
        9        -0.567164    0.070058  0.038791 -0.093082  0.044444    -0.062322   
        
           number\_of\_cylinders  engine\_size      bore    stroke  compression\_ratio  \textbackslash{}
        0            -0.036318     0.011790  0.100213 -0.277020          -0.072767   
        1            -0.036318     0.011790  0.100213 -0.277020          -0.072767   
        2             0.163682     0.094809 -0.464072  0.099171          -0.072767   
        3            -0.036318    -0.067455 -0.099787  0.065837          -0.010267   
        4             0.063682     0.034432 -0.099787  0.065837          -0.135267   
        5             0.063682     0.034432 -0.099787  0.065837          -0.104017   
        6             0.063682     0.034432 -0.099787  0.065837          -0.104017   
        7             0.063682     0.034432 -0.099787  0.065837          -0.104017   
        8             0.063682     0.015564 -0.142644  0.065837          -0.116517   
        9            -0.036318    -0.071229  0.121642 -0.219877          -0.085267   
        
           horsepower  peak\_rpm  city\_mpg  highway\_mpg     price  
        0    0.036151 -0.009517 -0.080102    -0.065648  0.007146  
        1    0.036151 -0.009517 -0.080102    -0.065648  0.081745  
        2    0.237086 -0.009517 -0.120918    -0.084167  0.081745  
        3   -0.005905  0.066240 -0.018878    -0.010093  0.018442  
        4    0.054843  0.066240 -0.141327    -0.158241  0.105329  
        5    0.031478  0.066240 -0.120918    -0.102685  0.050714  
        6    0.031478  0.066240 -0.120918    -0.102685  0.111784  
        7    0.031478  0.066240 -0.120918    -0.102685  0.141822  
        8    0.171665  0.066240 -0.161735    -0.195278  0.264830  
        9   -0.010578  0.111695 -0.039286    -0.028611  0.080008  
\end{Verbatim}
            
    \begin{Verbatim}[commandchars=\\\{\}]
{\color{incolor}In [{\color{incolor}8}]:} \PY{k+kn}{import} \PY{n+nn}{math} 
        
        \PY{n}{n\PYZus{}linhas} \PY{o}{=} \PY{n+nb}{len}\PY{p}{(}\PY{n}{cols\PYZus{}numericas}\PY{p}{)}
        \PY{n}{fig}\PY{p}{,}\PY{n}{ax} \PY{o}{=} \PY{n}{plt}\PY{o}{.}\PY{n}{subplots}\PY{p}{(}\PY{n}{nrows}\PY{o}{=}\PY{n}{n\PYZus{}linhas}\PY{p}{,}\PY{n}{ncols}\PY{o}{=}\PY{l+m+mi}{2}\PY{p}{,} \PY{n}{sharey}\PY{o}{=}\PY{k+kc}{True}\PY{p}{)}
        
        
        \PY{n}{fig}\PY{o}{.}\PY{n}{set\PYZus{}size\PYZus{}inches}\PY{p}{(}\PY{l+m+mi}{18}\PY{p}{,} \PY{l+m+mi}{88}\PY{p}{)}
        \PY{k}{for} \PY{n}{i} \PY{o+ow}{in} \PY{n+nb}{range}\PY{p}{(}\PY{n}{n\PYZus{}linhas}\PY{p}{)}\PY{p}{:}    
            \PY{n}{sns}\PY{o}{.}\PY{n}{boxplot}\PY{p}{(}\PY{n}{ax}\PY{o}{=}\PY{n}{ax}\PY{p}{[}\PY{n}{i}\PY{p}{,}\PY{l+m+mi}{0}\PY{p}{]}\PY{p}{,} \PY{n}{data}\PY{o}{=}\PY{p}{[}\PY{n}{automobile\PYZus{}norm}\PY{p}{[}\PY{n}{cols\PYZus{}numericas}\PY{p}{[}\PY{n}{i}\PY{p}{]}\PY{p}{]}\PY{p}{]}\PY{p}{)}\PY{o}{.}\PY{n}{set\PYZus{}title}\PY{p}{(}\PY{n}{cols\PYZus{}numericas}\PY{p}{[}\PY{n}{i}\PY{p}{]}\PY{p}{)}
            \PY{n}{sns}\PY{o}{.}\PY{n}{violinplot}\PY{p}{(}\PY{n}{ax}\PY{o}{=}\PY{n}{ax}\PY{p}{[}\PY{n}{i}\PY{p}{,}\PY{l+m+mi}{1}\PY{p}{]}\PY{p}{,} \PY{n}{data}\PY{o}{=}\PY{p}{[}\PY{n}{automobile\PYZus{}norm}\PY{p}{[}\PY{n}{cols\PYZus{}numericas}\PY{p}{[}\PY{n}{i}\PY{p}{]}\PY{p}{]}\PY{p}{]}\PY{p}{)}\PY{o}{.}\PY{n}{set\PYZus{}title}\PY{p}{(}\PY{n}{cols\PYZus{}numericas}\PY{p}{[}\PY{n}{i}\PY{p}{]}\PY{p}{)}    
\end{Verbatim}


    \begin{center}
    \adjustimage{max size={0.9\linewidth}{0.9\paperheight}}{output_18_0.png}
    \end{center}
    { \hspace*{\fill} \\}
    
    Podemos verificar visualmente que há candidatos a serem outliers para as
seguinte características dos veículos: wheel\_base, length, width,
curb\_weight, number\_of\_cylinders, engine\_size, bore, stroke,
compression\_ratio, horsepower, peak\_rpm, city\_mpg, highway\_mpg e
price.

    \begin{Verbatim}[commandchars=\\\{\}]
{\color{incolor}In [{\color{incolor}9}]:} \PY{n}{estat} \PY{o}{=} \PY{n}{automobile\PYZus{}norm}\PY{o}{.}\PY{n}{describe}\PY{p}{(}\PY{p}{)}
        \PY{k}{for} \PY{n}{col} \PY{o+ow}{in} \PY{n}{cols\PYZus{}numericas}\PY{p}{:}
            \PY{c+c1}{\PYZsh{}Intervalo inter\PYZhy{}quartil}
            \PY{n}{iq} \PY{o}{=} \PY{p}{(}\PY{n}{estat}\PY{p}{[}\PY{n}{col}\PY{p}{]}\PY{p}{[}\PY{l+s+s2}{\PYZdq{}}\PY{l+s+s2}{75}\PY{l+s+s2}{\PYZpc{}}\PY{l+s+s2}{\PYZdq{}}\PY{p}{]} \PY{o}{\PYZhy{}} \PY{n}{estat}\PY{p}{[}\PY{n}{col}\PY{p}{]}\PY{p}{[}\PY{l+s+s2}{\PYZdq{}}\PY{l+s+s2}{25}\PY{l+s+s2}{\PYZpc{}}\PY{l+s+s2}{\PYZdq{}}\PY{p}{]}\PY{p}{)}
            \PY{c+c1}{\PYZsh{}Limite superior para outlier é 1,5 vezes o intervalo interquartílico acima do 75\PYZpc{} percentil}
            \PY{n}{lim\PYZus{}sup} \PY{o}{=} \PY{n}{estat}\PY{p}{[}\PY{n}{col}\PY{p}{]}\PY{p}{[}\PY{l+s+s2}{\PYZdq{}}\PY{l+s+s2}{75}\PY{l+s+s2}{\PYZpc{}}\PY{l+s+s2}{\PYZdq{}}\PY{p}{]} \PY{o}{+} \PY{l+m+mf}{1.5} \PY{o}{*} \PY{n}{iq} 
            \PY{c+c1}{\PYZsh{}Limite inferior para outlier é 1,5 vezes o intervalo interquartílico abaixo do 25\PYZpc{} percentil}
            \PY{n}{lim\PYZus{}inf} \PY{o}{=} \PY{n}{estat}\PY{p}{[}\PY{n}{col}\PY{p}{]}\PY{p}{[}\PY{l+s+s2}{\PYZdq{}}\PY{l+s+s2}{25}\PY{l+s+s2}{\PYZpc{}}\PY{l+s+s2}{\PYZdq{}}\PY{p}{]} \PY{o}{\PYZhy{}} \PY{l+m+mf}{1.5} \PY{o}{*} \PY{n}{iq} 
            \PY{c+c1}{\PYZsh{}Seleciona os carros que seguem a regra calculada acima}
            \PY{n}{vet\PYZus{}out} \PY{o}{=} \PY{n}{automobile\PYZus{}norm}\PY{p}{[}\PY{p}{(}\PY{n}{automobile\PYZus{}norm}\PY{p}{[}\PY{n}{col}\PY{p}{]} \PY{o}{\PYZgt{}} \PY{n}{lim\PYZus{}sup}\PY{p}{)} \PY{o}{|} \PY{p}{(}\PY{n}{automobile\PYZus{}norm}\PY{p}{[}\PY{n}{col}\PY{p}{]} \PY{o}{\PYZlt{}} \PY{n}{lim\PYZus{}inf}\PY{p}{)}\PY{p}{]}\PY{p}{[}\PY{p}{[}\PY{n}{col}\PY{p}{]}\PY{p}{]}
            \PY{c+c1}{\PYZsh{}Se algum elemento foi selecionado, imprima}
            \PY{k}{if} \PY{p}{(}\PY{n+nb}{len}\PY{p}{(}\PY{n}{vet\PYZus{}out}\PY{p}{)} \PY{o}{\PYZgt{}} \PY{l+m+mi}{0} \PY{p}{)}\PY{p}{:}
               \PY{n+nb}{print}\PY{p}{(}\PY{l+s+s2}{\PYZdq{}}\PY{l+s+s2}{Candidatos a outliers para }\PY{l+s+s2}{\PYZdq{}} \PY{o}{+} \PY{n}{col}\PY{p}{)} 
               \PY{n+nb}{print}\PY{p}{(}\PY{n}{vet\PYZus{}out}\PY{p}{)}
               \PY{n+nb}{print}\PY{p}{(}\PY{l+s+s2}{\PYZdq{}}\PY{l+s+se}{\PYZbs{}n}\PY{l+s+se}{\PYZbs{}n}\PY{l+s+s2}{\PYZdq{}}\PY{p}{)}
\end{Verbatim}


    \begin{Verbatim}[commandchars=\\\{\}]
Candidatos a outliers para wheel\_base
    wheel\_base
67    0.489883
68    0.489883
70    0.644402



Candidatos a outliers para length
      length
17 -0.494045



Candidatos a outliers para width
       width
6   0.471021
7   0.471021
8   0.471021
16  0.428286
17 -0.477697
46  0.402645
67  0.496662
68  0.496662
69  0.394098
70  0.496662
71  0.522303



Candidatos a outliers para curb\_weight
    curb\_weight
44     0.585855
45     0.585855



Candidatos a outliers para number\_of\_cylinders
     number\_of\_cylinders
2               0.163682
4               0.063682
5               0.063682
6               0.063682
7               0.063682
8               0.063682
11              0.163682
12              0.163682
13              0.163682
14              0.163682
15              0.163682
16              0.163682
17             -0.136318
44              0.163682
45              0.163682
46              0.763682
52             -0.236318
53             -0.236318
54             -0.236318
55             -0.236318
64              0.063682
65              0.063682
66              0.063682
67              0.063682
68              0.363682
69              0.363682
70              0.363682
71              0.363682
98              0.163682
99              0.163682
100             0.163682
101             0.163682
102             0.163682
103             0.163682
123             0.163682
124             0.163682
125             0.163682
174             0.163682
175             0.163682
176             0.163682
177             0.163682
187             0.063682
198             0.163682
199             0.163682



Candidatos a outliers para engine\_size
    engine\_size
14     0.309903
15     0.309903
16     0.309903
44     0.494809
45     0.494809
46     0.751413
68     0.404243
69     0.404243
70     0.683488
71     0.668394



Candidatos a outliers para stroke
       stroke
44   0.432504
45   0.432504
108 -0.510353
110 -0.510353
130 -0.567496
134 -0.429401



Candidatos a outliers para compression\_ratio
     compression\_ratio
28           -0.197767
46            0.083483
60            0.783483
63            0.739733
64            0.708483
65            0.708483
66            0.708483
67            0.708483
79           -0.197767
80           -0.197767
81           -0.197767
87            0.733483
105           0.677233
107           0.677233
109           0.677233
111           0.677233
113           0.677233
114          -0.197767
121          -0.197767
154           0.770983
155           0.770983
170           0.770983
178           0.802233
180           0.802233
183           0.802233
188           0.802233
199           0.802233



Candidatos a outliers para horsepower
     horsepower
46     0.741758
102    0.452039
123    0.484749
124    0.484749
125    0.484749



Candidatos a outliers para peak\_rpm
     peak\_rpm
32  -0.767093
35  -0.767093
161  0.232907
162  0.232907



Candidatos a outliers para city\_mpg
     city\_mpg
17   0.450510
29   0.491327
66  -0.508673
181 -0.508673



Candidatos a outliers para highway\_mpg
    highway\_mpg
17     0.415833
29     0.434352
67    -0.565648
87     0.360278



Candidatos a outliers para price
        price
14   0.435750
15   0.697777
16   0.587679
44   0.472739
45   0.554661
46   0.565833
67   0.456603
68   0.520750
69   0.542398
70   0.688965
71   0.799187
123  0.479640
124  0.516878
125  0.591353




    \end{Verbatim}

    \begin{Verbatim}[commandchars=\\\{\}]
{\color{incolor}In [{\color{incolor}10}]:} \PY{c+c1}{\PYZsh{}Análise dos atributos categóricos nominais}
         \PY{n}{cols\PYZus{}alfa} \PY{o}{=} \PY{p}{[}\PY{l+s+s2}{\PYZdq{}}\PY{l+s+s2}{make}\PY{l+s+s2}{\PYZdq{}}\PY{p}{,}\PY{l+s+s2}{\PYZdq{}}\PY{l+s+s2}{fuel\PYZus{}type}\PY{l+s+s2}{\PYZdq{}}\PY{p}{,}\PY{l+s+s2}{\PYZdq{}}\PY{l+s+s2}{aspiration}\PY{l+s+s2}{\PYZdq{}}\PY{p}{,}\PY{l+s+s2}{\PYZdq{}}\PY{l+s+s2}{body\PYZus{}style}\PY{l+s+s2}{\PYZdq{}}\PY{p}{,}\PY{l+s+s2}{\PYZdq{}}\PY{l+s+s2}{drive\PYZus{}wheels}\PY{l+s+s2}{\PYZdq{}}\PY{p}{,}\PY{l+s+s2}{\PYZdq{}}\PY{l+s+s2}{engine\PYZus{}type}\PY{l+s+s2}{\PYZdq{}}\PY{p}{,}\PY{l+s+s2}{\PYZdq{}}\PY{l+s+s2}{engine\PYZus{}location}\PY{l+s+s2}{\PYZdq{}}\PY{p}{,}\PY{l+s+s2}{\PYZdq{}}\PY{l+s+s2}{fuel\PYZus{}system}\PY{l+s+s2}{\PYZdq{}}\PY{p}{]}
         \PY{n}{n\PYZus{}linhas} \PY{o}{=} \PY{n+nb}{len}\PY{p}{(}\PY{n}{cols\PYZus{}alfa}\PY{p}{)}
         \PY{n}{fig}\PY{p}{,}\PY{n}{ax} \PY{o}{=} \PY{n}{plt}\PY{o}{.}\PY{n}{subplots}\PY{p}{(}\PY{n}{nrows}\PY{o}{=}\PY{n}{n\PYZus{}linhas}\PY{p}{,} \PY{n}{sharey}\PY{o}{=}\PY{k+kc}{True}\PY{p}{)}
         \PY{n}{fig}\PY{o}{.}\PY{n}{set\PYZus{}size\PYZus{}inches}\PY{p}{(}\PY{l+m+mi}{18}\PY{p}{,} \PY{l+m+mi}{38}\PY{p}{)}
         \PY{c+c1}{\PYZsh{}Exibe o histograma de quantidades de elementos por categoria para verificar se há alguma informação discrepante}
         \PY{k}{for} \PY{n}{i} \PY{o+ow}{in} \PY{n+nb}{range}\PY{p}{(}\PY{n}{n\PYZus{}linhas}\PY{p}{)}\PY{p}{:}
             \PY{n}{sns}\PY{o}{.}\PY{n}{countplot}\PY{p}{(}\PY{n}{ax}\PY{o}{=}\PY{n}{ax}\PY{p}{[}\PY{n}{i}\PY{p}{]}\PY{p}{,}\PY{n}{x}\PY{o}{=}\PY{n}{cols\PYZus{}alfa}\PY{p}{[}\PY{n}{i}\PY{p}{]}\PY{p}{,}\PY{n}{data}\PY{o}{=}\PY{n}{automobile\PYZus{}norm}\PY{p}{)}
             
\end{Verbatim}


    \begin{center}
    \adjustimage{max size={0.9\linewidth}{0.9\paperheight}}{output_21_0.png}
    \end{center}
    { \hspace*{\fill} \\}
    
    Observando os atributos categóricos sobre veículos, não há valor
discrepante que não represente a informação correspondente. Deste modo,
os fabricantes de veículos são realmente nomes de fabricantes, o tipo de
combustível é coerente com o esperado, e assim sucessivamente.

    Antes de explorar melhor as correlações, vamos analisar se há dados
discrepante no conjunto de dados. Analisar os dados sem esta análise
preliminar pode nos levar a algum viés indesejado.

    \subsubsection{Análise de valores
inválidos}\label{anuxe1lise-de-valores-invuxe1lidos}

    Analise e busque por valores considerados nulos.

    \begin{Verbatim}[commandchars=\\\{\}]
{\color{incolor}In [{\color{incolor}11}]:} \PY{c+c1}{\PYZsh{}Verifica se há valores nulos}
         \PY{n}{automobile}\PY{o}{.}\PY{n}{isnull}\PY{p}{(}\PY{p}{)}\PY{o}{.}\PY{n}{any}\PY{p}{(}\PY{p}{)}
\end{Verbatim}


\begin{Verbatim}[commandchars=\\\{\}]
{\color{outcolor}Out[{\color{outcolor}11}]:} make                   False
         fuel\_type              False
         aspiration             False
         number\_of\_doors        False
         body\_style             False
         drive\_wheels           False
         engine\_location        False
         wheel\_base             False
         length                 False
         width                  False
         height                 False
         curb\_weight            False
         engine\_type            False
         number\_of\_cylinders    False
         engine\_size            False
         fuel\_system            False
         bore                   False
         stroke                 False
         compression\_ratio      False
         horsepower             False
         peak\_rpm                True
         city\_mpg                True
         highway\_mpg             True
         price                  False
         dtype: bool
\end{Verbatim}
            
    \begin{Verbatim}[commandchars=\\\{\}]
{\color{incolor}In [{\color{incolor}12}]:} \PY{c+c1}{\PYZsh{}Automóveis com peak\PYZus{}rpm nulo}
         \PY{n}{automobile}\PY{p}{[}\PY{n}{automobile}\PY{p}{[}\PY{l+s+s2}{\PYZdq{}}\PY{l+s+s2}{peak\PYZus{}rpm}\PY{l+s+s2}{\PYZdq{}}\PY{p}{]}\PY{o}{.}\PY{n}{isnull}\PY{p}{(}\PY{p}{)}\PY{p}{]}
\end{Verbatim}


\begin{Verbatim}[commandchars=\\\{\}]
{\color{outcolor}Out[{\color{outcolor}12}]:}            make fuel\_type aspiration  number\_of\_doors body\_style drive\_wheels  \textbackslash{}
         136      subaru       gas        std                2  hatchback          4wd   
         184  volkswagen       gas        std                4      sedan          fwd   
         
             engine\_location  wheel\_base  length  width  {\ldots}    engine\_size  \textbackslash{}
         136           front        93.3   157.3   63.8  {\ldots}            108   
         184           front        97.3   171.7   65.5  {\ldots}            109   
         
              fuel\_system  bore  stroke  compression\_ratio horsepower  peak\_rpm  \textbackslash{}
         136         2bbl  3.62    2.64                8.7         73       NaN   
         184         mpfi  3.19    3.40               10.0        100       NaN   
         
              city\_mpg  highway\_mpg  price  
         136      26.0         31.0   7603  
         184      26.0         32.0   9995  
         
         [2 rows x 24 columns]
\end{Verbatim}
            
    \begin{Verbatim}[commandchars=\\\{\}]
{\color{incolor}In [{\color{incolor}13}]:} \PY{c+c1}{\PYZsh{}Automóveis com city\PYZus{}mpg nulo}
         \PY{n}{automobile}\PY{p}{[}\PY{n}{automobile}\PY{p}{[}\PY{l+s+s2}{\PYZdq{}}\PY{l+s+s2}{city\PYZus{}mpg}\PY{l+s+s2}{\PYZdq{}}\PY{p}{]}\PY{o}{.}\PY{n}{isnull}\PY{p}{(}\PY{p}{)}\PY{p}{]}
\end{Verbatim}


\begin{Verbatim}[commandchars=\\\{\}]
{\color{outcolor}Out[{\color{outcolor}13}]:}      make fuel\_type aspiration  number\_of\_doors body\_style drive\_wheels  \textbackslash{}
         36  honda       gas        std                2  hatchback          fwd   
         
            engine\_location  wheel\_base  length  width  {\ldots}    engine\_size  \textbackslash{}
         36           front        96.5   167.5   65.2  {\ldots}            110   
         
             fuel\_system  bore  stroke  compression\_ratio horsepower  peak\_rpm  \textbackslash{}
         36         1bbl  3.15    3.58                9.0         86    5800.0   
         
             city\_mpg  highway\_mpg  price  
         36       NaN         33.0   7895  
         
         [1 rows x 24 columns]
\end{Verbatim}
            
    \begin{Verbatim}[commandchars=\\\{\}]
{\color{incolor}In [{\color{incolor}14}]:} \PY{c+c1}{\PYZsh{}Automóveis com highway\PYZus{}mpg nulo}
         \PY{n}{automobile}\PY{p}{[}\PY{n}{automobile}\PY{p}{[}\PY{l+s+s2}{\PYZdq{}}\PY{l+s+s2}{highway\PYZus{}mpg}\PY{l+s+s2}{\PYZdq{}}\PY{p}{]}\PY{o}{.}\PY{n}{isnull}\PY{p}{(}\PY{p}{)}\PY{p}{]}
\end{Verbatim}


\begin{Verbatim}[commandchars=\\\{\}]
{\color{outcolor}Out[{\color{outcolor}14}]:}      make fuel\_type aspiration  number\_of\_doors body\_style drive\_wheels  \textbackslash{}
         32  honda       gas        std                2  hatchback          fwd   
         
            engine\_location  wheel\_base  length  width  {\ldots}    engine\_size  \textbackslash{}
         32           front        93.7   150.0   64.0  {\ldots}             92   
         
             fuel\_system  bore  stroke  compression\_ratio horsepower  peak\_rpm  \textbackslash{}
         32         1bbl  2.91    3.41                9.2         76       0.0   
         
             city\_mpg  highway\_mpg  price  
         32      30.0          NaN   6529  
         
         [1 rows x 24 columns]
\end{Verbatim}
            
    \begin{Verbatim}[commandchars=\\\{\}]
{\color{incolor}In [{\color{incolor}15}]:} \PY{c+c1}{\PYZsh{}Verifica\PYZhy{}se a correlação entre o consumo na estrada e o consumo na cidade}
         \PY{c+c1}{\PYZsh{}e percebe\PYZhy{}se que há uma alta correlação entre os atributos }
         \PY{n+nb}{print}\PY{p}{(}\PY{n}{automobile}\PY{p}{[}\PY{p}{[}\PY{l+s+s2}{\PYZdq{}}\PY{l+s+s2}{highway\PYZus{}mpg}\PY{l+s+s2}{\PYZdq{}}\PY{p}{,} \PY{l+s+s2}{\PYZdq{}}\PY{l+s+s2}{city\PYZus{}mpg}\PY{l+s+s2}{\PYZdq{}}\PY{p}{]}\PY{p}{]}\PY{o}{.}\PY{n}{corr}\PY{p}{(}\PY{p}{)}\PY{p}{)}
         
         
         \PY{c+c1}{\PYZsh{}Calcula a relação entre o consumo da estrada e o consumo da cidade para quando há os valores ou quando }
         \PY{c+c1}{\PYZsh{}os valores não são zerados }
         \PY{n}{automobile}\PY{p}{[}\PY{l+s+s2}{\PYZdq{}}\PY{l+s+s2}{relacao\PYZus{}estrada\PYZus{}cidade}\PY{l+s+s2}{\PYZdq{}}\PY{p}{]} \PY{o}{=} \PY{n}{np}\PY{o}{.}\PY{n}{where}\PY{p}{(}\PY{p}{(}\PY{n}{automobile}\PY{p}{[}\PY{l+s+s2}{\PYZdq{}}\PY{l+s+s2}{city\PYZus{}mpg}\PY{l+s+s2}{\PYZdq{}}\PY{p}{]}\PY{o}{.}\PY{n}{isnull}\PY{p}{(}\PY{p}{)}\PY{p}{)}  \PY{o}{|} \PY{p}{(}\PY{n}{automobile}\PY{p}{[}\PY{l+s+s2}{\PYZdq{}}\PY{l+s+s2}{city\PYZus{}mpg}\PY{l+s+s2}{\PYZdq{}}\PY{p}{]}\PY{o}{==}\PY{l+m+mi}{0}\PY{p}{)} \PY{o}{|}
                                                         \PY{p}{(}\PY{n}{automobile}\PY{p}{[}\PY{l+s+s2}{\PYZdq{}}\PY{l+s+s2}{highway\PYZus{}mpg}\PY{l+s+s2}{\PYZdq{}}\PY{p}{]}\PY{o}{.}\PY{n}{isnull}\PY{p}{(}\PY{p}{)}\PY{p}{)} \PY{o}{|} \PY{p}{(}\PY{n}{automobile}\PY{p}{[}\PY{l+s+s2}{\PYZdq{}}\PY{l+s+s2}{highway\PYZus{}mpg}\PY{l+s+s2}{\PYZdq{}}\PY{p}{]}\PY{o}{==}\PY{l+m+mi}{0}\PY{p}{)}\PY{p}{,}
                                                          \PY{n}{np}\PY{o}{.}\PY{n}{nan}\PY{p}{,}
                                                         \PY{n}{automobile}\PY{p}{[}\PY{l+s+s2}{\PYZdq{}}\PY{l+s+s2}{highway\PYZus{}mpg}\PY{l+s+s2}{\PYZdq{}}\PY{p}{]} \PY{o}{/} \PY{n}{automobile}\PY{p}{[}\PY{l+s+s2}{\PYZdq{}}\PY{l+s+s2}{city\PYZus{}mpg}\PY{l+s+s2}{\PYZdq{}}\PY{p}{]}\PY{p}{)}
         
         
         \PY{c+c1}{\PYZsh{}Percebe\PYZhy{}se pelo gráfico que a distribuição entre a relacao\PYZus{}estrada\PYZus{}cidade é muito concentrada ao redor da média,}
         \PY{c+c1}{\PYZsh{}semelhante a uma distribuição normal }
         \PY{n}{ax} \PY{o}{=} \PY{n}{plt}\PY{o}{.}\PY{n}{hist}\PY{p}{(}\PY{n}{automobile}\PY{p}{[}\PY{n}{automobile}\PY{p}{[}\PY{l+s+s2}{\PYZdq{}}\PY{l+s+s2}{relacao\PYZus{}estrada\PYZus{}cidade}\PY{l+s+s2}{\PYZdq{}}\PY{p}{]}\PY{o}{.}\PY{n}{notnull}\PY{p}{(}\PY{p}{)}\PY{p}{]}\PY{p}{[}\PY{l+s+s2}{\PYZdq{}}\PY{l+s+s2}{relacao\PYZus{}estrada\PYZus{}cidade}\PY{l+s+s2}{\PYZdq{}}\PY{p}{]}\PY{p}{)}
\end{Verbatim}


    \begin{Verbatim}[commandchars=\\\{\}]
             highway\_mpg  city\_mpg
highway\_mpg     1.000000  0.875933
city\_mpg        0.875933  1.000000

    \end{Verbatim}

    \begin{center}
    \adjustimage{max size={0.9\linewidth}{0.9\paperheight}}{output_30_1.png}
    \end{center}
    { \hspace*{\fill} \\}
    
    \begin{Verbatim}[commandchars=\\\{\}]
{\color{incolor}In [{\color{incolor}16}]:} \PY{c+c1}{\PYZsh{}Verifica\PYZhy{}se agora que o peak\PYZus{}rpm não possui uma alta correlação com nenhum outro atributo sobre automóveis}
         \PY{c+c1}{\PYZsh{}conforme percebe\PYZhy{}se abaixo }
         \PY{n}{automobile}\PY{o}{.}\PY{n}{corr}\PY{p}{(}\PY{p}{)}\PY{p}{[}\PY{l+s+s2}{\PYZdq{}}\PY{l+s+s2}{peak\PYZus{}rpm}\PY{l+s+s2}{\PYZdq{}}\PY{p}{]}
\end{Verbatim}


\begin{Verbatim}[commandchars=\\\{\}]
{\color{outcolor}Out[{\color{outcolor}16}]:} number\_of\_doors          -0.163910
         wheel\_base               -0.202727
         length                   -0.056354
         width                    -0.092631
         height                   -0.275686
         curb\_weight              -0.098857
         number\_of\_cylinders      -0.077775
         engine\_size              -0.104959
         bore                     -0.046081
         stroke                   -0.065078
         compression\_ratio        -0.284202
         horsepower                0.132510
         peak\_rpm                  1.000000
         city\_mpg                 -0.127533
         highway\_mpg              -0.050670
         price                    -0.006104
         relacao\_estrada\_cidade    0.317387
         Name: peak\_rpm, dtype: float64
\end{Verbatim}
            
    Há valores nulos para peak\_rpm, city\_mpg e highway\_mpg.

Para os valores de city\_mpg e highway\_mpg, por haver forte correlação
entre essas variáveis é possível realizar uma estimativa de um valor a
partir do outro.

No caso de peak\_rpm, não há essa característica percebida.

    Para os dados considerados nulos, realize a imputação de dados
utilizando um valor apropriado (note que pode haver dados paramétricos e
dados numéricos). Justique sua resposta.

    Para o caso das varíaveis city\_mpg e highway\_mpg a estimativa será
numérica estimada em função um do outro atributo, será assumido que há
uma proporção média entre essas duas variáveis.

Para o caso da variável peak\_rpm será considerada a substituição
simples pela média do valor.

    \begin{Verbatim}[commandchars=\\\{\}]
{\color{incolor}In [{\color{incolor}17}]:} \PY{c+c1}{\PYZsh{}calcula a média da relação entre o consumo da estrada / consumo da cidade }
         \PY{n}{media\PYZus{}estrada\PYZus{}cidade} \PY{o}{=} \PY{n}{automobile}\PY{p}{[}\PY{l+s+s2}{\PYZdq{}}\PY{l+s+s2}{relacao\PYZus{}estrada\PYZus{}cidade}\PY{l+s+s2}{\PYZdq{}}\PY{p}{]}\PY{o}{.}\PY{n}{mean}\PY{p}{(}\PY{n}{skipna}\PY{o}{=}\PY{k+kc}{True}\PY{p}{)}
         \PY{n}{media\PYZus{}estrada\PYZus{}cidade}
\end{Verbatim}


\begin{Verbatim}[commandchars=\\\{\}]
{\color{outcolor}Out[{\color{outcolor}17}]:} 1.2318479649443899
\end{Verbatim}
            
    \begin{Verbatim}[commandchars=\\\{\}]
{\color{incolor}In [{\color{incolor}18}]:} \PY{c+c1}{\PYZsh{}Converte o consumo na cidade quando null em função do consumo da estrada }
         \PY{k}{for} \PY{n}{i}\PY{p}{,}\PY{n}{row} \PY{o+ow}{in} \PY{n}{automobile}\PY{o}{.}\PY{n}{loc}\PY{p}{[}\PY{n}{automobile}\PY{p}{[}\PY{l+s+s2}{\PYZdq{}}\PY{l+s+s2}{city\PYZus{}mpg}\PY{l+s+s2}{\PYZdq{}}\PY{p}{]}\PY{o}{.}\PY{n}{isnull}\PY{p}{(}\PY{p}{)}\PY{p}{,}\PY{p}{:}\PY{p}{]}\PY{o}{.}\PY{n}{iterrows}\PY{p}{(}\PY{p}{)}\PY{p}{:}
             \PY{n}{automobile}\PY{o}{.}\PY{n}{loc}\PY{p}{[}\PY{n}{i}\PY{p}{,}\PY{l+s+s2}{\PYZdq{}}\PY{l+s+s2}{city\PYZus{}mpg}\PY{l+s+s2}{\PYZdq{}}\PY{p}{]} \PY{o}{=} \PY{n}{automobile}\PY{o}{.}\PY{n}{loc}\PY{p}{[}\PY{n}{i}\PY{p}{,}\PY{l+s+s2}{\PYZdq{}}\PY{l+s+s2}{highway\PYZus{}mpg}\PY{l+s+s2}{\PYZdq{}}\PY{p}{]} \PY{o}{/} \PY{n}{media\PYZus{}estrada\PYZus{}cidade}
             
         
         \PY{c+c1}{\PYZsh{}Converte o consumo na estrada quando null em função do consumo da cidade }
         \PY{k}{for} \PY{n}{i}\PY{p}{,}\PY{n}{row} \PY{o+ow}{in} \PY{n}{automobile}\PY{o}{.}\PY{n}{loc}\PY{p}{[}\PY{n}{automobile}\PY{p}{[}\PY{l+s+s2}{\PYZdq{}}\PY{l+s+s2}{highway\PYZus{}mpg}\PY{l+s+s2}{\PYZdq{}}\PY{p}{]}\PY{o}{.}\PY{n}{isnull}\PY{p}{(}\PY{p}{)}\PY{p}{,}\PY{p}{:}\PY{p}{]}\PY{o}{.}\PY{n}{iterrows}\PY{p}{(}\PY{p}{)}\PY{p}{:}
             \PY{n}{automobile}\PY{o}{.}\PY{n}{loc}\PY{p}{[}\PY{n}{i}\PY{p}{,}\PY{l+s+s2}{\PYZdq{}}\PY{l+s+s2}{highway\PYZus{}mpg}\PY{l+s+s2}{\PYZdq{}}\PY{p}{]} \PY{o}{=} \PY{n}{automobile}\PY{o}{.}\PY{n}{loc}\PY{p}{[}\PY{n}{i}\PY{p}{,}\PY{l+s+s2}{\PYZdq{}}\PY{l+s+s2}{city\PYZus{}mpg}\PY{l+s+s2}{\PYZdq{}}\PY{p}{]} \PY{o}{*} \PY{n}{media\PYZus{}estrada\PYZus{}cidade}
\end{Verbatim}


    \begin{Verbatim}[commandchars=\\\{\}]
{\color{incolor}In [{\color{incolor}19}]:} \PY{c+c1}{\PYZsh{}Tratamento para os carros com peak\PYZus{}rpm nulo}
         
         \PY{c+c1}{\PYZsh{}Seleciona os carros com peak\PYZus{}rpm que não são nulos e nem tem valor de peak\PYZus{}rpm com valores iguais a zero }
         \PY{n}{peak\PYZus{}rpm\PYZus{}tratado} \PY{o}{=} \PY{n}{automobile}\PY{p}{[}\PY{p}{(}\PY{n}{automobile}\PY{p}{[}\PY{l+s+s2}{\PYZdq{}}\PY{l+s+s2}{peak\PYZus{}rpm}\PY{l+s+s2}{\PYZdq{}}\PY{p}{]}\PY{o}{.}\PY{n}{notnull}\PY{p}{(}\PY{p}{)}\PY{p}{)} \PY{o}{\PYZam{}} \PY{p}{(}\PY{n}{automobile}\PY{p}{[}\PY{l+s+s2}{\PYZdq{}}\PY{l+s+s2}{peak\PYZus{}rpm}\PY{l+s+s2}{\PYZdq{}}\PY{p}{]} \PY{o}{\PYZgt{}} \PY{l+m+mi}{0}\PY{p}{)}\PY{p}{]}
         
         
         \PY{c+c1}{\PYZsh{}Obtem a média do peak\PYZus{}rpm}
         \PY{n}{peak\PYZus{}rpm\PYZus{}medio} \PY{o}{=} \PY{n}{peak\PYZus{}rpm\PYZus{}tratado}\PY{p}{[}\PY{l+s+s2}{\PYZdq{}}\PY{l+s+s2}{peak\PYZus{}rpm}\PY{l+s+s2}{\PYZdq{}}\PY{p}{]}\PY{o}{.}\PY{n}{mean}\PY{p}{(}\PY{p}{)}
         
         \PY{k}{for} \PY{n}{i}\PY{p}{,}\PY{n}{row} \PY{o+ow}{in} \PY{n}{automobile}\PY{o}{.}\PY{n}{loc}\PY{p}{[}\PY{n}{automobile}\PY{p}{[}\PY{l+s+s2}{\PYZdq{}}\PY{l+s+s2}{peak\PYZus{}rpm}\PY{l+s+s2}{\PYZdq{}}\PY{p}{]}\PY{o}{.}\PY{n}{isnull}\PY{p}{(}\PY{p}{)}\PY{p}{,}\PY{p}{:}\PY{p}{]}\PY{o}{.}\PY{n}{iterrows}\PY{p}{(}\PY{p}{)}\PY{p}{:}
             \PY{n}{automobile}\PY{o}{.}\PY{n}{loc}\PY{p}{[}\PY{n}{i}\PY{p}{,}\PY{l+s+s2}{\PYZdq{}}\PY{l+s+s2}{peak\PYZus{}rpm}\PY{l+s+s2}{\PYZdq{}}\PY{p}{]} \PY{o}{=} \PY{n}{peak\PYZus{}rpm\PYZus{}medio}
\end{Verbatim}


    \subsubsection{Análise de valores com valores iguais a
0}\label{anuxe1lise-de-valores-com-valores-iguais-a-0}

    Analise se no conjunto de dados se há valores iguais a 0 e verifique se
faz parte do contexto. Caso não faça parte do contexto, utilize alguma
técnica de imputação de dados apropriada.

Inspecione o dataset por dados iguais a 0.

    \begin{Verbatim}[commandchars=\\\{\}]
{\color{incolor}In [{\color{incolor}20}]:} \PY{c+c1}{\PYZsh{}Lista as colunas que possuem valores zerados }
         \PY{k}{for} \PY{n}{col} \PY{o+ow}{in} \PY{n}{cols\PYZus{}numericas}\PY{p}{:} 
             \PY{n}{vet\PYZus{}0} \PY{o}{=} \PY{n}{automobile}\PY{p}{[}\PY{n}{automobile}\PY{p}{[}\PY{n}{col}\PY{p}{]} \PY{o}{==} \PY{l+m+mi}{0}\PY{p}{]}
             \PY{k}{if} \PY{p}{(}\PY{n+nb}{len}\PY{p}{(}\PY{n}{vet\PYZus{}0}\PY{p}{)} \PY{o}{\PYZgt{}} \PY{l+m+mi}{0}\PY{p}{)}\PY{p}{:}
                \PY{n+nb}{print}\PY{p}{(}\PY{l+s+s2}{\PYZdq{}}\PY{l+s+s2}{Coluna = }\PY{l+s+s2}{\PYZdq{}} \PY{o}{+} \PY{n}{col} \PY{o}{+} \PY{l+s+s2}{\PYZdq{}}\PY{l+s+s2}{ com valores zerados}\PY{l+s+se}{\PYZbs{}n}\PY{l+s+s2}{\PYZdq{}}\PY{p}{)}
                \PY{n+nb}{print}\PY{p}{(}\PY{n}{automobile}\PY{p}{[}\PY{n}{automobile}\PY{p}{[}\PY{n}{col}\PY{p}{]} \PY{o}{==}\PY{l+m+mi}{0}\PY{p}{]}\PY{p}{[}\PY{n}{col}\PY{p}{]}\PY{p}{)}
                \PY{n+nb}{print}\PY{p}{(}\PY{l+s+s2}{\PYZdq{}}\PY{l+s+se}{\PYZbs{}n}\PY{l+s+se}{\PYZbs{}n}\PY{l+s+s2}{\PYZdq{}}\PY{p}{)}
\end{Verbatim}


    \begin{Verbatim}[commandchars=\\\{\}]
Coluna = peak\_rpm com valores zerados

32    0.0
35    0.0
Name: peak\_rpm, dtype: float64



Coluna = city\_mpg com valores zerados

66     0.0
181    0.0
Name: city\_mpg, dtype: float64



Coluna = highway\_mpg com valores zerados

67    0.0
Name: highway\_mpg, dtype: float64




    \end{Verbatim}

    A imputação de dados pode seguir algum padrão dos demais exemplos ou
pode ser simplesmente atribuído um valor. Avalie tais condições de
acordo com as inspeções de dados.

    Os mesmos atributos que apresentaram valores nulos também apresentam
valores zerados.

Os atributos citados não deveriam ter valores zerados, pois um veículo
com motor não pode ter o número máximo de rotações por minuto
(peak\_rpm) sendo 0 e nem ter o consumo de combustível nem na estrada e
nem na cidade (highway\_mpg, city\_mpg) como sendo iguais a zero.

O tratamento sugerido será o mesmo aplicado no caso de valores nulos.

Para os atributos city\_mpg e highway\_mpg teremos um cálculo em função
do outro e o valor do peak\_rpm será substituído pelo valor médio.

    \begin{Verbatim}[commandchars=\\\{\}]
{\color{incolor}In [{\color{incolor}21}]:} \PY{c+c1}{\PYZsh{}Converte o consumo na cidade quando zero em função do consumo da estrada }
         \PY{k}{for} \PY{n}{i}\PY{p}{,}\PY{n}{row} \PY{o+ow}{in} \PY{n}{automobile}\PY{o}{.}\PY{n}{loc}\PY{p}{[}\PY{n}{automobile}\PY{p}{[}\PY{l+s+s2}{\PYZdq{}}\PY{l+s+s2}{city\PYZus{}mpg}\PY{l+s+s2}{\PYZdq{}}\PY{p}{]}\PY{o}{==}\PY{l+m+mi}{0}\PY{p}{,}\PY{p}{:}\PY{p}{]}\PY{o}{.}\PY{n}{iterrows}\PY{p}{(}\PY{p}{)}\PY{p}{:}
             \PY{n}{automobile}\PY{o}{.}\PY{n}{loc}\PY{p}{[}\PY{n}{i}\PY{p}{,}\PY{l+s+s2}{\PYZdq{}}\PY{l+s+s2}{city\PYZus{}mpg}\PY{l+s+s2}{\PYZdq{}}\PY{p}{]} \PY{o}{=} \PY{n}{automobile}\PY{o}{.}\PY{n}{loc}\PY{p}{[}\PY{n}{i}\PY{p}{,}\PY{l+s+s2}{\PYZdq{}}\PY{l+s+s2}{highway\PYZus{}mpg}\PY{l+s+s2}{\PYZdq{}}\PY{p}{]} \PY{o}{/} \PY{n}{media\PYZus{}estrada\PYZus{}cidade}
             
         
         \PY{c+c1}{\PYZsh{}Converte o consumo na estrada quando zero em função do consumo da cidade }
         \PY{k}{for} \PY{n}{i}\PY{p}{,}\PY{n}{row} \PY{o+ow}{in} \PY{n}{automobile}\PY{o}{.}\PY{n}{loc}\PY{p}{[}\PY{n}{automobile}\PY{p}{[}\PY{l+s+s2}{\PYZdq{}}\PY{l+s+s2}{highway\PYZus{}mpg}\PY{l+s+s2}{\PYZdq{}}\PY{p}{]}\PY{o}{==}\PY{l+m+mi}{0}\PY{p}{,}\PY{p}{:}\PY{p}{]}\PY{o}{.}\PY{n}{iterrows}\PY{p}{(}\PY{p}{)}\PY{p}{:}
             \PY{n}{automobile}\PY{o}{.}\PY{n}{loc}\PY{p}{[}\PY{n}{i}\PY{p}{,}\PY{l+s+s2}{\PYZdq{}}\PY{l+s+s2}{highway\PYZus{}mpg}\PY{l+s+s2}{\PYZdq{}}\PY{p}{]} \PY{o}{=} \PY{n}{automobile}\PY{o}{.}\PY{n}{loc}\PY{p}{[}\PY{n}{i}\PY{p}{,}\PY{l+s+s2}{\PYZdq{}}\PY{l+s+s2}{city\PYZus{}mpg}\PY{l+s+s2}{\PYZdq{}}\PY{p}{]} \PY{o}{*} \PY{n}{media\PYZus{}estrada\PYZus{}cidade}
\end{Verbatim}


    \begin{Verbatim}[commandchars=\\\{\}]
{\color{incolor}In [{\color{incolor}22}]:} \PY{c+c1}{\PYZsh{}Converte o peak\PYZus{}rpm para o valor médio quando o valor é zero}
         \PY{k}{for} \PY{n}{i}\PY{p}{,}\PY{n}{row} \PY{o+ow}{in} \PY{n}{automobile}\PY{o}{.}\PY{n}{loc}\PY{p}{[}\PY{n}{automobile}\PY{p}{[}\PY{l+s+s2}{\PYZdq{}}\PY{l+s+s2}{peak\PYZus{}rpm}\PY{l+s+s2}{\PYZdq{}}\PY{p}{]}\PY{o}{==}\PY{l+m+mi}{0}\PY{p}{,}\PY{p}{:}\PY{p}{]}\PY{o}{.}\PY{n}{iterrows}\PY{p}{(}\PY{p}{)}\PY{p}{:}
             \PY{n}{automobile}\PY{o}{.}\PY{n}{loc}\PY{p}{[}\PY{n}{i}\PY{p}{,}\PY{l+s+s2}{\PYZdq{}}\PY{l+s+s2}{peak\PYZus{}rpm}\PY{l+s+s2}{\PYZdq{}}\PY{p}{]} \PY{o}{=} \PY{n}{peak\PYZus{}rpm\PYZus{}medio}
\end{Verbatim}


    \subsubsection{Análise Numérica de
Outliers}\label{anuxe1lise-numuxe9rica-de-outliers}

    Da análise visual de outliers realizada acima vamos utilizar uma métrica
de verificação mais apropriada e objetiva afim de criar um patamar
aceitável de dados não discrepantes.

Neste projeto vamos considerar
\href{http://datapigtechnologies.com/blog/index.php/highlighting-outliers-in-your-data-with-the-tukey-method/}{o
Método Turco para identificar discrepantes}, que utiliza um
\textbf{passo de limite} que é 5 vezes (em nosso projeto) a diferença
entre o terceiro (Q3) e o primeiro quartil (Q1). Deste modo, valores que
sejam maiores que o Q3 + passo de limite ou menor Q1 - passo de limite
sejam sinalizados como outliers.

Construa uma função que receba como parâmetro de entrada um série de
dados e exiba os valores discrepantes. Utilize o boilerplate abaixo para
completar as instruções faltantes.

\emph{Dica: utilize a função do Numpy de percentil, np.percentile(serie,
25) para quartil 1 e np.percentile(serie, 75) para quartil 3.}

    \begin{Verbatim}[commandchars=\\\{\}]
{\color{incolor}In [{\color{incolor}23}]:} \PY{k}{def} \PY{n+nf}{identificacao\PYZus{}outlier}\PY{p}{(}\PY{n}{df}\PY{p}{,} \PY{n}{col}\PY{p}{)}\PY{p}{:} 
             \PY{n+nb}{print}\PY{p}{(}\PY{n}{col}\PY{p}{)}
          
             \PY{n}{estat} \PY{o}{=} \PY{n}{df}\PY{o}{.}\PY{n}{describe}\PY{p}{(}\PY{p}{)}
             \PY{c+c1}{\PYZsh{}Intervalo inter\PYZhy{}quartil}
             \PY{n}{iq} \PY{o}{=} \PY{p}{(}\PY{n}{estat}\PY{p}{[}\PY{n}{col}\PY{p}{]}\PY{p}{[}\PY{l+s+s2}{\PYZdq{}}\PY{l+s+s2}{75}\PY{l+s+s2}{\PYZpc{}}\PY{l+s+s2}{\PYZdq{}}\PY{p}{]} \PY{o}{\PYZhy{}} \PY{n}{estat}\PY{p}{[}\PY{n}{col}\PY{p}{]}\PY{p}{[}\PY{l+s+s2}{\PYZdq{}}\PY{l+s+s2}{25}\PY{l+s+s2}{\PYZpc{}}\PY{l+s+s2}{\PYZdq{}}\PY{p}{]}\PY{p}{)}
             \PY{c+c1}{\PYZsh{}Limite superior para outlier é 5 vezes o intervalo interquartílica acima do 75\PYZpc{} percentil}
             \PY{n}{lim\PYZus{}sup} \PY{o}{=} \PY{n}{estat}\PY{p}{[}\PY{n}{col}\PY{p}{]}\PY{p}{[}\PY{l+s+s2}{\PYZdq{}}\PY{l+s+s2}{75}\PY{l+s+s2}{\PYZpc{}}\PY{l+s+s2}{\PYZdq{}}\PY{p}{]} \PY{o}{+} \PY{l+m+mi}{5} \PY{o}{*} \PY{n}{iq} 
             \PY{c+c1}{\PYZsh{}Limite inferior para outlier é 5 vezes o intervalo interquarílica abaixo do 25\PYZpc{} percentil}
             \PY{n}{lim\PYZus{}inf} \PY{o}{=} \PY{n}{estat}\PY{p}{[}\PY{n}{col}\PY{p}{]}\PY{p}{[}\PY{l+s+s2}{\PYZdq{}}\PY{l+s+s2}{25}\PY{l+s+s2}{\PYZpc{}}\PY{l+s+s2}{\PYZdq{}}\PY{p}{]} \PY{o}{\PYZhy{}} \PY{l+m+mi}{5} \PY{o}{*} \PY{n}{iq} 
             \PY{c+c1}{\PYZsh{}Seleciona os elementos que seguem a regra calculada acima}
             \PY{n}{vet\PYZus{}out\PYZus{}sup} \PY{o}{=} \PY{n}{df}\PY{p}{[}\PY{p}{(}\PY{n}{df}\PY{p}{[}\PY{n}{col}\PY{p}{]} \PY{o}{\PYZgt{}} \PY{n}{lim\PYZus{}sup}\PY{p}{)}\PY{p}{]}\PY{p}{[}\PY{p}{[}\PY{n}{col}\PY{p}{]}\PY{p}{]}
             \PY{n}{vet\PYZus{}out\PYZus{}inf} \PY{o}{=} \PY{n}{df}\PY{p}{[}\PY{p}{(}\PY{n}{df}\PY{p}{[}\PY{n}{col}\PY{p}{]} \PY{o}{\PYZlt{}} \PY{n}{lim\PYZus{}inf}\PY{p}{)}\PY{p}{]}\PY{p}{[}\PY{p}{[}\PY{n}{col}\PY{p}{]}\PY{p}{]}
             \PY{c+c1}{\PYZsh{}Se algum elemento foi selecionado, imprima}
             \PY{k}{if} \PY{p}{(}\PY{n+nb}{len}\PY{p}{(}\PY{n}{vet\PYZus{}out\PYZus{}sup}\PY{p}{)} \PY{o}{\PYZgt{}} \PY{l+m+mi}{0} \PY{p}{)}\PY{p}{:}
                \PY{n+nb}{print}\PY{p}{(}\PY{l+s+s2}{\PYZdq{}}\PY{l+s+s2}{Candidatos a outliers superiores para }\PY{l+s+s2}{\PYZdq{}} \PY{o}{+} \PY{n}{col}\PY{p}{)} 
                \PY{n+nb}{print}\PY{p}{(}\PY{n}{vet\PYZus{}out\PYZus{}sup}\PY{p}{)}
                \PY{n+nb}{print}\PY{p}{(}\PY{l+s+s2}{\PYZdq{}}\PY{l+s+se}{\PYZbs{}n}\PY{l+s+se}{\PYZbs{}n}\PY{l+s+s2}{\PYZdq{}}\PY{p}{)}
             
             \PY{k}{if} \PY{p}{(}\PY{n+nb}{len}\PY{p}{(}\PY{n}{vet\PYZus{}out\PYZus{}inf}\PY{p}{)} \PY{o}{\PYZgt{}} \PY{l+m+mi}{0} \PY{p}{)}\PY{p}{:}
                \PY{n+nb}{print}\PY{p}{(}\PY{l+s+s2}{\PYZdq{}}\PY{l+s+s2}{Candidatos a outliers inferiores para }\PY{l+s+s2}{\PYZdq{}} \PY{o}{+} \PY{n}{col}\PY{p}{)} 
                \PY{n+nb}{print}\PY{p}{(}\PY{n}{vet\PYZus{}out\PYZus{}inf}\PY{p}{)}
                \PY{n+nb}{print}\PY{p}{(}\PY{l+s+s2}{\PYZdq{}}\PY{l+s+se}{\PYZbs{}n}\PY{l+s+se}{\PYZbs{}n}\PY{l+s+s2}{\PYZdq{}}\PY{p}{)}
         
             
\end{Verbatim}


    \begin{Verbatim}[commandchars=\\\{\}]
{\color{incolor}In [{\color{incolor}24}]:} \PY{k}{for} \PY{n}{col} \PY{o+ow}{in} \PY{n}{cols\PYZus{}numericas}\PY{p}{:} 
             \PY{n}{identificacao\PYZus{}outlier}\PY{p}{(}\PY{n}{automobile}\PY{p}{,}\PY{n}{col}\PY{p}{)}
\end{Verbatim}


    \begin{Verbatim}[commandchars=\\\{\}]
number\_of\_doors
wheel\_base
length
width
height
curb\_weight
number\_of\_cylinders
Candidatos a outliers superiores para number\_of\_cylinders
     number\_of\_cylinders
2                      6
4                      5
5                      5
6                      5
7                      5
8                      5
11                     6
12                     6
13                     6
14                     6
15                     6
16                     6
44                     6
45                     6
46                    12
64                     5
65                     5
66                     5
67                     5
68                     8
69                     8
70                     8
71                     8
98                     6
99                     6
100                    6
101                    6
102                    6
103                    6
123                    6
124                    6
125                    6
174                    6
175                    6
176                    6
177                    6
187                    5
198                    6
199                    6



Candidatos a outliers inferiores para number\_of\_cylinders
    number\_of\_cylinders
17                    3
52                    2
53                    2
54                    2
55                    2



engine\_size
bore
stroke
compression\_ratio
Candidatos a outliers superiores para compression\_ratio
     compression\_ratio
60                22.7
63                22.0
64                21.5
65                21.5
66                21.5
67                21.5
87                21.9
105               21.0
107               21.0
109               21.0
111               21.0
113               21.0
154               22.5
155               22.5
170               22.5
178               23.0
180               23.0
183               23.0
188               23.0
199               23.0



horsepower
peak\_rpm
city\_mpg
highway\_mpg
price

    \end{Verbatim}

    Pelo método utilizado, há potenciais outliers para dois atributos:
número de cilindros (number\_of\_cylinders) e taxa de compressão
(compression\_ratio).

Pesquisando na Internet, encontram-se referências em vários sites
(revista 4 rodas, jornal do veículo) para motores de veículos cujo
número de cilindros variam de 2 a 12, sendo que a maioria dos veículos
possuem 4 (vide estatística abaixo). Deste modo, não faz sentido excluir
este veículos pois os números parecem passíveis de serem reais

    \begin{Verbatim}[commandchars=\\\{\}]
{\color{incolor}In [{\color{incolor}25}]:} \PY{n}{automobile}\PY{o}{.}\PY{n}{groupby}\PY{p}{(}\PY{p}{[}\PY{l+s+s2}{\PYZdq{}}\PY{l+s+s2}{number\PYZus{}of\PYZus{}cylinders}\PY{l+s+s2}{\PYZdq{}}\PY{p}{]}\PY{p}{)}\PY{p}{[}\PY{l+s+s2}{\PYZdq{}}\PY{l+s+s2}{number\PYZus{}of\PYZus{}cylinders}\PY{l+s+s2}{\PYZdq{}}\PY{p}{]}\PY{o}{.}\PY{n}{count}\PY{p}{(}\PY{p}{)}
\end{Verbatim}


\begin{Verbatim}[commandchars=\\\{\}]
{\color{outcolor}Out[{\color{outcolor}25}]:} number\_of\_cylinders
         2       4
         3       1
         4     157
         5      10
         6      24
         8       4
         12      1
         Name: number\_of\_cylinders, dtype: int64
\end{Verbatim}
            
    No caso do atributo de taxa de compressão, percebemos que todos os
elementos selecionados possuem motores a diesel, e segundo a wikipedia
(https://pt.wikipedia.org/wiki/Motor\_a\_diesel), motores a diesel tem a
taxa de compressão variando de 15 a 25, ou seja, os dados parecem
verossímeis.

    \begin{Verbatim}[commandchars=\\\{\}]
{\color{incolor}In [{\color{incolor}26}]:} \PY{n}{automobile}\PY{p}{[}\PY{n}{automobile}\PY{p}{[}\PY{l+s+s2}{\PYZdq{}}\PY{l+s+s2}{compression\PYZus{}ratio}\PY{l+s+s2}{\PYZdq{}}\PY{p}{]}\PY{o}{\PYZgt{}}\PY{o}{=}\PY{l+m+mf}{21.0}\PY{p}{]}\PY{p}{[}\PY{p}{[}\PY{l+s+s2}{\PYZdq{}}\PY{l+s+s2}{fuel\PYZus{}type}\PY{l+s+s2}{\PYZdq{}}\PY{p}{,}\PY{l+s+s2}{\PYZdq{}}\PY{l+s+s2}{compression\PYZus{}ratio}\PY{l+s+s2}{\PYZdq{}}\PY{p}{]}\PY{p}{]}
\end{Verbatim}


\begin{Verbatim}[commandchars=\\\{\}]
{\color{outcolor}Out[{\color{outcolor}26}]:}     fuel\_type  compression\_ratio
         60     diesel               22.7
         63     diesel               22.0
         64     diesel               21.5
         65     diesel               21.5
         66     diesel               21.5
         67     diesel               21.5
         87     diesel               21.9
         105    diesel               21.0
         107    diesel               21.0
         109    diesel               21.0
         111    diesel               21.0
         113    diesel               21.0
         154    diesel               22.5
         155    diesel               22.5
         170    diesel               22.5
         178    diesel               23.0
         180    diesel               23.0
         183    diesel               23.0
         188    diesel               23.0
         199    diesel               23.0
\end{Verbatim}
            
    \textbf{Pergunta:} Houve dados discrepantes localizados pela metodologia
sugerida? Qual foi a sua conclusão, são realmente dados que devem ser
removidos ou mantidos? Justifique.

\textbf{Resposta:} Sim. Pela metodologia sugerida foram localizados
dados discrepantes. Contudo os dados devem ser mantidos, pois há uma
enorme variedade no número de cilindros dos veículos, compatível com
aquilo observado na amostra e a taxa de compressão de veículos é muito
mais alta naqueles com motores à diesel, também condizentes com o
identificado na amostra

    \subsubsection{Mapeamento de Dados
Paramétricos}\label{mapeamento-de-dados-paramuxe9tricos}

    Os algoritmos de aprendizado de máquina precisam receber dados que sejam
inteiramente numéricos. Dados que representam uma classificação como por
exemplo um tipo de carro, como sedan ou hatchback, deve ser convertido
em um valor numérico associado, como por exemplo 1 ou 2.

Crie uma função que receba uma lista única e retorne um dicionário com a
categoria e um código numérico crescente e incremental para
posteriormente utilizarmos como mapeamento.

    \begin{Verbatim}[commandchars=\\\{\}]
{\color{incolor}In [{\color{incolor}27}]:} \PY{k}{def} \PY{n+nf}{mapear\PYZus{}serie}\PY{p}{(}\PY{n}{serie}\PY{p}{)}\PY{p}{:}
             \PY{n}{dict\PYZus{}gen} \PY{o}{=} \PY{p}{\PYZob{}}\PY{p}{\PYZcb{}}
             \PY{n}{valores} \PY{o}{=} \PY{n}{serie}\PY{o}{.}\PY{n}{unique}\PY{p}{(}\PY{p}{)}
             \PY{n}{ordinal} \PY{o}{=} \PY{l+m+mi}{0} 
             \PY{k}{for} \PY{n}{valor} \PY{o+ow}{in} \PY{n}{valores}\PY{p}{:}
                 \PY{n}{ordinal} \PY{o}{+}\PY{o}{=} \PY{l+m+mi}{1}
                 \PY{n}{dict\PYZus{}gen}\PY{p}{[}\PY{n}{valor}\PY{p}{]} \PY{o}{=} \PY{n}{ordinal} 
                     
             \PY{k}{return} \PY{p}{(}\PY{n}{dict\PYZus{}gen}\PY{p}{)}
\end{Verbatim}


    Com a funcão criada, crie dicionários para cada coluna paramétrica.
Lembre-se que é necessário passar somente valores únicos.

\emph{Dica: utilize a função unique() do dataframe para obter valores
únicos de uma determinada série (ou coluna).}

    \begin{Verbatim}[commandchars=\\\{\}]
{\color{incolor}In [{\color{incolor}28}]:} \PY{c+c1}{\PYZsh{}Converte todas as colunas alfanuméricas para os seus valores ordinais}
         \PY{k}{for} \PY{n}{col} \PY{o+ow}{in} \PY{n}{cols\PYZus{}alfa}\PY{p}{:}
             \PY{n}{mapa\PYZus{}ordinal} \PY{o}{=} \PY{n}{mapear\PYZus{}serie}\PY{p}{(}\PY{n}{automobile}\PY{p}{[}\PY{n}{col}\PY{p}{]}\PY{p}{)}
             \PY{n}{automobile}\PY{p}{[}\PY{n}{col}\PY{p}{]} \PY{o}{=} \PY{n}{automobile}\PY{p}{[}\PY{n}{col}\PY{p}{]}\PY{o}{.}\PY{n}{map}\PY{p}{(}\PY{n}{mapa\PYZus{}ordinal}\PY{p}{)}
\end{Verbatim}


    Até este momento seu conjunto de dados não deve conter nenhum dado
paramétrico. Todos os dados armazenados com valores como texto, por
exemplo, "diesel", "gas" deve estar preenchido com valores numéricos,
como 1 ou 2.

Inspecione seus dados e certifique de que tudo está certo.

\emph{Dica: utilize uma inspeção simples visual, com 20 amostras.}

    \begin{Verbatim}[commandchars=\\\{\}]
{\color{incolor}In [{\color{incolor}29}]:} \PY{c+c1}{\PYZsh{}Colunas marcadas como alfanuméricas agora já estão convertidas}
         \PY{n}{automobile}\PY{p}{[}\PY{n}{cols\PYZus{}alfa}\PY{p}{]}
\end{Verbatim}


\begin{Verbatim}[commandchars=\\\{\}]
{\color{outcolor}Out[{\color{outcolor}29}]:}      make  fuel\_type  aspiration  body\_style  drive\_wheels  engine\_type  \textbackslash{}
         0       1          1           1           1             1            1   
         1       1          1           1           1             1            1   
         2       1          1           1           2             1            2   
         3       2          1           1           3             2            3   
         4       2          1           1           3             3            3   
         5       2          1           1           3             2            3   
         6       2          1           1           3             2            3   
         7       2          1           1           4             2            3   
         8       2          1           2           3             2            3   
         9       3          1           1           3             1            3   
         10      3          1           1           3             1            3   
         11      3          1           1           3             1            3   
         12      3          1           1           3             1            3   
         13      3          1           1           3             1            3   
         14      3          1           1           3             1            3   
         15      3          1           1           3             1            3   
         16      3          1           1           3             1            3   
         17      4          1           1           2             2            4   
         18      4          1           1           2             2            3   
         19      4          1           1           3             2            3   
         20      5          1           1           2             2            3   
         21      5          1           1           2             2            3   
         22      5          1           2           2             2            3   
         23      5          1           1           2             2            3   
         24      5          1           1           3             2            3   
         25      5          1           1           3             2            3   
         26      5          1           2           3             2            3   
         27      5          1           1           4             2            3   
         28      5          1           2           2             2            3   
         29      6          1           1           2             2            3   
         ..    {\ldots}        {\ldots}         {\ldots}         {\ldots}           {\ldots}          {\ldots}   
         171    20          1           1           2             2            3   
         172    20          1           1           3             2            3   
         173    20          1           1           2             2            3   
         174    20          1           1           2             1            1   
         175    20          1           1           2             1            1   
         176    20          1           1           3             1            1   
         177    20          1           1           4             1            1   
         178    21          2           1           3             2            3   
         179    21          1           1           3             2            3   
         180    21          2           1           3             2            3   
         181    21          1           1           3             2            3   
         182    21          1           1           3             2            3   
         183    21          2           2           3             2            3   
         184    21          1           1           3             2            3   
         185    21          1           1           1             2            3   
         186    21          1           1           2             2            3   
         187    21          1           1           3             2            3   
         188    21          2           2           3             2            3   
         189    21          1           1           4             2            3   
         190    22          1           1           3             1            3   
         191    22          1           1           4             1            3   
         192    22          1           1           3             1            3   
         193    22          1           1           4             1            3   
         194    22          1           2           3             1            3   
         195    22          1           2           4             1            3   
         196    22          1           1           3             1            3   
         197    22          1           2           3             1            3   
         198    22          1           1           3             1            2   
         199    22          2           2           3             1            3   
         200    22          1           2           3             1            3   
         
              engine\_location  fuel\_system  
         0                  1            1  
         1                  1            1  
         2                  1            1  
         3                  1            1  
         4                  1            1  
         5                  1            1  
         6                  1            1  
         7                  1            1  
         8                  1            1  
         9                  1            1  
         10                 1            1  
         11                 1            1  
         12                 1            1  
         13                 1            1  
         14                 1            1  
         15                 1            1  
         16                 1            1  
         17                 1            2  
         18                 1            2  
         19                 1            2  
         20                 1            2  
         21                 1            2  
         22                 1            1  
         23                 1            2  
         24                 1            2  
         25                 1            2  
         26                 1            1  
         27                 1            2  
         28                 1            3  
         29                 1            4  
         ..               {\ldots}          {\ldots}  
         171                1            1  
         172                1            1  
         173                1            1  
         174                1            1  
         175                1            1  
         176                1            1  
         177                1            1  
         178                1            7  
         179                1            1  
         180                1            7  
         181                1            1  
         182                1            1  
         183                1            7  
         184                1            1  
         185                1            1  
         186                1            1  
         187                1            1  
         188                1            7  
         189                1            1  
         190                1            1  
         191                1            1  
         192                1            1  
         193                1            1  
         194                1            1  
         195                1            1  
         196                1            1  
         197                1            1  
         198                1            1  
         199                1            7  
         200                1            1  
         
         [201 rows x 8 columns]
\end{Verbatim}
            
    \begin{Verbatim}[commandchars=\\\{\}]
{\color{incolor}In [{\color{incolor}30}]:} \PY{c+c1}{\PYZsh{}Se existe a coluna auxiliar incluída para o tratamento de dados faltantes então a exclui}
         \PY{k}{if} \PY{l+s+s2}{\PYZdq{}}\PY{l+s+s2}{relacao\PYZus{}estrada\PYZus{}cidade}\PY{l+s+s2}{\PYZdq{}} \PY{o+ow}{in} \PY{n+nb}{list}\PY{p}{(}\PY{n}{automobile}\PY{o}{.}\PY{n}{columns}\PY{o}{.}\PY{n}{values}\PY{p}{)}\PY{p}{:}
            \PY{n}{automobile} \PY{o}{=} \PY{n}{automobile}\PY{o}{.}\PY{n}{drop}\PY{p}{(}\PY{l+s+s2}{\PYZdq{}}\PY{l+s+s2}{relacao\PYZus{}estrada\PYZus{}cidade}\PY{l+s+s2}{\PYZdq{}}\PY{p}{,}\PY{l+m+mi}{1}\PY{p}{)}
\end{Verbatim}


    \begin{Verbatim}[commandchars=\\\{\}]
{\color{incolor}In [{\color{incolor}31}]:} \PY{k}{def} \PY{n+nf}{is\PYZus{}numeric\PYZus{}array}\PY{p}{(}\PY{n}{array}\PY{p}{)}\PY{p}{:}
             \PY{n}{numerical\PYZus{}dtype\PYZus{}kinds} \PY{o}{=} \PY{p}{\PYZob{}}\PY{l+s+s1}{\PYZsq{}}\PY{l+s+s1}{b}\PY{l+s+s1}{\PYZsq{}}\PY{p}{,} \PY{c+c1}{\PYZsh{} boolean}
                                      \PY{l+s+s1}{\PYZsq{}}\PY{l+s+s1}{u}\PY{l+s+s1}{\PYZsq{}}\PY{p}{,} \PY{c+c1}{\PYZsh{} unsigned integer}
                                      \PY{l+s+s1}{\PYZsq{}}\PY{l+s+s1}{i}\PY{l+s+s1}{\PYZsq{}}\PY{p}{,} \PY{c+c1}{\PYZsh{} signed integer}
                                      \PY{l+s+s1}{\PYZsq{}}\PY{l+s+s1}{f}\PY{l+s+s1}{\PYZsq{}}\PY{p}{,} \PY{c+c1}{\PYZsh{} floats}
                                      \PY{l+s+s1}{\PYZsq{}}\PY{l+s+s1}{c}\PY{l+s+s1}{\PYZsq{}}\PY{p}{\PYZcb{}} \PY{c+c1}{\PYZsh{} complex}
             \PY{k}{try}\PY{p}{:}
                 \PY{k}{return} \PY{n}{array}\PY{o}{.}\PY{n}{dtype}\PY{o}{.}\PY{n}{kind} \PY{o+ow}{in} \PY{n}{numerical\PYZus{}dtype\PYZus{}kinds}
             \PY{k}{except} \PY{n+ne}{AttributeError}\PY{p}{:}
                 \PY{c+c1}{\PYZsh{} in case it\PYZsq{}s not a numpy array it will probably have no dtype.}
                 \PY{k}{return} \PY{n}{np}\PY{o}{.}\PY{n}{asarray}\PY{p}{(}\PY{n}{array}\PY{p}{)}\PY{o}{.}\PY{n}{dtype}\PY{o}{.}\PY{n}{kind} \PY{o+ow}{in} \PY{n}{numerical\PYZus{}dtype\PYZus{}kinds}
\end{Verbatim}


    \begin{Verbatim}[commandchars=\\\{\}]
{\color{incolor}In [{\color{incolor}32}]:} \PY{c+c1}{\PYZsh{}Verifica se todos os atributos dos automóveis são numéricos}
         \PY{n+nb}{print}\PY{p}{(}\PY{n}{is\PYZus{}numeric\PYZus{}array}\PY{p}{(}\PY{n}{automobile}\PY{p}{)}\PY{p}{)}
\end{Verbatim}


    \begin{Verbatim}[commandchars=\\\{\}]
True

    \end{Verbatim}

    Nesse ponto, temos a garantia que todas as colunas do dataframe
automobile são numéricas. Pode-se prosseguir para a utilização de
modelos de aprendizagem de máquina para realizar os treinamentos, testes
e predições.

    \subsection{Desenvolvimento do Modelo}\label{desenvolvimento-do-modelo}

    O conjunto de dados que temos a nossa disposição indica a aplicação de
um modelo voltado a regressão, ou seja, queremos prever um preço de um
veículo dada certas condições. É um problema típico de predição de série
numérica.

Podemos aplicar quaisquer algoritmos de regressão existente que tenha
aplicação de classificação, pois é de um domínio diferente.

Iremos explorar 3 modelos de algoritmos de regressão para testar a
performance de cada um deles. Ao final será eleito o que apresentar
melhor performance de pontuação R2.

Os algoritmos são:

\begin{enumerate}
\def\labelenumi{\arabic{enumi}.}
\tightlist
\item
  \href{http://scikit-learn.org/stable/auto_examples/linear_model/plot_ols.html}{Regressão
  Linear}
\item
  \href{http://scikit-learn.org/stable/modules/generated/sklearn.tree.DecisionTreeRegressor.html}{Regressão
  de Árvore de Decisão}
\item
  \href{http://scikit-learn.org/stable/modules/linear_model.html\#ridge-regression}{Regressão
  Ridge}
\end{enumerate}

    \textbf{Pergunta:} Explique como cada modelo de regressão funciona,
indicando pelo menos um caso de uso em cada um deles.

\textbf{Resposta:}

\begin{itemize}
\tightlist
\item
  \textbf{Regressão Linear}
\end{itemize}

Modelo que estima o valor desejado, traçando a melhor linha reta baseada
na distribuição das observações. Os coeficientes dessa linha são
calculados pelos valores que minimizam a soma residual do médoto dos
mínimos quadrados entre as observações e a estimativa. O resíduo
consiste em subtrair o valor observado com o valor estimado e elevar ao
quadrado. Assim, a regressão linear busca coeficientes que ao serem
utilizados para calcular o valor estimado produzem o menor resíduo para
cada observação.

Caso de uso: Estimar o tempo de percurso de bicicleta entre estações de
locação da cidade de Los Angeles, utilizando o dataset
\href{https://www.kaggle.com/cityofLA/los-angeles-metro-bike-share-trip-data}{``Los
Angeles Metro Bike Share Trip Data''} do Kaggle. A regressão pode ser
realizada utilizando o ID das estações de início e fim do percurso e o
horário de início da viagem.

\begin{itemize}
\tightlist
\item
  \textbf{Regressão por Árvore de Decisão}
\end{itemize}

Modelo que estima o valor desejado utilizando regras de decisão
inferidas pelos atributos. As regras podem estar no formato
``if-then-else''. Para cada decisão tomada, uma nova regra pode ser
aplicada aumentando a profundidade da árvore de decisão. Quanto mais
profunda a árvore, maior a acurácia do modelo, porém o risco da
ocorrência de overfitting é aumentado.

Caso de uso: Utilizando o dataset
\href{https://www.kaggle.com/kemical/kickstarter-projects}{``Kickstarter
Projects''} do Kaggle, estimar a chance de sucesso de um projeto.
Utilizando os nós: categoria, categoria principal, período de
arrecadação e valor pleiteado, a árvore de decisão pode combinar os nós
de modo a apresentar o melhor resultado.

\begin{itemize}
\tightlist
\item
  \textbf{Regressão Ridge}
\end{itemize}

Dependendo da quantidade de atributos e da correlação entre eles, o
método dos mínimos quadrados, utilizado na regressão linear, pode
ocasionar overfitting ou até falhar em encontrar coeficientes únicos,
pois pode atribuir pesos grandes a atributos que poderiam ser removidos
do modelo caso sejam correlacionados a outros atributos. A regressão
Ridge corrige essas falhas, adionando um parâmetro à soma residual do
método dos mínimos quadrados, isso regulariza o modelo, penalizando
coeficientes grandes, e, por consequência, encolhendo os coeficientes da
regressão.

Caso de uso: Utilizando o dataset
\href{https://www.kaggle.com/uciml/breast-cancer-wisconsin-data}{``Breast
Cancer Wisconsin Data''} do Kaggle, estimar o tipo de câncer de mama
(maligno ou benigno) baseando-se em todos os atributos do dataset. Nesse
caso, os atributos perímetro e área são calculados pelo atributo raio,
ou seja, são correlacionados, dessa forma, utilizando a regressão de
Ridge evitamos overfitting

    Antes de partimos para o treinamento do modelo, precisaremos separar os
dados em subconjuntos para permitir avaliar adequadamente o modelo.

Uma boa razão de tamanho de conjunto de treinamento e testes é 80\% e
20\% ou ainda, 70\% e 30\%. O importante é não misturar dados de
treinamento com os de teste para termos uma ideia melhor sobre a
performance do modelo com dados previamente não treinados.

Antes de separarmos os dados, a partir das análises realizadas
anteriormente, quais seriam os atributos a serem utilizados e qual seria
o atributo preditor?

    \begin{Verbatim}[commandchars=\\\{\}]
{\color{incolor}In [{\color{incolor}33}]:} \PY{c+c1}{\PYZsh{} Para a predição do preço dos automóveis, vamos utilizar os atributos com maiores correlação com relação ao atributo procurado}
         \PY{n+nb}{print}\PY{p}{(}\PY{n}{automobile}\PY{o}{.}\PY{n}{corr}\PY{p}{(}\PY{p}{)}\PY{p}{[}\PY{l+s+s2}{\PYZdq{}}\PY{l+s+s2}{price}\PY{l+s+s2}{\PYZdq{}}\PY{p}{]}\PY{o}{.}\PY{n}{sort\PYZus{}values}\PY{p}{(}\PY{n}{ascending}\PY{o}{=}\PY{k+kc}{False}\PY{p}{)}\PY{o}{.}\PY{n}{round}\PY{p}{(}\PY{l+m+mi}{2}\PY{p}{)}\PY{o}{*}\PY{l+m+mi}{100}\PY{p}{)}
         
         \PY{c+c1}{\PYZsh{} Os atributos são: drive\PYZus{}wheels, wheel\PYZus{}base, length, width, curb\PYZus{}weight, number\PYZus{}of\PYZus{}cylinders, engine\PYZus{}size, bore, horsepower,}
         \PY{c+c1}{\PYZsh{} city\PYZus{}mpg e highway\PYZus{}mpg}
         \PY{n}{feature\PYZus{}col\PYZus{}names} \PY{o}{=} \PY{p}{[}\PY{l+s+s1}{\PYZsq{}}\PY{l+s+s1}{drive\PYZus{}wheels}\PY{l+s+s1}{\PYZsq{}}\PY{p}{,}\PY{l+s+s1}{\PYZsq{}}\PY{l+s+s1}{wheel\PYZus{}base}\PY{l+s+s1}{\PYZsq{}}\PY{p}{,}\PY{l+s+s1}{\PYZsq{}}\PY{l+s+s1}{length}\PY{l+s+s1}{\PYZsq{}}\PY{p}{,}\PY{l+s+s1}{\PYZsq{}}\PY{l+s+s1}{width}\PY{l+s+s1}{\PYZsq{}}\PY{p}{,}\PY{l+s+s1}{\PYZsq{}}\PY{l+s+s1}{curb\PYZus{}weight}\PY{l+s+s1}{\PYZsq{}}\PY{p}{,}\PY{l+s+s1}{\PYZsq{}}\PY{l+s+s1}{number\PYZus{}of\PYZus{}cylinders}\PY{l+s+s1}{\PYZsq{}}\PY{p}{,}\PY{l+s+s1}{\PYZsq{}}\PY{l+s+s1}{engine\PYZus{}size}\PY{l+s+s1}{\PYZsq{}}\PY{p}{,}\PY{l+s+s1}{\PYZsq{}}\PY{l+s+s1}{bore}\PY{l+s+s1}{\PYZsq{}}\PY{p}{,}\PY{l+s+s1}{\PYZsq{}}\PY{l+s+s1}{horsepower}\PY{l+s+s1}{\PYZsq{}}\PY{p}{,}\PY{l+s+s1}{\PYZsq{}}\PY{l+s+s1}{city\PYZus{}mpg}\PY{l+s+s1}{\PYZsq{}}\PY{p}{,}\PY{l+s+s1}{\PYZsq{}}\PY{l+s+s1}{highway\PYZus{}mpg}\PY{l+s+s1}{\PYZsq{}}\PY{p}{]}
\end{Verbatim}


    \begin{Verbatim}[commandchars=\\\{\}]
price                  100.0
engine\_size             87.0
curb\_weight             83.0
horsepower              81.0
width                   75.0
number\_of\_cylinders     71.0
length                  69.0
wheel\_base              58.0
bore                    54.0
engine\_location         33.0
aspiration              18.0
body\_style              18.0
height                  14.0
fuel\_type               11.0
stroke                   8.0
compression\_ratio        7.0
number\_of\_doors          5.0
peak\_rpm               -10.0
fuel\_system            -12.0
engine\_type            -14.0
make                   -16.0
drive\_wheels           -59.0
city\_mpg               -69.0
highway\_mpg            -70.0
Name: price, dtype: float64

    \end{Verbatim}

    Crie subsets de treinamento e teste utilizado uma razão adequada de
tamanho. Utilze o \texttt{train\_test\_split} passando como parâmetros

    \begin{Verbatim}[commandchars=\\\{\}]
{\color{incolor}In [{\color{incolor}34}]:} \PY{k+kn}{from} \PY{n+nn}{sklearn}\PY{n+nn}{.}\PY{n+nn}{model\PYZus{}selection} \PY{k}{import} \PY{n}{train\PYZus{}test\PYZus{}split}
         
         \PY{c+c1}{\PYZsh{} armazena os atributos preditores}
         \PY{n}{X} \PY{o}{=} \PY{n}{automobile}\PY{p}{[}\PY{n}{feature\PYZus{}col\PYZus{}names}\PY{p}{]}\PY{o}{.}\PY{n}{values}
         \PY{c+c1}{\PYZsh{} armazena os valores a estimar}
         \PY{n}{y} \PY{o}{=} \PY{n}{automobile}\PY{p}{[}\PY{l+s+s1}{\PYZsq{}}\PY{l+s+s1}{price}\PY{l+s+s1}{\PYZsq{}}\PY{p}{]}\PY{o}{.}\PY{n}{values}
         
         \PY{c+c1}{\PYZsh{} define que 30\PYZpc{} do data set será utilizado para teste e 70\PYZpc{} treinamento}
         \PY{n}{split\PYZus{}test\PYZus{}size} \PY{o}{=} \PY{l+m+mf}{0.30}
         
         \PY{c+c1}{\PYZsh{} gera randomicamente os data sets de treinamento e teste, utilizando como seed o numero 42}
         \PY{n}{X\PYZus{}train}\PY{p}{,} \PY{n}{X\PYZus{}test}\PY{p}{,} \PY{n}{y\PYZus{}train}\PY{p}{,} \PY{n}{y\PYZus{}test} \PY{o}{=} \PY{n}{train\PYZus{}test\PYZus{}split}\PY{p}{(}\PY{n}{X}\PY{p}{,} \PY{n}{y}\PY{p}{,} \PY{n}{test\PYZus{}size}\PY{o}{=}\PY{n}{split\PYZus{}test\PYZus{}size}\PY{p}{,} \PY{n}{random\PYZus{}state}\PY{o}{=}\PY{l+m+mi}{42}\PY{p}{)}
\end{Verbatim}


    Inspecione cada subconjunto de dados obtidos do
\texttt{train\_test\_split}. Note que nos conjuntos X devemos ter
atributos, logo é esperado uma matriz com mais de uma coluna. Nos
conjuntos y, é a classe de predição, logo é esperado apenas um atributo.

    \begin{Verbatim}[commandchars=\\\{\}]
{\color{incolor}In [{\color{incolor}35}]:} \PY{c+c1}{\PYZsh{}Inspeção de X\PYZus{}train}
         \PY{n+nb}{print}\PY{p}{(}\PY{l+s+s2}{\PYZdq{}}\PY{l+s+s2}{Atributos de treinamento: }\PY{l+s+s2}{\PYZdq{}}\PY{p}{)}
         \PY{n+nb}{print}\PY{p}{(}\PY{n}{X\PYZus{}train}\PY{p}{[}\PY{l+m+mi}{0}\PY{p}{:}\PY{l+m+mi}{6}\PY{p}{]}\PY{p}{)}
         
         \PY{c+c1}{\PYZsh{} array com arrays de 11 valores, a mesma quantidade de atributos escolhida}
\end{Verbatim}


    \begin{Verbatim}[commandchars=\\\{\}]
Atributos de treinamento: 
[[  2.00000000e+00   9.72000000e+01   1.73400000e+02   6.52000000e+01
    2.30200000e+03   4.00000000e+00   1.20000000e+02   3.33000000e+00
    9.70000000e+01   2.70000000e+01   3.40000000e+01]
 [  2.00000000e+00   9.37000000e+01   1.50000000e+02   6.40000000e+01
    1.83700000e+03   4.00000000e+00   7.90000000e+01   2.91000000e+00
    6.00000000e+01   3.80000000e+01   4.20000000e+01]
 [  1.00000000e+00   1.01200000e+02   1.76800000e+02   6.48000000e+01
    2.76500000e+03   6.00000000e+00   1.64000000e+02   3.31000000e+00
    1.21000000e+02   2.10000000e+01   2.80000000e+01]
 [  2.00000000e+00   9.65000000e+01   1.57100000e+02   6.39000000e+01
    2.02400000e+03   4.00000000e+00   9.20000000e+01   2.92000000e+00
    7.60000000e+01   3.00000000e+01   3.40000000e+01]
 [  2.00000000e+00   9.37000000e+01   1.67300000e+02   6.38000000e+01
    2.19100000e+03   4.00000000e+00   9.80000000e+01   2.97000000e+00
    6.80000000e+01   3.10000000e+01   3.80000000e+01]
 [  1.00000000e+00   9.43000000e+01   1.70700000e+02   6.18000000e+01
    2.33700000e+03   4.00000000e+00   1.11000000e+02   3.31000000e+00
    7.80000000e+01   2.40000000e+01   2.90000000e+01]]

    \end{Verbatim}

    \begin{Verbatim}[commandchars=\\\{\}]
{\color{incolor}In [{\color{incolor}36}]:} \PY{c+c1}{\PYZsh{}Inspeção de y\PYZus{}train}
         \PY{n+nb}{print}\PY{p}{(}\PY{l+s+s2}{\PYZdq{}}\PY{l+s+s2}{Atributo procurado: }\PY{l+s+s2}{\PYZdq{}}\PY{p}{)}
         \PY{n+nb}{print}\PY{p}{(}\PY{n}{y\PYZus{}train}\PY{p}{[}\PY{l+m+mi}{0}\PY{p}{:}\PY{l+m+mi}{6}\PY{p}{]}\PY{p}{)}
\end{Verbatim}


    \begin{Verbatim}[commandchars=\\\{\}]
Atributo procurado: 
[ 9549  5399 21105  7295  7609  6785]

    \end{Verbatim}

    Verifique também se a razão dos conjuntos está coerente com a divisão
estabelecida. Para estes cálculos divida o número de itens do conjunto
de treino pelo total e também o de testes pelo total.

    \begin{Verbatim}[commandchars=\\\{\}]
{\color{incolor}In [{\color{incolor}37}]:} \PY{c+c1}{\PYZsh{}IMPLEMENTAÇÃO}
         \PY{n+nb}{print}\PY{p}{(}\PY{l+s+s2}{\PYZdq{}}\PY{l+s+s2}{Relação dos dados de treinamento com o conjunto total: }\PY{l+s+si}{\PYZpc{}.2f}\PY{l+s+s2}{\PYZdq{}} \PY{o}{\PYZpc{}} \PY{p}{(}\PY{n+nb}{len}\PY{p}{(}\PY{n}{X\PYZus{}train}\PY{p}{)} \PY{o}{/} \PY{n}{automobile}\PY{p}{[}\PY{l+s+s1}{\PYZsq{}}\PY{l+s+s1}{price}\PY{l+s+s1}{\PYZsq{}}\PY{p}{]}\PY{o}{.}\PY{n}{size}\PY{p}{)} \PY{p}{)}
         \PY{n+nb}{print}\PY{p}{(}\PY{l+s+s2}{\PYZdq{}}\PY{l+s+s2}{Relação dos dados de teste com o conjunto total: }\PY{l+s+si}{\PYZpc{}.2f}\PY{l+s+s2}{\PYZdq{}} \PY{o}{\PYZpc{}} \PY{p}{(}\PY{n+nb}{len}\PY{p}{(}\PY{n}{X\PYZus{}test}\PY{p}{)} \PY{o}{/} \PY{n}{automobile}\PY{p}{[}\PY{l+s+s1}{\PYZsq{}}\PY{l+s+s1}{price}\PY{l+s+s1}{\PYZsq{}}\PY{p}{]}\PY{o}{.}\PY{n}{size}\PY{p}{)} \PY{p}{)}
\end{Verbatim}


    \begin{Verbatim}[commandchars=\\\{\}]
Relação dos dados de treinamento com o conjunto total: 0.70
Relação dos dados de teste com o conjunto total: 0.30

    \end{Verbatim}

    \subsection{Treinamento e teste do
modelo}\label{treinamento-e-teste-do-modelo}

Após separarmos os dados adequadamente, selecionar os atributos que irão
compor como o modelo deve treinar e qual atributo deve perseguir, o
próximo passo é treinar este modelo e verificar, pelos testes, sua
performance.

Este estudo também irá levar a escolhermos qual algoritmo devemos
utilizar dentre os três selecionados neste ajuste.

Para avaliarmos a performance dos modelos, vamos criar uma função que
determinará a pontuação R2.

Não esqueça de avaliar os conjuntos de dados de treino
(\texttt{X\_train}, \texttt{y\_train} e de teste, \texttt{X\_test} e
\texttt{y\_test})

    \begin{Verbatim}[commandchars=\\\{\}]
{\color{incolor}In [{\color{incolor}38}]:} \PY{k+kn}{from} \PY{n+nn}{sklearn}\PY{n+nn}{.}\PY{n+nn}{metrics} \PY{k}{import} \PY{n}{mean\PYZus{}squared\PYZus{}error}\PY{p}{,} \PY{n}{r2\PYZus{}score}
         
         \PY{k}{def} \PY{n+nf}{pontuacao}\PY{p}{(}\PY{n}{modelo}\PY{p}{,} \PY{n}{X\PYZus{}test}\PY{p}{,} \PY{n}{y\PYZus{}test}\PY{p}{)}\PY{p}{:}
             \PY{n}{y\PYZus{}pred} \PY{o}{=} \PY{n}{modelo}\PY{o}{.}\PY{n}{predict}\PY{p}{(}\PY{n}{X\PYZus{}test}\PY{p}{)}
             \PY{n+nb}{print}\PY{p}{(}\PY{l+s+s2}{\PYZdq{}}\PY{l+s+si}{\PYZpc{}s}\PY{l+s+s2}{ R2 score: }\PY{l+s+si}{\PYZpc{}.2f}\PY{l+s+s2}{\PYZdq{}} \PY{o}{\PYZpc{}}\PY{p}{(}\PY{n}{modelo}\PY{p}{,}\PY{n}{r2\PYZus{}score}\PY{p}{(}\PY{n}{y\PYZus{}test}\PY{p}{,} \PY{n}{y\PYZus{}pred}\PY{p}{)}\PY{p}{)}\PY{p}{)}
\end{Verbatim}


    \subsubsection{Regressão Linear}\label{regressuxe3o-linear}

Utilize para a variável preditora a função \texttt{.ravel()} para
converter os dados no formato que o \texttt{sklearn} espera.

    \begin{Verbatim}[commandchars=\\\{\}]
{\color{incolor}In [{\color{incolor}39}]:} \PY{k+kn}{from} \PY{n+nn}{sklearn} \PY{k}{import} \PY{n}{linear\PYZus{}model}
         
         \PY{n}{lr\PYZus{}model} \PY{o}{=} \PY{n}{linear\PYZus{}model}\PY{o}{.}\PY{n}{LinearRegression}\PY{p}{(}\PY{n}{normalize}\PY{o}{=}\PY{k+kc}{True}\PY{p}{)}
         \PY{n}{lr\PYZus{}model}\PY{o}{.}\PY{n}{fit}\PY{p}{(}\PY{n}{X\PYZus{}train}\PY{p}{,} \PY{n}{y\PYZus{}train}\PY{o}{.}\PY{n}{ravel}\PY{p}{(}\PY{p}{)}\PY{p}{)}
\end{Verbatim}


\begin{Verbatim}[commandchars=\\\{\}]
{\color{outcolor}Out[{\color{outcolor}39}]:} LinearRegression(copy\_X=True, fit\_intercept=True, n\_jobs=1, normalize=True)
\end{Verbatim}
            
    Obténha a pontuação R2, utilizando a função \texttt{pontuacao} para os
conjuntos de treino e de teste.

    \begin{Verbatim}[commandchars=\\\{\}]
{\color{incolor}In [{\color{incolor}40}]:} \PY{n+nb}{print}\PY{p}{(}\PY{l+s+s2}{\PYZdq{}}\PY{l+s+s2}{Pontuação R2 para os dados de treinamento:}\PY{l+s+s2}{\PYZdq{}}\PY{p}{)}
         \PY{n}{pontuacao}\PY{p}{(}\PY{n}{lr\PYZus{}model}\PY{p}{,} \PY{n}{X\PYZus{}train}\PY{p}{,} \PY{n}{y\PYZus{}train}\PY{p}{)}
         \PY{n+nb}{print}\PY{p}{(}\PY{l+s+s2}{\PYZdq{}}\PY{l+s+se}{\PYZbs{}n}\PY{l+s+s2}{\PYZdq{}}\PY{p}{)}
         \PY{n+nb}{print}\PY{p}{(}\PY{l+s+s2}{\PYZdq{}}\PY{l+s+s2}{Pontuação R2 para os dados de test:}\PY{l+s+s2}{\PYZdq{}}\PY{p}{)}
         \PY{n}{pontuacao}\PY{p}{(}\PY{n}{lr\PYZus{}model}\PY{p}{,} \PY{n}{X\PYZus{}test}\PY{p}{,} \PY{n}{y\PYZus{}test}\PY{p}{)}
\end{Verbatim}


    \begin{Verbatim}[commandchars=\\\{\}]
Pontuação R2 para os dados de treinamento:
LinearRegression(copy\_X=True, fit\_intercept=True, n\_jobs=1, normalize=True) R2 score: 0.84


Pontuação R2 para os dados de test:
LinearRegression(copy\_X=True, fit\_intercept=True, n\_jobs=1, normalize=True) R2 score: 0.77

    \end{Verbatim}

    \begin{Verbatim}[commandchars=\\\{\}]
{\color{incolor}In [{\color{incolor}41}]:} \PY{c+c1}{\PYZsh{} Plot the results}
         \PY{n}{plt}\PY{o}{.}\PY{n}{figure}\PY{p}{(}\PY{p}{)}
         \PY{n}{plt}\PY{o}{.}\PY{n}{scatter}\PY{p}{(}\PY{n}{y\PYZus{}train}\PY{p}{,} \PY{n}{lr\PYZus{}model}\PY{o}{.}\PY{n}{predict}\PY{p}{(}\PY{n}{X\PYZus{}train}\PY{p}{)}\PY{p}{,} \PY{n}{edgecolor}\PY{o}{=}\PY{l+s+s2}{\PYZdq{}}\PY{l+s+s2}{black}\PY{l+s+s2}{\PYZdq{}}\PY{p}{,}\PY{n}{c}\PY{o}{=}\PY{l+s+s2}{\PYZdq{}}\PY{l+s+s2}{darkorange}\PY{l+s+s2}{\PYZdq{}}\PY{p}{,} \PY{n}{label}\PY{o}{=}\PY{l+s+s2}{\PYZdq{}}\PY{l+s+s2}{Treinamento}\PY{l+s+s2}{\PYZdq{}}\PY{p}{,}\PY{n}{marker}\PY{o}{=}\PY{l+s+s2}{\PYZdq{}}\PY{l+s+s2}{+}\PY{l+s+s2}{\PYZdq{}}\PY{p}{)}
         \PY{n}{plt}\PY{o}{.}\PY{n}{scatter}\PY{p}{(}\PY{n}{y\PYZus{}test}\PY{p}{,} \PY{n}{lr\PYZus{}model}\PY{o}{.}\PY{n}{predict}\PY{p}{(}\PY{n}{X\PYZus{}test}\PY{p}{)}\PY{p}{,} \PY{n}{edgecolor}\PY{o}{=}\PY{l+s+s2}{\PYZdq{}}\PY{l+s+s2}{black}\PY{l+s+s2}{\PYZdq{}}\PY{p}{,}\PY{n}{c}\PY{o}{=}\PY{l+s+s2}{\PYZdq{}}\PY{l+s+s2}{blue}\PY{l+s+s2}{\PYZdq{}}\PY{p}{,} \PY{n}{label}\PY{o}{=}\PY{l+s+s2}{\PYZdq{}}\PY{l+s+s2}{Teste}\PY{l+s+s2}{\PYZdq{}}\PY{p}{,}\PY{n}{marker}\PY{o}{=}\PY{l+s+s2}{\PYZdq{}}\PY{l+s+s2}{\PYZus{}}\PY{l+s+s2}{\PYZdq{}}\PY{p}{)}
         \PY{n}{plt}\PY{o}{.}\PY{n}{xlabel}\PY{p}{(}\PY{l+s+s2}{\PYZdq{}}\PY{l+s+s2}{Dados}\PY{l+s+s2}{\PYZdq{}}\PY{p}{)}
         \PY{n}{plt}\PY{o}{.}\PY{n}{ylabel}\PY{p}{(}\PY{l+s+s2}{\PYZdq{}}\PY{l+s+s2}{Predição}\PY{l+s+s2}{\PYZdq{}}\PY{p}{)}
         \PY{n}{plt}\PY{o}{.}\PY{n}{title}\PY{p}{(}\PY{l+s+s2}{\PYZdq{}}\PY{l+s+s2}{Regressão Linear}\PY{l+s+s2}{\PYZdq{}}\PY{p}{)}
         \PY{n}{plt}\PY{o}{.}\PY{n}{legend}\PY{p}{(}\PY{p}{)}
         \PY{n}{plt}\PY{o}{.}\PY{n}{show}\PY{p}{(}\PY{p}{)}
\end{Verbatim}


    \begin{center}
    \adjustimage{max size={0.9\linewidth}{0.9\paperheight}}{output_83_0.png}
    \end{center}
    { \hspace*{\fill} \\}
    
    \subsubsection{Regressão de árvore de
Decisão}\label{regressuxe3o-de-uxe1rvore-de-decisuxe3o}

    \begin{Verbatim}[commandchars=\\\{\}]
{\color{incolor}In [{\color{incolor}42}]:} \PY{k+kn}{from} \PY{n+nn}{sklearn} \PY{k}{import} \PY{n}{tree}
         
         \PY{n}{dtr\PYZus{}model} \PY{o}{=} \PY{n}{tree}\PY{o}{.}\PY{n}{DecisionTreeRegressor}\PY{p}{(}\PY{n}{random\PYZus{}state}\PY{o}{=}\PY{l+m+mi}{6}\PY{p}{)}\PY{o}{.}\PY{n}{fit}\PY{p}{(}\PY{n}{X\PYZus{}train}\PY{p}{,} \PY{n}{y\PYZus{}train}\PY{o}{.}\PY{n}{ravel}\PY{p}{(}\PY{p}{)}\PY{p}{)}
\end{Verbatim}


    Obténha a pontuação R2, utilizando a função \texttt{pontuacao} para os
conjuntos de treino e de teste.

    \begin{Verbatim}[commandchars=\\\{\}]
{\color{incolor}In [{\color{incolor}43}]:} \PY{n+nb}{print}\PY{p}{(}\PY{l+s+s2}{\PYZdq{}}\PY{l+s+s2}{Pontuação R2 para os dados de treinamento:}\PY{l+s+s2}{\PYZdq{}}\PY{p}{)}
         \PY{n}{pontuacao}\PY{p}{(}\PY{n}{dtr\PYZus{}model}\PY{p}{,} \PY{n}{X\PYZus{}train}\PY{p}{,} \PY{n}{y\PYZus{}train}\PY{p}{)}
         \PY{n+nb}{print}\PY{p}{(}\PY{l+s+s2}{\PYZdq{}}\PY{l+s+se}{\PYZbs{}n}\PY{l+s+s2}{\PYZdq{}}\PY{p}{)}
         \PY{n+nb}{print}\PY{p}{(}\PY{l+s+s2}{\PYZdq{}}\PY{l+s+s2}{Pontuação R2 para os dados de test:}\PY{l+s+s2}{\PYZdq{}}\PY{p}{)}
         \PY{n}{pontuacao}\PY{p}{(}\PY{n}{dtr\PYZus{}model}\PY{p}{,} \PY{n}{X\PYZus{}test}\PY{p}{,} \PY{n}{y\PYZus{}test}\PY{p}{)}
\end{Verbatim}


    \begin{Verbatim}[commandchars=\\\{\}]
Pontuação R2 para os dados de treinamento:
DecisionTreeRegressor(criterion='mse', max\_depth=None, max\_features=None,
           max\_leaf\_nodes=None, min\_impurity\_decrease=0.0,
           min\_impurity\_split=None, min\_samples\_leaf=1,
           min\_samples\_split=2, min\_weight\_fraction\_leaf=0.0,
           presort=False, random\_state=6, splitter='best') R2 score: 1.00


Pontuação R2 para os dados de test:
DecisionTreeRegressor(criterion='mse', max\_depth=None, max\_features=None,
           max\_leaf\_nodes=None, min\_impurity\_decrease=0.0,
           min\_impurity\_split=None, min\_samples\_leaf=1,
           min\_samples\_split=2, min\_weight\_fraction\_leaf=0.0,
           presort=False, random\_state=6, splitter='best') R2 score: 0.92

    \end{Verbatim}

    \begin{Verbatim}[commandchars=\\\{\}]
{\color{incolor}In [{\color{incolor}44}]:} \PY{c+c1}{\PYZsh{} Plot the results}
         \PY{n}{plt}\PY{o}{.}\PY{n}{figure}\PY{p}{(}\PY{p}{)}
         \PY{n}{plt}\PY{o}{.}\PY{n}{scatter}\PY{p}{(}\PY{n}{y\PYZus{}train}\PY{p}{,} \PY{n}{dtr\PYZus{}model}\PY{o}{.}\PY{n}{predict}\PY{p}{(}\PY{n}{X\PYZus{}train}\PY{p}{)}\PY{p}{,} \PY{n}{edgecolor}\PY{o}{=}\PY{l+s+s2}{\PYZdq{}}\PY{l+s+s2}{black}\PY{l+s+s2}{\PYZdq{}}\PY{p}{,}\PY{n}{c}\PY{o}{=}\PY{l+s+s2}{\PYZdq{}}\PY{l+s+s2}{darkorange}\PY{l+s+s2}{\PYZdq{}}\PY{p}{,} \PY{n}{label}\PY{o}{=}\PY{l+s+s2}{\PYZdq{}}\PY{l+s+s2}{Treinamento}\PY{l+s+s2}{\PYZdq{}}\PY{p}{,}\PY{n}{marker}\PY{o}{=}\PY{l+s+s2}{\PYZdq{}}\PY{l+s+s2}{+}\PY{l+s+s2}{\PYZdq{}}\PY{p}{)}
         \PY{n}{plt}\PY{o}{.}\PY{n}{scatter}\PY{p}{(}\PY{n}{y\PYZus{}test}\PY{p}{,} \PY{n}{dtr\PYZus{}model}\PY{o}{.}\PY{n}{predict}\PY{p}{(}\PY{n}{X\PYZus{}test}\PY{p}{)}\PY{p}{,} \PY{n}{edgecolor}\PY{o}{=}\PY{l+s+s2}{\PYZdq{}}\PY{l+s+s2}{black}\PY{l+s+s2}{\PYZdq{}}\PY{p}{,}\PY{n}{c}\PY{o}{=}\PY{l+s+s2}{\PYZdq{}}\PY{l+s+s2}{blue}\PY{l+s+s2}{\PYZdq{}}\PY{p}{,} \PY{n}{label}\PY{o}{=}\PY{l+s+s2}{\PYZdq{}}\PY{l+s+s2}{Teste}\PY{l+s+s2}{\PYZdq{}}\PY{p}{,}\PY{n}{marker}\PY{o}{=}\PY{l+s+s2}{\PYZdq{}}\PY{l+s+s2}{\PYZus{}}\PY{l+s+s2}{\PYZdq{}}\PY{p}{)}
         \PY{n}{plt}\PY{o}{.}\PY{n}{xlabel}\PY{p}{(}\PY{l+s+s2}{\PYZdq{}}\PY{l+s+s2}{Dados}\PY{l+s+s2}{\PYZdq{}}\PY{p}{)}
         \PY{n}{plt}\PY{o}{.}\PY{n}{ylabel}\PY{p}{(}\PY{l+s+s2}{\PYZdq{}}\PY{l+s+s2}{Predição}\PY{l+s+s2}{\PYZdq{}}\PY{p}{)}
         \PY{n}{plt}\PY{o}{.}\PY{n}{title}\PY{p}{(}\PY{l+s+s2}{\PYZdq{}}\PY{l+s+s2}{Regressão de árvore de Decisão}\PY{l+s+s2}{\PYZdq{}}\PY{p}{)}
         \PY{n}{plt}\PY{o}{.}\PY{n}{legend}\PY{p}{(}\PY{p}{)}
         \PY{n}{plt}\PY{o}{.}\PY{n}{show}\PY{p}{(}\PY{p}{)}
\end{Verbatim}


    \begin{center}
    \adjustimage{max size={0.9\linewidth}{0.9\paperheight}}{output_88_0.png}
    \end{center}
    { \hspace*{\fill} \\}
    
    \subsubsection{Regressão Ridge}\label{regressuxe3o-ridge}

    \begin{Verbatim}[commandchars=\\\{\}]
{\color{incolor}In [{\color{incolor}45}]:} \PY{k+kn}{from} \PY{n+nn}{sklearn} \PY{k}{import} \PY{n}{linear\PYZus{}model}
         
         \PY{n}{lrr\PYZus{}model} \PY{o}{=} \PY{n}{linear\PYZus{}model}\PY{o}{.}\PY{n}{Ridge}\PY{p}{(}\PY{n}{alpha}\PY{o}{=}\PY{l+m+mf}{0.1}\PY{p}{)}\PY{o}{.}\PY{n}{fit}\PY{p}{(}\PY{n}{X\PYZus{}train}\PY{p}{,} \PY{n}{y\PYZus{}train}\PY{o}{.}\PY{n}{ravel}\PY{p}{(}\PY{p}{)}\PY{p}{)}
\end{Verbatim}


    Obténha a pontuação R2, utilizando a função \texttt{pontuacao} para os
conjuntos de treino e de teste.

    \begin{Verbatim}[commandchars=\\\{\}]
{\color{incolor}In [{\color{incolor}46}]:} \PY{n+nb}{print}\PY{p}{(}\PY{l+s+s2}{\PYZdq{}}\PY{l+s+s2}{Pontuação R2 para os dados de treinamento:}\PY{l+s+s2}{\PYZdq{}}\PY{p}{)}
         \PY{n}{pontuacao}\PY{p}{(}\PY{n}{lrr\PYZus{}model}\PY{p}{,} \PY{n}{X\PYZus{}train}\PY{p}{,} \PY{n}{y\PYZus{}train}\PY{p}{)}
         \PY{n+nb}{print}\PY{p}{(}\PY{l+s+s2}{\PYZdq{}}\PY{l+s+se}{\PYZbs{}n}\PY{l+s+s2}{\PYZdq{}}\PY{p}{)}
         \PY{n+nb}{print}\PY{p}{(}\PY{l+s+s2}{\PYZdq{}}\PY{l+s+s2}{Pontuação R2 para os dados de test:}\PY{l+s+s2}{\PYZdq{}}\PY{p}{)}
         \PY{n}{pontuacao}\PY{p}{(}\PY{n}{lrr\PYZus{}model}\PY{p}{,} \PY{n}{X\PYZus{}test}\PY{p}{,} \PY{n}{y\PYZus{}test}\PY{p}{)}
\end{Verbatim}


    \begin{Verbatim}[commandchars=\\\{\}]
Pontuação R2 para os dados de treinamento:
Ridge(alpha=0.1, copy\_X=True, fit\_intercept=True, max\_iter=None,
   normalize=False, random\_state=None, solver='auto', tol=0.001) R2 score: 0.84


Pontuação R2 para os dados de test:
Ridge(alpha=0.1, copy\_X=True, fit\_intercept=True, max\_iter=None,
   normalize=False, random\_state=None, solver='auto', tol=0.001) R2 score: 0.77

    \end{Verbatim}

    \begin{Verbatim}[commandchars=\\\{\}]
{\color{incolor}In [{\color{incolor}47}]:} \PY{c+c1}{\PYZsh{} Plot the results}
         \PY{n}{plt}\PY{o}{.}\PY{n}{figure}\PY{p}{(}\PY{p}{)}
         \PY{n}{plt}\PY{o}{.}\PY{n}{scatter}\PY{p}{(}\PY{n}{y\PYZus{}train}\PY{p}{,} \PY{n}{lrr\PYZus{}model}\PY{o}{.}\PY{n}{predict}\PY{p}{(}\PY{n}{X\PYZus{}train}\PY{p}{)}\PY{p}{,} \PY{n}{edgecolor}\PY{o}{=}\PY{l+s+s2}{\PYZdq{}}\PY{l+s+s2}{black}\PY{l+s+s2}{\PYZdq{}}\PY{p}{,}\PY{n}{c}\PY{o}{=}\PY{l+s+s2}{\PYZdq{}}\PY{l+s+s2}{darkorange}\PY{l+s+s2}{\PYZdq{}}\PY{p}{,} \PY{n}{label}\PY{o}{=}\PY{l+s+s2}{\PYZdq{}}\PY{l+s+s2}{Treinamento}\PY{l+s+s2}{\PYZdq{}}\PY{p}{,}\PY{n}{marker}\PY{o}{=}\PY{l+s+s2}{\PYZdq{}}\PY{l+s+s2}{+}\PY{l+s+s2}{\PYZdq{}}\PY{p}{)}
         \PY{n}{plt}\PY{o}{.}\PY{n}{scatter}\PY{p}{(}\PY{n}{y\PYZus{}test}\PY{p}{,} \PY{n}{lrr\PYZus{}model}\PY{o}{.}\PY{n}{predict}\PY{p}{(}\PY{n}{X\PYZus{}test}\PY{p}{)}\PY{p}{,} \PY{n}{edgecolor}\PY{o}{=}\PY{l+s+s2}{\PYZdq{}}\PY{l+s+s2}{black}\PY{l+s+s2}{\PYZdq{}}\PY{p}{,}\PY{n}{c}\PY{o}{=}\PY{l+s+s2}{\PYZdq{}}\PY{l+s+s2}{blue}\PY{l+s+s2}{\PYZdq{}}\PY{p}{,} \PY{n}{label}\PY{o}{=}\PY{l+s+s2}{\PYZdq{}}\PY{l+s+s2}{Teste}\PY{l+s+s2}{\PYZdq{}}\PY{p}{,}\PY{n}{marker}\PY{o}{=}\PY{l+s+s2}{\PYZdq{}}\PY{l+s+s2}{\PYZus{}}\PY{l+s+s2}{\PYZdq{}}\PY{p}{)}
         \PY{n}{plt}\PY{o}{.}\PY{n}{xlabel}\PY{p}{(}\PY{l+s+s2}{\PYZdq{}}\PY{l+s+s2}{Dados}\PY{l+s+s2}{\PYZdq{}}\PY{p}{)}
         \PY{n}{plt}\PY{o}{.}\PY{n}{ylabel}\PY{p}{(}\PY{l+s+s2}{\PYZdq{}}\PY{l+s+s2}{Predição}\PY{l+s+s2}{\PYZdq{}}\PY{p}{)}
         \PY{n}{plt}\PY{o}{.}\PY{n}{title}\PY{p}{(}\PY{l+s+s2}{\PYZdq{}}\PY{l+s+s2}{Regressão Ridge}\PY{l+s+s2}{\PYZdq{}}\PY{p}{)}
         \PY{n}{plt}\PY{o}{.}\PY{n}{legend}\PY{p}{(}\PY{p}{)}
         \PY{n}{plt}\PY{o}{.}\PY{n}{show}\PY{p}{(}\PY{p}{)}
\end{Verbatim}


    \begin{center}
    \adjustimage{max size={0.9\linewidth}{0.9\paperheight}}{output_93_0.png}
    \end{center}
    { \hspace*{\fill} \\}
    
    \textbf{Pergunta:} Nesta análise preliminar, como foi a performance de
cada algoritmo? Explique os resultados e detalhe como a característica
de cada algoritmo influenciou no resultado.

\textbf{Resposta:} Os algoritmos de Regressão Linear e Regressão Ridge
chegaram aos exatos mesmos resultados, mesmo para a base de testes. Já a
Regressão por Árvore de Decisão apesar de apresentar caracteristicas de
overfitting, obteve o melhor resultado dos três, tanto nos dados de
treino e quanto nos de teste. Como a Regressão Linear e a Regressão
Ridge possuem semelhanças ao utilizar coeficientes regressores afins de
diminuir a complexidade do modelo, ambas procuram suavizar pesos e
eliminar atributos muito correlacionados, assim dado que na base de
treino escolhida, 8/11 variaveis tem mais de 50\% de correlação enquando
3/11 tem mais de 50\% correlação negativa, ambos algoritmos
possivelmente foram prejudicados na seleção. No caso do algoritmo de
Regressão por Árvore de Decisão, como ele estima o valor desejado
utilizando regras de decisão inferidas pelos atributos, a caracteristica
de alta correlação das variaveis da base de treino levaram a uma melhor
performance pela árvore.

    \subsection{Validação e Otimização do
Modelo}\label{validauxe7uxe3o-e-otimizauxe7uxe3o-do-modelo}

Cada algoritmo de modelo pode oferecer a possibilidade de ajustes de
seus parâmetros. Tais ajustes podem contribuir para melhorar ou piorar o
modelo, portanto esta fase de otimização e validação é importante
entender o patamar de partida, com os valores padrões obtidos nos passos
anteriores versus as otimizações.

É comum que as otimizações não sejam capazes de alterar os patamares
padrão.

O Scikit Learn oferece uma forma simplificada de testar diversas
condições de parâmetros diferentes por meio do \texttt{GridSearchCV}.
Após diversos testes é apresentado os parâmetros que obtiveram os
melhores resultados.

    \subsubsection{Regulação dos parâmetros dos
modelos}\label{regulauxe7uxe3o-dos-paruxe2metros-dos-modelos}

Analise os parâmetros possíveis de cada algortimo de regressão e crie um
dicionário para ser utilizado no \texttt{GridSearchCV}. O dicionário é
composto pelo nome do parâmetro como chave. Seu valor será uma lista de
valores que se deseja otimizar. Não deixe de revisar a
\href{http://scikit-learn.org/stable/modules/generated/sklearn.model_selection.GridSearchCV.html}{documentação}

Verfique a pontuação R2 para os conjuntos de dados de treino e de teste,
pois nem sempre preditores que se saem bem durante o treinamento terão a
mesma performance com os dados de teste.

    \subsubsection{Regulação do Modelo de Regressão
Linear}\label{regulauxe7uxe3o-do-modelo-de-regressuxe3o-linear}

Escolha quais atributos incluir na variável \texttt{parameters}para
serem otimizados. Essa variável é um dicionário onde cada chave
representa uma configuração do modelo, o valor pode ser único ou uma
lista, neste caso utilize \texttt{{[}{]}} para incluir múltiplos
valores. Como nosso problema é relacionado a regressão, utilize a
pontuação R2 em \texttt{scoring} na configuração do
\texttt{GridSearchCV}.

    \begin{Verbatim}[commandchars=\\\{\}]
{\color{incolor}In [{\color{incolor}48}]:} \PY{k+kn}{from} \PY{n+nn}{sklearn}\PY{n+nn}{.}\PY{n+nn}{model\PYZus{}selection} \PY{k}{import} \PY{n}{GridSearchCV}
         
         \PY{n+nb}{print}\PY{p}{(}\PY{n}{lr\PYZus{}model}\PY{o}{.}\PY{n}{get\PYZus{}params}\PY{p}{(}\PY{p}{)}\PY{p}{)}
         \PY{n}{parameters} \PY{o}{=} \PY{p}{\PYZob{}}\PY{l+s+s1}{\PYZsq{}}\PY{l+s+s1}{fit\PYZus{}intercept}\PY{l+s+s1}{\PYZsq{}}\PY{p}{:} \PY{p}{[}\PY{k+kc}{True}\PY{p}{,} \PY{k+kc}{False}\PY{p}{]}\PY{p}{,}
                       \PY{l+s+s1}{\PYZsq{}}\PY{l+s+s1}{normalize}\PY{l+s+s1}{\PYZsq{}}\PY{p}{:} \PY{p}{[}\PY{k+kc}{True}\PY{p}{,} \PY{k+kc}{False}\PY{p}{]}\PY{p}{\PYZcb{}}
         
         \PY{n}{opt\PYZus{}model\PYZus{}lr} \PY{o}{=} \PY{n}{GridSearchCV}\PY{p}{(}\PY{n}{lr\PYZus{}model}\PY{p}{,} \PY{n}{parameters}\PY{p}{,} \PY{n}{scoring}\PY{o}{=}\PY{l+s+s1}{\PYZsq{}}\PY{l+s+s1}{r2}\PY{l+s+s1}{\PYZsq{}}\PY{p}{)}
         \PY{n}{opt\PYZus{}model\PYZus{}lr}\PY{o}{.}\PY{n}{fit}\PY{p}{(}\PY{n}{X\PYZus{}train}\PY{p}{,} \PY{n}{y\PYZus{}train}\PY{o}{.}\PY{n}{ravel}\PY{p}{(}\PY{p}{)}\PY{p}{)}
\end{Verbatim}


    \begin{Verbatim}[commandchars=\\\{\}]
\{'copy\_X': True, 'fit\_intercept': True, 'n\_jobs': 1, 'normalize': True\}

    \end{Verbatim}

\begin{Verbatim}[commandchars=\\\{\}]
{\color{outcolor}Out[{\color{outcolor}48}]:} GridSearchCV(cv=None, error\_score='raise',
                estimator=LinearRegression(copy\_X=True, fit\_intercept=True, n\_jobs=1, normalize=True),
                fit\_params=None, iid=True, n\_jobs=1,
                param\_grid=\{'fit\_intercept': [True, False], 'normalize': [True, False]\},
                pre\_dispatch='2*n\_jobs', refit=True, return\_train\_score='warn',
                scoring='r2', verbose=0)
\end{Verbatim}
            
    Calcule as pontuações para o melhor estimador com dados de treino.

    \begin{Verbatim}[commandchars=\\\{\}]
{\color{incolor}In [{\color{incolor}49}]:} \PY{n}{opt\PYZus{}model\PYZus{}lr}\PY{o}{.}\PY{n}{score}\PY{p}{(}\PY{n}{X\PYZus{}train}\PY{p}{,} \PY{n}{y\PYZus{}train}\PY{o}{.}\PY{n}{ravel}\PY{p}{(}\PY{p}{)}\PY{p}{)}\PY{o}{.}\PY{n}{round}\PY{p}{(}\PY{l+m+mi}{2}\PY{p}{)}\PY{o}{*}\PY{l+m+mi}{100}\PY{p}{,} \PY{n}{opt\PYZus{}model\PYZus{}lr}\PY{o}{.}\PY{n}{best\PYZus{}estimator\PYZus{}}
\end{Verbatim}


\begin{Verbatim}[commandchars=\\\{\}]
{\color{outcolor}Out[{\color{outcolor}49}]:} (84.0,
          LinearRegression(copy\_X=True, fit\_intercept=True, n\_jobs=1, normalize=False))
\end{Verbatim}
            
    E também para os dados de testes.

    \begin{Verbatim}[commandchars=\\\{\}]
{\color{incolor}In [{\color{incolor}50}]:} \PY{n}{opt\PYZus{}model\PYZus{}lr}\PY{o}{.}\PY{n}{score}\PY{p}{(}\PY{n}{X\PYZus{}test}\PY{p}{,} \PY{n}{y\PYZus{}test}\PY{o}{.}\PY{n}{ravel}\PY{p}{(}\PY{p}{)}\PY{p}{)}\PY{o}{.}\PY{n}{round}\PY{p}{(}\PY{l+m+mi}{2}\PY{p}{)}\PY{o}{*}\PY{l+m+mi}{100}\PY{p}{,} \PY{n}{opt\PYZus{}model\PYZus{}lr}\PY{o}{.}\PY{n}{best\PYZus{}estimator\PYZus{}}
\end{Verbatim}


\begin{Verbatim}[commandchars=\\\{\}]
{\color{outcolor}Out[{\color{outcolor}50}]:} (77.0,
          LinearRegression(copy\_X=True, fit\_intercept=True, n\_jobs=1, normalize=False))
\end{Verbatim}
            
    \subsubsection{Regulação do Modelo de Regressão de Árvore de
Decisão}\label{regulauxe7uxe3o-do-modelo-de-regressuxe3o-de-uxe1rvore-de-decisuxe3o}

    \begin{Verbatim}[commandchars=\\\{\}]
{\color{incolor}In [{\color{incolor}51}]:} \PY{n+nb}{print}\PY{p}{(}\PY{n}{dtr\PYZus{}model}\PY{o}{.}\PY{n}{get\PYZus{}params}\PY{p}{(}\PY{p}{)}\PY{p}{)}
         \PY{n}{parameters} \PY{o}{=} \PY{p}{\PYZob{}}\PY{l+s+s1}{\PYZsq{}}\PY{l+s+s1}{max\PYZus{}depth}\PY{l+s+s1}{\PYZsq{}}\PY{p}{:} \PY{n}{np}\PY{o}{.}\PY{n}{arange}\PY{p}{(}\PY{l+m+mi}{1}\PY{p}{,}\PY{l+m+mi}{10}\PY{p}{)}\PY{p}{,}
                       \PY{l+s+s1}{\PYZsq{}}\PY{l+s+s1}{max\PYZus{}features}\PY{l+s+s1}{\PYZsq{}}\PY{p}{:} \PY{n}{np}\PY{o}{.}\PY{n}{arange}\PY{p}{(}\PY{l+m+mi}{1}\PY{p}{,} \PY{n}{X\PYZus{}train}\PY{o}{.}\PY{n}{shape}\PY{p}{[}\PY{l+m+mi}{1}\PY{p}{]}\PY{o}{+}\PY{l+m+mi}{1}\PY{p}{)}\PY{p}{,}
                       \PY{l+s+s1}{\PYZsq{}}\PY{l+s+s1}{min\PYZus{}samples\PYZus{}split}\PY{l+s+s1}{\PYZsq{}}\PY{p}{:} \PY{n}{np}\PY{o}{.}\PY{n}{arange}\PY{p}{(}\PY{l+m+mf}{0.1}\PY{p}{,} \PY{l+m+mi}{1}\PY{p}{,} \PY{l+m+mf}{0.1}\PY{p}{)}\PY{p}{\PYZcb{}}
         
         \PY{n}{opt\PYZus{}model\PYZus{}dtr} \PY{o}{=} \PY{n}{GridSearchCV}\PY{p}{(}\PY{n}{dtr\PYZus{}model}\PY{p}{,} \PY{n}{parameters}\PY{p}{,} \PY{n}{scoring}\PY{o}{=}\PY{l+s+s1}{\PYZsq{}}\PY{l+s+s1}{r2}\PY{l+s+s1}{\PYZsq{}}\PY{p}{)}
         \PY{n}{opt\PYZus{}model\PYZus{}dtr}\PY{o}{.}\PY{n}{fit}\PY{p}{(}\PY{n}{X\PYZus{}train}\PY{p}{,} \PY{n}{y\PYZus{}train}\PY{o}{.}\PY{n}{ravel}\PY{p}{(}\PY{p}{)}\PY{p}{)}
\end{Verbatim}


    \begin{Verbatim}[commandchars=\\\{\}]
\{'criterion': 'mse', 'max\_depth': None, 'max\_features': None, 'max\_leaf\_nodes': None, 'min\_impurity\_decrease': 0.0, 'min\_impurity\_split': None, 'min\_samples\_leaf': 1, 'min\_samples\_split': 2, 'min\_weight\_fraction\_leaf': 0.0, 'presort': False, 'random\_state': 6, 'splitter': 'best'\}

    \end{Verbatim}

\begin{Verbatim}[commandchars=\\\{\}]
{\color{outcolor}Out[{\color{outcolor}51}]:} GridSearchCV(cv=None, error\_score='raise',
                estimator=DecisionTreeRegressor(criterion='mse', max\_depth=None, max\_features=None,
                    max\_leaf\_nodes=None, min\_impurity\_decrease=0.0,
                    min\_impurity\_split=None, min\_samples\_leaf=1,
                    min\_samples\_split=2, min\_weight\_fraction\_leaf=0.0,
                    presort=False, random\_state=6, splitter='best'),
                fit\_params=None, iid=True, n\_jobs=1,
                param\_grid=\{'max\_depth': array([1, 2, 3, 4, 5, 6, 7, 8, 9]), 'max\_features': array([ 1,  2,  3,  4,  5,  6,  7,  8,  9, 10, 11]), 'min\_samples\_split': array([ 0.1,  0.2,  0.3,  0.4,  0.5,  0.6,  0.7,  0.8,  0.9])\},
                pre\_dispatch='2*n\_jobs', refit=True, return\_train\_score='warn',
                scoring='r2', verbose=0)
\end{Verbatim}
            
    Pontuação dos dados de treino.

    \begin{Verbatim}[commandchars=\\\{\}]
{\color{incolor}In [{\color{incolor}52}]:} \PY{n}{opt\PYZus{}model\PYZus{}dtr}\PY{o}{.}\PY{n}{score}\PY{p}{(}\PY{n}{X\PYZus{}train}\PY{p}{,} \PY{n}{y\PYZus{}train}\PY{o}{.}\PY{n}{ravel}\PY{p}{(}\PY{p}{)}\PY{p}{)}\PY{o}{.}\PY{n}{round}\PY{p}{(}\PY{l+m+mi}{2}\PY{p}{)}\PY{o}{*}\PY{l+m+mi}{100}\PY{p}{,} \PY{n}{opt\PYZus{}model\PYZus{}dtr}\PY{o}{.}\PY{n}{best\PYZus{}estimator\PYZus{}}
\end{Verbatim}


\begin{Verbatim}[commandchars=\\\{\}]
{\color{outcolor}Out[{\color{outcolor}52}]:} (92.0, DecisionTreeRegressor(criterion='mse', max\_depth=8, max\_features=8,
                     max\_leaf\_nodes=None, min\_impurity\_decrease=0.0,
                     min\_impurity\_split=None, min\_samples\_leaf=1,
                     min\_samples\_split=0.10000000000000001,
                     min\_weight\_fraction\_leaf=0.0, presort=False, random\_state=6,
                     splitter='best'))
\end{Verbatim}
            
    Pontuação dos dados de teste.

    \begin{Verbatim}[commandchars=\\\{\}]
{\color{incolor}In [{\color{incolor}53}]:} \PY{n}{opt\PYZus{}model\PYZus{}dtr}\PY{o}{.}\PY{n}{score}\PY{p}{(}\PY{n}{X\PYZus{}test}\PY{p}{,} \PY{n}{y\PYZus{}test}\PY{o}{.}\PY{n}{ravel}\PY{p}{(}\PY{p}{)}\PY{p}{)}\PY{o}{.}\PY{n}{round}\PY{p}{(}\PY{l+m+mi}{2}\PY{p}{)}\PY{o}{*}\PY{l+m+mi}{100}\PY{p}{,} \PY{n}{opt\PYZus{}model\PYZus{}dtr}\PY{o}{.}\PY{n}{best\PYZus{}estimator\PYZus{}}
\end{Verbatim}


\begin{Verbatim}[commandchars=\\\{\}]
{\color{outcolor}Out[{\color{outcolor}53}]:} (93.0, DecisionTreeRegressor(criterion='mse', max\_depth=8, max\_features=8,
                     max\_leaf\_nodes=None, min\_impurity\_decrease=0.0,
                     min\_impurity\_split=None, min\_samples\_leaf=1,
                     min\_samples\_split=0.10000000000000001,
                     min\_weight\_fraction\_leaf=0.0, presort=False, random\_state=6,
                     splitter='best'))
\end{Verbatim}
            
    \subsubsection{Regulação do Modelo de Regressão
Ridge}\label{regulauxe7uxe3o-do-modelo-de-regressuxe3o-ridge}

    \begin{Verbatim}[commandchars=\\\{\}]
{\color{incolor}In [{\color{incolor}54}]:} \PY{n+nb}{print}\PY{p}{(}\PY{n}{lrr\PYZus{}model}\PY{o}{.}\PY{n}{get\PYZus{}params}\PY{p}{(}\PY{p}{)}\PY{p}{)}
         \PY{n}{parameters} \PY{o}{=} \PY{p}{\PYZob{}}\PY{l+s+s1}{\PYZsq{}}\PY{l+s+s1}{normalize}\PY{l+s+s1}{\PYZsq{}}\PY{p}{:} \PY{p}{[}\PY{k+kc}{True}\PY{p}{,} \PY{k+kc}{False}\PY{p}{]}\PY{p}{,}
                       \PY{l+s+s1}{\PYZsq{}}\PY{l+s+s1}{fit\PYZus{}intercept}\PY{l+s+s1}{\PYZsq{}}\PY{p}{:} \PY{p}{[}\PY{k+kc}{True}\PY{p}{,} \PY{k+kc}{False}\PY{p}{]}\PY{p}{,}
                       \PY{l+s+s1}{\PYZsq{}}\PY{l+s+s1}{alpha}\PY{l+s+s1}{\PYZsq{}}\PY{p}{:} \PY{n}{np}\PY{o}{.}\PY{n}{array}\PY{p}{(}\PY{p}{[}\PY{l+m+mi}{1}\PY{p}{,}\PY{l+m+mf}{0.1}\PY{p}{,}\PY{l+m+mf}{0.01}\PY{p}{,}\PY{l+m+mf}{0.001}\PY{p}{,}\PY{l+m+mf}{0.0001}\PY{p}{,}\PY{l+m+mi}{0}\PY{p}{]}\PY{p}{)}\PY{p}{\PYZcb{}}
         
         \PY{n}{opt\PYZus{}model\PYZus{}lrr} \PY{o}{=} \PY{n}{GridSearchCV}\PY{p}{(}\PY{n}{lrr\PYZus{}model}\PY{p}{,} \PY{n}{parameters}\PY{p}{,} \PY{n}{scoring}\PY{o}{=}\PY{l+s+s1}{\PYZsq{}}\PY{l+s+s1}{r2}\PY{l+s+s1}{\PYZsq{}}\PY{p}{)}
         \PY{n}{opt\PYZus{}model\PYZus{}lrr}\PY{o}{.}\PY{n}{fit}\PY{p}{(}\PY{n}{X\PYZus{}train}\PY{p}{,} \PY{n}{y\PYZus{}train}\PY{o}{.}\PY{n}{ravel}\PY{p}{(}\PY{p}{)}\PY{p}{)}
\end{Verbatim}


    \begin{Verbatim}[commandchars=\\\{\}]
\{'alpha': 0.1, 'copy\_X': True, 'fit\_intercept': True, 'max\_iter': None, 'normalize': False, 'random\_state': None, 'solver': 'auto', 'tol': 0.001\}

    \end{Verbatim}

\begin{Verbatim}[commandchars=\\\{\}]
{\color{outcolor}Out[{\color{outcolor}54}]:} GridSearchCV(cv=None, error\_score='raise',
                estimator=Ridge(alpha=0.1, copy\_X=True, fit\_intercept=True, max\_iter=None,
            normalize=False, random\_state=None, solver='auto', tol=0.001),
                fit\_params=None, iid=True, n\_jobs=1,
                param\_grid=\{'normalize': [True, False], 'fit\_intercept': [True, False], 'alpha': array([  1.00000e+00,   1.00000e-01,   1.00000e-02,   1.00000e-03,
                  1.00000e-04,   0.00000e+00])\},
                pre\_dispatch='2*n\_jobs', refit=True, return\_train\_score='warn',
                scoring='r2', verbose=0)
\end{Verbatim}
            
    Pontuação dos dados de treino.

    \begin{Verbatim}[commandchars=\\\{\}]
{\color{incolor}In [{\color{incolor}55}]:} \PY{n}{opt\PYZus{}model\PYZus{}lrr}\PY{o}{.}\PY{n}{score}\PY{p}{(}\PY{n}{X\PYZus{}train}\PY{p}{,} \PY{n}{y\PYZus{}train}\PY{o}{.}\PY{n}{ravel}\PY{p}{(}\PY{p}{)}\PY{p}{)}\PY{o}{.}\PY{n}{round}\PY{p}{(}\PY{l+m+mi}{2}\PY{p}{)}\PY{o}{*}\PY{l+m+mi}{100}\PY{p}{,} \PY{n}{opt\PYZus{}model\PYZus{}lrr}\PY{o}{.}\PY{n}{best\PYZus{}estimator\PYZus{}}
\end{Verbatim}


\begin{Verbatim}[commandchars=\\\{\}]
{\color{outcolor}Out[{\color{outcolor}55}]:} (84.0, Ridge(alpha=0.10000000000000001, copy\_X=True, fit\_intercept=True,
             max\_iter=None, normalize=True, random\_state=None, solver='auto',
             tol=0.001))
\end{Verbatim}
            
    Pontuação dos dados de teste.

    \begin{Verbatim}[commandchars=\\\{\}]
{\color{incolor}In [{\color{incolor}56}]:} \PY{n}{opt\PYZus{}model\PYZus{}lrr}\PY{o}{.}\PY{n}{score}\PY{p}{(}\PY{n}{X\PYZus{}test}\PY{p}{,} \PY{n}{y\PYZus{}test}\PY{o}{.}\PY{n}{ravel}\PY{p}{(}\PY{p}{)}\PY{p}{)}\PY{o}{.}\PY{n}{round}\PY{p}{(}\PY{l+m+mi}{2}\PY{p}{)}\PY{o}{*}\PY{l+m+mi}{100}\PY{p}{,} \PY{n}{opt\PYZus{}model\PYZus{}lrr}\PY{o}{.}\PY{n}{best\PYZus{}estimator\PYZus{}}
\end{Verbatim}


\begin{Verbatim}[commandchars=\\\{\}]
{\color{outcolor}Out[{\color{outcolor}56}]:} (77.0, Ridge(alpha=0.10000000000000001, copy\_X=True, fit\_intercept=True,
             max\_iter=None, normalize=True, random\_state=None, solver='auto',
             tol=0.001))
\end{Verbatim}
            
    Sumarize na tabela abaixo os indicadores para cada um dos preditores e
suas respectivas pontuações para os conjuntos de dados de treino e de
testes

    \begin{longtable}[]{@{}llll@{}}
\toprule
& Regressão linear & Regressão árvore de decisão & Regressão
ridge\tabularnewline
\midrule
\endhead
R2 treino & 84\% & 92\% & 84\%\tabularnewline
R2 teste & 77\% & 93\% & 77\%\tabularnewline
\bottomrule
\end{longtable}

    \textbf{Pergunta:} Qual dos algoritmos de regressão obteve os melhores
resultados? Quais caracaterísticas deste algoritmo podem ajudar a
justificar tal resultado?

\textbf{Resposta:} O algoritmo de regressão que obteve o melhor
resultado foi a Árvore de Decisão com uma acuracidade de 93\% no subset
de teste. Nesta etapa apesar da regulação dos modelos, os algoritmos de
Regressão Linear e Regressão Ridge mantiveram suas pontuações R2,
indicando que dada a base de treino altamente correlacionada, alcançaram
sua melhor performance. Para o algoritmo de Regressão por Árvore de
Decisão com a regularização do overfitting houve melhora na sua
performance com a base de teste.

    \subsubsection{Implementação do algoritmo
otimizado}\label{implementauxe7uxe3o-do-algoritmo-otimizado}

Configure o classificador selecionado com os parâmetros otimizados
obtidos anteriormente.

    \begin{Verbatim}[commandchars=\\\{\}]
{\color{incolor}In [{\color{incolor}57}]:} \PY{k+kn}{from} \PY{n+nn}{sklearn}\PY{n+nn}{.}\PY{n+nn}{tree} \PY{k}{import} \PY{n}{DecisionTreeRegressor}
         
         \PY{n}{dtr\PYZus{}model\PYZus{}tunned} \PY{o}{=} \PY{n}{DecisionTreeRegressor}\PY{p}{(}\PY{n}{random\PYZus{}state}\PY{o}{=}\PY{l+m+mi}{6}\PY{p}{,} \PY{n}{max\PYZus{}depth}\PY{o}{=}\PY{l+m+mi}{8}\PY{p}{,} \PY{n}{max\PYZus{}features}\PY{o}{=}\PY{l+m+mi}{8}\PY{p}{,} \PY{n}{min\PYZus{}samples\PYZus{}split}\PY{o}{=}\PY{l+m+mf}{0.1}\PY{p}{)}\PY{o}{.}\PY{n}{fit}\PY{p}{(}\PY{n}{X\PYZus{}train}\PY{p}{,} \PY{n}{y\PYZus{}train}\PY{o}{.}\PY{n}{ravel}\PY{p}{(}\PY{p}{)}\PY{p}{)}
\end{Verbatim}


    \subsubsection{Teste com exemplos}\label{teste-com-exemplos}

Utilize 3 exemplos criados por você mesmo para obter um valor de venda
de veículo. Escolha caracaterísticas que demonstrem como o regressor
deveria se comportar. Para tanto, imagine exemplos dos quais você espera
um valor baixo, mediano e alto do preço do veículo baseado nos atributos
escolhidos

    \begin{Verbatim}[commandchars=\\\{\}]
{\color{incolor}In [{\color{incolor}58}]:} \PY{c+c1}{\PYZsh{}Lembre\PYZhy{}se que os atributos são os armazenados na lista feature\PYZus{}col\PYZus{}names}
         
         \PY{n}{feature\PYZus{}col\PYZus{}names}
\end{Verbatim}


\begin{Verbatim}[commandchars=\\\{\}]
{\color{outcolor}Out[{\color{outcolor}58}]:} ['drive\_wheels',
          'wheel\_base',
          'length',
          'width',
          'curb\_weight',
          'number\_of\_cylinders',
          'engine\_size',
          'bore',
          'horsepower',
          'city\_mpg',
          'highway\_mpg']
\end{Verbatim}
            
    \begin{Verbatim}[commandchars=\\\{\}]
{\color{incolor}In [{\color{incolor}59}]:} \PY{c+c1}{\PYZsh{}A entrada de dados deve ser uma matriz do seguinte formato, note que há dois colchetes pois é uma }
         \PY{c+c1}{\PYZsh{}matriz dentro de outra matriz [[a,b,c,d,e]]}
         
         \PY{n}{test\PYZus{}example\PYZus{}1} \PY{o}{=} \PY{p}{[}\PY{p}{[}\PY{l+m+mi}{1}\PY{p}{,}  \PY{l+m+mi}{95}\PY{p}{,} \PY{l+m+mi}{150}\PY{p}{,} \PY{l+m+mi}{62}\PY{p}{,} \PY{l+m+mi}{2000}\PY{p}{,} \PY{l+m+mi}{4}\PY{p}{,}  \PY{l+m+mi}{90}\PY{p}{,} \PY{l+m+mf}{3.0}\PY{p}{,}  \PY{l+m+mi}{50}\PY{p}{,} \PY{l+m+mi}{37}\PY{p}{,} \PY{l+m+mi}{44}\PY{p}{]}\PY{p}{]}
         \PY{n}{test\PYZus{}example\PYZus{}2} \PY{o}{=} \PY{p}{[}\PY{p}{[}\PY{l+m+mi}{1}\PY{p}{,} \PY{l+m+mi}{100}\PY{p}{,} \PY{l+m+mi}{180}\PY{p}{,} \PY{l+m+mi}{66}\PY{p}{,} \PY{l+m+mi}{2500}\PY{p}{,} \PY{l+m+mi}{4}\PY{p}{,} \PY{l+m+mi}{120}\PY{p}{,} \PY{l+m+mf}{3.5}\PY{p}{,} \PY{l+m+mi}{120}\PY{p}{,} \PY{l+m+mi}{26}\PY{p}{,} \PY{l+m+mi}{30}\PY{p}{]}\PY{p}{]}
         \PY{n}{test\PYZus{}example\PYZus{}3} \PY{o}{=} \PY{p}{[}\PY{p}{[}\PY{l+m+mi}{2}\PY{p}{,} \PY{l+m+mi}{110}\PY{p}{,} \PY{l+m+mi}{200}\PY{p}{,} \PY{l+m+mi}{70}\PY{p}{,} \PY{l+m+mi}{3600}\PY{p}{,} \PY{l+m+mi}{6}\PY{p}{,} \PY{l+m+mi}{200}\PY{p}{,} \PY{l+m+mf}{3.7}\PY{p}{,} \PY{l+m+mi}{200}\PY{p}{,} \PY{l+m+mi}{15}\PY{p}{,} \PY{l+m+mi}{18}\PY{p}{]}\PY{p}{]}
         
         \PY{n}{predicted\PYZus{}price1} \PY{o}{=} \PY{n}{dtr\PYZus{}model\PYZus{}tunned}\PY{o}{.}\PY{n}{predict}\PY{p}{(}\PY{n}{test\PYZus{}example\PYZus{}1}\PY{p}{)}
         \PY{n+nb}{print}\PY{p}{(}\PY{l+s+s2}{\PYZdq{}}\PY{l+s+s2}{O preço previsto do valor de venda do exemplo 1 é: }\PY{l+s+si}{\PYZpc{}.2f}\PY{l+s+s2}{\PYZdq{}} \PY{o}{\PYZpc{}}\PY{p}{(}\PY{n}{predicted\PYZus{}price1}\PY{p}{)}\PY{p}{)}
         
         \PY{n}{predicted\PYZus{}price2} \PY{o}{=} \PY{n}{dtr\PYZus{}model\PYZus{}tunned}\PY{o}{.}\PY{n}{predict}\PY{p}{(}\PY{n}{test\PYZus{}example\PYZus{}2}\PY{p}{)}
         \PY{n+nb}{print}\PY{p}{(}\PY{l+s+s2}{\PYZdq{}}\PY{l+s+s2}{O preço previsto do valor de venda do exemplo 2 é: }\PY{l+s+si}{\PYZpc{}.2f}\PY{l+s+s2}{\PYZdq{}} \PY{o}{\PYZpc{}}\PY{p}{(}\PY{n}{predicted\PYZus{}price2}\PY{p}{)}\PY{p}{)}
         
         \PY{n}{predicted\PYZus{}price3} \PY{o}{=} \PY{n}{dtr\PYZus{}model\PYZus{}tunned}\PY{o}{.}\PY{n}{predict}\PY{p}{(}\PY{n}{test\PYZus{}example\PYZus{}3}\PY{p}{)}
         \PY{n+nb}{print}\PY{p}{(}\PY{l+s+s2}{\PYZdq{}}\PY{l+s+s2}{O preço previsto do valor de venda do exemplo 3 é: }\PY{l+s+si}{\PYZpc{}.2f}\PY{l+s+s2}{\PYZdq{}} \PY{o}{\PYZpc{}}\PY{p}{(}\PY{n}{predicted\PYZus{}price3}\PY{p}{)}\PY{p}{)}
\end{Verbatim}


    \begin{Verbatim}[commandchars=\\\{\}]
O preço previsto do valor de venda do exemplo 1 é: 5955.00
O preço previsto do valor de venda do exemplo 2 é: 10737.50
O preço previsto do valor de venda do exemplo 3 é: 34778.57

    \end{Verbatim}

    \subsection{Conclusões finais}\label{conclusuxf5es-finais}

Este projeto apresentou de forma simplificada o \textbf{Worflow de
Machine Learning} que pode servir como base para estudos relacionados a
classificação ou predição de séries numéricas.

A fase de preparação, evidenciada no projeto, é uma das mais importantes
da qual se precisa investir um bom tempo para dar os dados organizados e
confiáveis, pois é a forma como os classificadores irão aprender com os
exemplos e tentarão desvencilhar de efeitos indesejáveis como os vieses.

Regressores são um tipo de algoritmo de machine learning que pode ser
aplicado em diversas áreas das quais é necessário predizer um número
baseado em um conjunto de dados numéricos ou série numérica, logo sua
aplicação é bem ampla.

    \textbf{Pergunta:} Seu modelo conseguiu prever adequadamente novos dados
a partir do treinamento dos dados de teste? O que você faria diferente?

\textbf{Resposta:} Sim. O modelo final com o algoritmo escolhido (árvore
de regressão) apresentou uma boa acuracidade no dataset de teste (93\%)
e conseguiu prever de forma satisfatória o preço dos 3 exemplos criados,
sendo que pelo fato da acuracidade dos datasets de treinamento e de
teste terem sido próximas é possível concluir que o modelo consegue
generealizar (não está com overfitting) e ser aplicado para os novos
dados que ele receba.

Acreditamos que a aplicação de um modelo em conjunto (emsemble) iria
produzir um resultado mais satisfatório. No caso, seria o uso do
algoritmo de Random Forest, o qual é composto de várias árvores de
decisão atuando em conjunto e provavelmente apresentaria melhor
acurácia.

    \textbf{Pergunta:}: Em que outras áreas você poderia aplicar estes tipos
de algoritmos de regressão de aprendizado de máquina?

\textbf{Resposta}: Esses algoritmos de regressão de aprendizado de
máquina podem ser aplicados em diversas áreas do conhecimento, haja
visto que a previsão de valores numéricos com base em dados passados é
uma característica para a maioria das indústrias. Alguns exemplos de
áreas de aplicações são: Mercado de ações - previsão dos valores de
ativos; Mercado imobiliário - previsão dos valores dos imóveis.


    % Add a bibliography block to the postdoc
    
    
    
    \end{document}
